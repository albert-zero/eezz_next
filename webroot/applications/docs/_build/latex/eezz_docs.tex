%% Generated by Sphinx.
\def\sphinxdocclass{report}
\documentclass[letterpaper,10pt,english]{sphinxmanual}
\ifdefined\pdfpxdimen
   \let\sphinxpxdimen\pdfpxdimen\else\newdimen\sphinxpxdimen
\fi \sphinxpxdimen=.75bp\relax
\ifdefined\pdfimageresolution
    \pdfimageresolution= \numexpr \dimexpr1in\relax/\sphinxpxdimen\relax
\fi
%% let collapsible pdf bookmarks panel have high depth per default
\PassOptionsToPackage{bookmarksdepth=5}{hyperref}
%% turn off hyperref patch of \index as sphinx.xdy xindy module takes care of
%% suitable \hyperpage mark-up, working around hyperref-xindy incompatibility
\PassOptionsToPackage{hyperindex=false}{hyperref}
%% memoir class requires extra handling
\makeatletter\@ifclassloaded{memoir}
{\ifdefined\memhyperindexfalse\memhyperindexfalse\fi}{}\makeatother

\PassOptionsToPackage{booktabs}{sphinx}
\PassOptionsToPackage{colorrows}{sphinx}

\PassOptionsToPackage{warn}{textcomp}

\catcode`^^^^00a0\active\protected\def^^^^00a0{\leavevmode\nobreak\ }
\usepackage{cmap}
\usepackage{fontspec}
\defaultfontfeatures[\rmfamily,\sffamily,\ttfamily]{}
\usepackage{amsmath,amssymb,amstext}
\usepackage{polyglossia}
\setmainlanguage{english}



\setmainfont{DejaVu Serif}
\setsansfont{DejaVu Sans}
\setmonofont{DejaVu Sans Mono}



\usepackage[Bjornstrup]{fncychap}
\usepackage{sphinx}

\fvset{fontsize=\small}
\usepackage{geometry}


% Include hyperref last.
\usepackage{hyperref}
% Fix anchor placement for figures with captions.
\usepackage{hypcap}% it must be loaded after hyperref.
% Set up styles of URL: it should be placed after hyperref.
\urlstyle{same}

\addto\captionsenglish{\renewcommand{\contentsname}{Contents:}}

\usepackage{sphinxmessages}
\setcounter{tocdepth}{1}


\usepackage[titles]{tocloft}
\cftsetpnumwidth {1.25cm}\cftsetrmarg{1.5cm}
\setlength{\cftchapnumwidth}{0.75cm}
\setlength{\cftsecindent}{\cftchapnumwidth}
\setlength{\cftsecnumwidth}{1.25cm}


\title{EEZZ}
\date{Feb 15, 2024}
\release{1.0}
\author{Albert Zedlitz}
\newcommand{\sphinxlogo}{\vbox{}}
\renewcommand{\releasename}{Release}
\makeindex
\begin{document}

\pagestyle{empty}
\sphinxmaketitle
\pagestyle{plain}
\sphinxtableofcontents
\pagestyle{normal}
\phantomsection\label{\detokenize{index::doc}}


\sphinxAtStartPar
EEZZ provides a bidirectional user interface for Python to an HTML browser using
WEB\sphinxhyphen{}sockets, which allows you to develop software in two galvanic seperated threads.

\sphinxAtStartPar
This project is free software: you can redistribute it and/or modify
it under the terms of the GNU General Public License as published by
the Free Software Foundation, either version 3 of the License, or
(at your option) any later version.

\sphinxAtStartPar
This program is distributed in the hope that it will be useful,
but WITHOUT ANY WARRANTY; without even the implied warranty of
MERCHANTABILITY or FITNESS FOR A PARTICULAR PURPOSE.  See the
GNU General Public License for more details.

\sphinxAtStartPar
You should have received a copy of the GNU General Public License
along with this program.  If not, see <\sphinxurl{http://www.gnu.org/licenses/}>.

\sphinxstepscope


\chapter{eezz}
\label{\detokenize{modules:eezz}}\label{\detokenize{modules::doc}}
\sphinxstepscope


\section{eezz package}
\label{\detokenize{eezz:eezz-package}}\label{\detokenize{eezz::doc}}

\subsection{Submodules}
\label{\detokenize{eezz:submodules}}

\subsection{eezz.blueserv module}
\label{\detokenize{eezz:module-eezz.blueserv}}\label{\detokenize{eezz:eezz-blueserv-module}}\index{module@\spxentry{module}!eezz.blueserv@\spxentry{eezz.blueserv}}\index{eezz.blueserv@\spxentry{eezz.blueserv}!module@\spxentry{module}}
\sphinxAtStartPar
This module implements the following classes
\begin{itemize}
\item {} 
\sphinxAtStartPar
\sphinxstylestrong{TBluetooth}:        TTable for listing bluetooth devices in range

\item {} 
\sphinxAtStartPar
\sphinxstylestrong{TBluetoothService}: Communicates with bluetooth\sphinxhyphen{}service EEZZ on mobile device

\end{itemize}
\index{TBluetooth (class in eezz.blueserv)@\spxentry{TBluetooth}\spxextra{class in eezz.blueserv}}

\begin{savenotes}\begin{fulllineitems}
\phantomsection\label{\detokenize{eezz:eezz.blueserv.TBluetooth}}
\pysigstartsignatures
\pysiglinewithargsret{\sphinxbfcode{\sphinxupquote{class\DUrole{w}{ }}}\sphinxcode{\sphinxupquote{eezz.blueserv.}}\sphinxbfcode{\sphinxupquote{TBluetooth}}}{\sphinxparam{\DUrole{n}{*}}\sphinxparamcomma \sphinxparam{\DUrole{n}{column\_names: \textasciitilde{}typing.List{[}str{]} = None}}\sphinxparamcomma \sphinxparam{\DUrole{n}{column\_names\_map: \textasciitilde{}typing.Dict{[}str}}\sphinxparamcomma \sphinxparam{\DUrole{n}{\textasciitilde{}table.TTableCell{]} | None = None}}\sphinxparamcomma \sphinxparam{\DUrole{n}{column\_names\_alias: \textasciitilde{}typing.Dict{[}str}}\sphinxparamcomma \sphinxparam{\DUrole{n}{str{]} | None = None}}\sphinxparamcomma \sphinxparam{\DUrole{n}{column\_names\_filter: \textasciitilde{}typing.List{[}int{]} | None = None}}\sphinxparamcomma \sphinxparam{\DUrole{n}{column\_descr: \textasciitilde{}typing.List{[}\textasciitilde{}table.TTableColumn{]} = None}}\sphinxparamcomma \sphinxparam{\DUrole{n}{table\_index: \textasciitilde{}typing.Dict{[}str}}\sphinxparamcomma \sphinxparam{\DUrole{n}{\textasciitilde{}table.TTableRow{]} = None}}\sphinxparamcomma \sphinxparam{\DUrole{n}{title: str = 'Table'}}\sphinxparamcomma \sphinxparam{\DUrole{n}{attrs: dict = None}}\sphinxparamcomma \sphinxparam{\DUrole{n}{visible\_items: int = 20}}\sphinxparamcomma \sphinxparam{\DUrole{n}{offset: int = 0}}\sphinxparamcomma \sphinxparam{\DUrole{n}{selected\_row: \textasciitilde{}table.TTableRow = None}}\sphinxparamcomma \sphinxparam{\DUrole{n}{header\_row: \textasciitilde{}table.TTableRow = None}}\sphinxparamcomma \sphinxparam{\DUrole{n}{apply\_filter\_column: bool = False}}\sphinxparamcomma \sphinxparam{\DUrole{n}{format\_types: dict = None}}\sphinxparamcomma \sphinxparam{\DUrole{n}{async\_condition: \textasciitilde{}threading.Condition = <Condition(<unlocked \_thread.RLock object owner=0 count=0>}}\sphinxparamcomma \sphinxparam{\DUrole{n}{0)>}}\sphinxparamcomma \sphinxparam{\DUrole{n}{async\_lock: \textasciitilde{}\_thread.allocate\_lock = <unlocked \_thread.lock object>}}}{}
\pysigstopsignatures
\sphinxAtStartPar
Bases: \sphinxcode{\sphinxupquote{TTable}}

\sphinxAtStartPar
The bluetooth class manages bluetooth devices in range
A scan\_thread is started to keep looking for new devices.
TBluetooth service is a singleton to manage this consistently
\begin{quote}\begin{description}
\sphinxlineitem{Parameters}\begin{itemize}
\item {} 
\sphinxAtStartPar
\sphinxstyleliteralstrong{\sphinxupquote{bt\_table\_changed}} – Condition triggered, if there are changes in the list of devices in range

\item {} 
\sphinxAtStartPar
\sphinxstyleliteralstrong{\sphinxupquote{column\_names}} – Fixed column names {[}‘Address’, ‘Name’{]}

\end{itemize}

\end{description}\end{quote}
\index{find\_devices() (eezz.blueserv.TBluetooth method)@\spxentry{find\_devices()}\spxextra{eezz.blueserv.TBluetooth method}}

\begin{savenotes}\begin{fulllineitems}
\phantomsection\label{\detokenize{eezz:eezz.blueserv.TBluetooth.find_devices}}
\pysigstartsignatures
\pysiglinewithargsret{\sphinxbfcode{\sphinxupquote{find\_devices}}}{}{{ $\rightarrow$ None}}
\pysigstopsignatures
\sphinxAtStartPar
This method is called frequently by thread \sphinxtitleref{self.scan:bluetooth} to keep track of devices in range.
The table is checked for new devices or devices which went out of range. Only if the list changes the
condition TTable.async\_condition is triggered with notify\_all.

\end{fulllineitems}\end{savenotes}

\index{get\_visible\_rows() (eezz.blueserv.TBluetooth method)@\spxentry{get\_visible\_rows()}\spxextra{eezz.blueserv.TBluetooth method}}

\begin{savenotes}\begin{fulllineitems}
\phantomsection\label{\detokenize{eezz:eezz.blueserv.TBluetooth.get_visible_rows}}
\pysigstartsignatures
\pysiglinewithargsret{\sphinxbfcode{\sphinxupquote{get\_visible\_rows}}}{\sphinxparam{\DUrole{n}{get\_all}\DUrole{p}{:}\DUrole{w}{ }\DUrole{n}{bool}\DUrole{w}{ }\DUrole{o}{=}\DUrole{w}{ }\DUrole{default_value}{False}}}{{ $\rightarrow$ List\DUrole{p}{{[}}{\hyperref[\detokenize{eezz:eezz.table.TTableRow}]{\sphinxcrossref{TTableRow}}}\DUrole{p}{{]}}}}
\pysigstopsignatures
\sphinxAtStartPar
Return the visible rows at the current cursor
\begin{quote}\begin{description}
\sphinxlineitem{Parameters}
\sphinxAtStartPar
\sphinxstyleliteralstrong{\sphinxupquote{get\_all}} – A bool value to overwrite the visible\_items for the current call

\sphinxlineitem{Returns}
\sphinxAtStartPar
A list of visible row items

\end{description}\end{quote}

\end{fulllineitems}\end{savenotes}


\end{fulllineitems}\end{savenotes}

\index{TBluetoothService (class in eezz.blueserv)@\spxentry{TBluetoothService}\spxextra{class in eezz.blueserv}}

\begin{savenotes}\begin{fulllineitems}
\phantomsection\label{\detokenize{eezz:eezz.blueserv.TBluetoothService}}
\pysigstartsignatures
\pysiglinewithargsret{\sphinxbfcode{\sphinxupquote{class\DUrole{w}{ }}}\sphinxcode{\sphinxupquote{eezz.blueserv.}}\sphinxbfcode{\sphinxupquote{TBluetoothService}}}{\sphinxparam{\DUrole{n}{*}}\sphinxparamcomma \sphinxparam{\DUrole{n}{address: str}}\sphinxparamcomma \sphinxparam{\DUrole{n}{eezz\_service: str = '07e30214\sphinxhyphen{}406b\sphinxhyphen{}11e3\sphinxhyphen{}8770\sphinxhyphen{}84a6c8168ea0'}}\sphinxparamcomma \sphinxparam{\DUrole{n}{m\_lock: \textasciitilde{}\_thread.allocate\_lock = <unlocked \_thread.lock object>}}\sphinxparamcomma \sphinxparam{\DUrole{n}{bt\_service: list = None}}\sphinxparamcomma \sphinxparam{\DUrole{n}{connected: bool = False}}}{}
\pysigstopsignatures
\sphinxAtStartPar
Bases: \sphinxcode{\sphinxupquote{object}}

\sphinxAtStartPar
The TBluetoothService handles a connection for a specific mobile given by bt\_address
\begin{quote}\begin{description}
\sphinxlineitem{Parameters}\begin{itemize}
\item {} 
\sphinxAtStartPar
\sphinxstyleliteralstrong{\sphinxupquote{address}} – The address of the bluetooth device

\item {} 
\sphinxAtStartPar
\sphinxstyleliteralstrong{\sphinxupquote{eezz\_service}} – The service GUID of the eezz App

\item {} 
\sphinxAtStartPar
\sphinxstyleliteralstrong{\sphinxupquote{m\_lock}} – Lock communication for a single request/response cycle

\item {} 
\sphinxAtStartPar
\sphinxstyleliteralstrong{\sphinxupquote{bt\_socket}} – The communication socket

\item {} 
\sphinxAtStartPar
\sphinxstyleliteralstrong{\sphinxupquote{bt\_service}} – The services associated with the eezz App

\item {} 
\sphinxAtStartPar
\sphinxstyleliteralstrong{\sphinxupquote{connected}} – Indicates that the service is active

\end{itemize}

\end{description}\end{quote}
\index{open\_connection() (eezz.blueserv.TBluetoothService method)@\spxentry{open\_connection()}\spxextra{eezz.blueserv.TBluetoothService method}}

\begin{savenotes}\begin{fulllineitems}
\phantomsection\label{\detokenize{eezz:eezz.blueserv.TBluetoothService.open_connection}}
\pysigstartsignatures
\pysiglinewithargsret{\sphinxbfcode{\sphinxupquote{open\_connection}}}{}{}
\pysigstopsignatures
\sphinxAtStartPar
Open a bluetooth connection

\end{fulllineitems}\end{savenotes}

\index{send\_request() (eezz.blueserv.TBluetoothService method)@\spxentry{send\_request()}\spxextra{eezz.blueserv.TBluetoothService method}}

\begin{savenotes}\begin{fulllineitems}
\phantomsection\label{\detokenize{eezz:eezz.blueserv.TBluetoothService.send_request}}
\pysigstartsignatures
\pysiglinewithargsret{\sphinxbfcode{\sphinxupquote{send\_request}}}{\sphinxparam{\DUrole{n}{command}\DUrole{p}{:}\DUrole{w}{ }\DUrole{n}{str}}\sphinxparamcomma \sphinxparam{\DUrole{n}{args}\DUrole{p}{:}\DUrole{w}{ }\DUrole{n}{list}}}{{ $\rightarrow$ dict}}
\pysigstopsignatures
\sphinxAtStartPar
A request is send to the device, waiting for a response.
The protocol use EEZZ\sphinxhyphen{}JSON structure:
send \sphinxhyphen{}>    \{message: str, args: list\}
receive \sphinxhyphen{}> \{return: dict \{ code: int, value: str\}, …\}
\begin{quote}\begin{description}
\sphinxlineitem{Parameters}\begin{itemize}
\item {} 
\sphinxAtStartPar
\sphinxstyleliteralstrong{\sphinxupquote{command}} – The command to execute

\item {} 
\sphinxAtStartPar
\sphinxstyleliteralstrong{\sphinxupquote{args}} – The arguments for the given command

\end{itemize}

\sphinxlineitem{Returns}
\sphinxAtStartPar
JSON structure send by device

\end{description}\end{quote}

\end{fulllineitems}\end{savenotes}

\index{shutdown() (eezz.blueserv.TBluetoothService method)@\spxentry{shutdown()}\spxextra{eezz.blueserv.TBluetoothService method}}

\begin{savenotes}\begin{fulllineitems}
\phantomsection\label{\detokenize{eezz:eezz.blueserv.TBluetoothService.shutdown}}
\pysigstartsignatures
\pysiglinewithargsret{\sphinxbfcode{\sphinxupquote{shutdown}}}{}{}
\pysigstopsignatures
\sphinxAtStartPar
Close a bluetooth connection

\end{fulllineitems}\end{savenotes}


\end{fulllineitems}\end{savenotes}



\subsection{eezz.database module}
\label{\detokenize{eezz:module-eezz.database}}\label{\detokenize{eezz:eezz-database-module}}\index{module@\spxentry{module}!eezz.database@\spxentry{eezz.database}}\index{eezz.database@\spxentry{eezz.database}!module@\spxentry{module}}\begin{itemize}
\item {} 
\sphinxAtStartPar
\sphinxstylestrong{TDatabaseTable}:   Create database from scratch. Encapsulate database access.

\item {} 
\sphinxAtStartPar
\sphinxstylestrong{TDatabaseColumn}:  Extends the TTableColumn by parameters, which are relevant only for database access

\end{itemize}
\index{TDatabaseColumn (class in eezz.database)@\spxentry{TDatabaseColumn}\spxextra{class in eezz.database}}

\begin{savenotes}\begin{fulllineitems}
\phantomsection\label{\detokenize{eezz:eezz.database.TDatabaseColumn}}
\pysigstartsignatures
\pysiglinewithargsret{\sphinxbfcode{\sphinxupquote{class\DUrole{w}{ }}}\sphinxcode{\sphinxupquote{eezz.database.}}\sphinxbfcode{\sphinxupquote{TDatabaseColumn}}}{\sphinxparam{\DUrole{o}{*}}\sphinxparamcomma \sphinxparam{\DUrole{n}{index}\DUrole{p}{:}\DUrole{w}{ }\DUrole{n}{int}}\sphinxparamcomma \sphinxparam{\DUrole{n}{header}\DUrole{p}{:}\DUrole{w}{ }\DUrole{n}{str}}\sphinxparamcomma \sphinxparam{\DUrole{n}{width}\DUrole{p}{:}\DUrole{w}{ }\DUrole{n}{int}\DUrole{w}{ }\DUrole{o}{=}\DUrole{w}{ }\DUrole{default_value}{10}}\sphinxparamcomma \sphinxparam{\DUrole{n}{filter}\DUrole{p}{:}\DUrole{w}{ }\DUrole{n}{str}\DUrole{w}{ }\DUrole{o}{=}\DUrole{w}{ }\DUrole{default_value}{''}}\sphinxparamcomma \sphinxparam{\DUrole{n}{sort}\DUrole{p}{:}\DUrole{w}{ }\DUrole{n}{bool}\DUrole{w}{ }\DUrole{o}{=}\DUrole{w}{ }\DUrole{default_value}{True}}\sphinxparamcomma \sphinxparam{\DUrole{n}{type}\DUrole{p}{:}\DUrole{w}{ }\DUrole{n}{str}\DUrole{w}{ }\DUrole{o}{=}\DUrole{w}{ }\DUrole{default_value}{''}}\sphinxparamcomma \sphinxparam{\DUrole{n}{attrs}\DUrole{p}{:}\DUrole{w}{ }\DUrole{n}{dict}\DUrole{w}{ }\DUrole{o}{=}\DUrole{w}{ }\DUrole{default_value}{None}}\sphinxparamcomma \sphinxparam{\DUrole{n}{primary\_key}\DUrole{p}{:}\DUrole{w}{ }\DUrole{n}{bool}\DUrole{w}{ }\DUrole{o}{=}\DUrole{w}{ }\DUrole{default_value}{False}}\sphinxparamcomma \sphinxparam{\DUrole{n}{options}\DUrole{p}{:}\DUrole{w}{ }\DUrole{n}{str}\DUrole{w}{ }\DUrole{o}{=}\DUrole{w}{ }\DUrole{default_value}{''}}\sphinxparamcomma \sphinxparam{\DUrole{n}{alias}\DUrole{p}{:}\DUrole{w}{ }\DUrole{n}{str}\DUrole{w}{ }\DUrole{o}{=}\DUrole{w}{ }\DUrole{default_value}{''}}}{}
\pysigstopsignatures
\sphinxAtStartPar
Bases: \sphinxcode{\sphinxupquote{TTableColumn}}

\sphinxAtStartPar
Extension for column descriptor TTableColumn
\begin{quote}\begin{description}
\sphinxlineitem{Parameters}\begin{itemize}
\item {} 
\sphinxAtStartPar
\sphinxstyleliteralstrong{\sphinxupquote{primary\_key}} – Makes a column a primary key. In TTable the row\sphinxhyphen{}id is calculated as SHA256 hash on primary key values.

\item {} 
\sphinxAtStartPar
\sphinxstyleliteralstrong{\sphinxupquote{options}} – Database option for column creation (e.g. not null).  \sphinxcode{\sphinxupquote{create table ... column text not null}}

\item {} 
\sphinxAtStartPar
\sphinxstyleliteralstrong{\sphinxupquote{alias}} – Name for the column in prepared statements:  \sphinxcode{\sphinxupquote{insert <column>... values(:<alias>, ...)}}

\end{itemize}

\end{description}\end{quote}

\end{fulllineitems}\end{savenotes}

\index{TDatabaseTable (class in eezz.database)@\spxentry{TDatabaseTable}\spxextra{class in eezz.database}}

\begin{savenotes}\begin{fulllineitems}
\phantomsection\label{\detokenize{eezz:eezz.database.TDatabaseTable}}
\pysigstartsignatures
\pysiglinewithargsret{\sphinxbfcode{\sphinxupquote{class\DUrole{w}{ }}}\sphinxcode{\sphinxupquote{eezz.database.}}\sphinxbfcode{\sphinxupquote{TDatabaseTable}}}{\sphinxparam{\DUrole{n}{*}}\sphinxparamcomma \sphinxparam{\DUrole{n}{column\_names: list = None}}\sphinxparamcomma \sphinxparam{\DUrole{n}{column\_names\_map: \textasciitilde{}typing.Dict{[}str}}\sphinxparamcomma \sphinxparam{\DUrole{n}{\textasciitilde{}table.TTableCell{]} | None = None}}\sphinxparamcomma \sphinxparam{\DUrole{n}{column\_names\_alias: \textasciitilde{}typing.Dict{[}str}}\sphinxparamcomma \sphinxparam{\DUrole{n}{str{]} | None = None}}\sphinxparamcomma \sphinxparam{\DUrole{n}{column\_names\_filter: \textasciitilde{}typing.List{[}int{]} | None = None}}\sphinxparamcomma \sphinxparam{\DUrole{n}{column\_descr: \textasciitilde{}typing.List{[}\textasciitilde{}eezz.database.TDatabaseColumn{]} = None}}\sphinxparamcomma \sphinxparam{\DUrole{n}{table\_index: \textasciitilde{}typing.Dict{[}str}}\sphinxparamcomma \sphinxparam{\DUrole{n}{\textasciitilde{}table.TTableRow{]} = None}}\sphinxparamcomma \sphinxparam{\DUrole{n}{title: str = 'Table'}}\sphinxparamcomma \sphinxparam{\DUrole{n}{attrs: dict = None}}\sphinxparamcomma \sphinxparam{\DUrole{n}{visible\_items: int = 20}}\sphinxparamcomma \sphinxparam{\DUrole{n}{offset: int = 0}}\sphinxparamcomma \sphinxparam{\DUrole{n}{selected\_row: \textasciitilde{}table.TTableRow = None}}\sphinxparamcomma \sphinxparam{\DUrole{n}{header\_row: \textasciitilde{}table.TTableRow = None}}\sphinxparamcomma \sphinxparam{\DUrole{n}{apply\_filter\_column: bool = False}}\sphinxparamcomma \sphinxparam{\DUrole{n}{format\_types: dict = None}}\sphinxparamcomma \sphinxparam{\DUrole{n}{async\_condition: \textasciitilde{}threading.Condition = <Condition(<unlocked \_thread.RLock object owner=0 count=0>}}\sphinxparamcomma \sphinxparam{\DUrole{n}{0)>}}\sphinxparamcomma \sphinxparam{\DUrole{n}{async\_lock: \textasciitilde{}\_thread.allocate\_lock = <unlocked \_thread.lock object>}}\sphinxparamcomma \sphinxparam{\DUrole{n}{statement\_select: str = None}}\sphinxparamcomma \sphinxparam{\DUrole{n}{statement\_count: str = None}}\sphinxparamcomma \sphinxparam{\DUrole{n}{statement\_create: str = None}}\sphinxparamcomma \sphinxparam{\DUrole{n}{statement\_insert: str = None}}\sphinxparamcomma \sphinxparam{\DUrole{n}{database\_path: \textasciitilde{}pathlib.Path = None}}\sphinxparamcomma \sphinxparam{\DUrole{n}{virtual\_len: int = 0}}\sphinxparamcomma \sphinxparam{\DUrole{n}{is\_synchron: bool = False}}}{}
\pysigstopsignatures
\sphinxAtStartPar
Bases: \sphinxcode{\sphinxupquote{TTable}}

\sphinxAtStartPar
General database management
Purpose of this class is a sophisticate work with database using an internal cache. All database
operations are mapped to TTable. The column descriptor is used to generate the database table.
The database results are restricted to the visible scope
Any sort of data is launched to the database
Only the first select statement is executed. For a new buffer, set member is\_synchron to False
\begin{quote}\begin{description}
\sphinxlineitem{Parameters}\begin{itemize}
\item {} 
\sphinxAtStartPar
\sphinxstyleliteralstrong{\sphinxupquote{statement\_select}} – Select statement, inserting limit and offset according to TTable settings:
\sphinxcode{\sphinxupquote{select <TTable.column\_names> from <TTable.title>... limit <TTable.visible\_items>... offset <TTable.offset>... where...}}

\item {} 
\sphinxAtStartPar
\sphinxstyleliteralstrong{\sphinxupquote{statement\_count}} – Evaluates the number of elements in the database 
\sphinxcode{\sphinxupquote{select count (*) ...}}

\item {} 
\sphinxAtStartPar
\sphinxstyleliteralstrong{\sphinxupquote{statement\_create}} – Create statement for database table:
\sphinxcode{\sphinxupquote{create <TTable.title> <{[}List of TTable.column\_names{]}> ... primary keys <{[}list of TDatabaseColumn.primary\_key{]}>}}

\end{itemize}

\end{description}\end{quote}
\index{append() (eezz.database.TDatabaseTable method)@\spxentry{append()}\spxextra{eezz.database.TDatabaseTable method}}

\begin{savenotes}\begin{fulllineitems}
\phantomsection\label{\detokenize{eezz:eezz.database.TDatabaseTable.append}}
\pysigstartsignatures
\pysiglinewithargsret{\sphinxbfcode{\sphinxupquote{append}}}{\sphinxparam{\DUrole{n}{table\_row}\DUrole{p}{:}\DUrole{w}{ }\DUrole{n}{list}}\sphinxparamcomma \sphinxparam{\DUrole{n}{attrs}\DUrole{p}{:}\DUrole{w}{ }\DUrole{n}{dict}\DUrole{w}{ }\DUrole{o}{=}\DUrole{w}{ }\DUrole{default_value}{None}}\sphinxparamcomma \sphinxparam{\DUrole{n}{row\_type}\DUrole{p}{:}\DUrole{w}{ }\DUrole{n}{str}\DUrole{w}{ }\DUrole{o}{=}\DUrole{w}{ }\DUrole{default_value}{'body'}}\sphinxparamcomma \sphinxparam{\DUrole{n}{row\_id}\DUrole{p}{:}\DUrole{w}{ }\DUrole{n}{str}\DUrole{w}{ }\DUrole{o}{=}\DUrole{w}{ }\DUrole{default_value}{''}}\sphinxparamcomma \sphinxparam{\DUrole{n}{exists\_ok}\DUrole{p}{:}\DUrole{w}{ }\DUrole{n}{bool}\DUrole{w}{ }\DUrole{o}{=}\DUrole{w}{ }\DUrole{default_value}{True}}}{{ $\rightarrow$ {\hyperref[\detokenize{eezz:eezz.table.TTableRow}]{\sphinxcrossref{TTableRow}}}}}
\pysigstopsignatures
\sphinxAtStartPar
Append data to the internal table, creating a unique row\sphinxhyphen{}key
The row key is generated using the primary key values as comma separated list. You select from list as (no spaces)
do\_select(row\_id = ‘key\_value1,key\_value2,…’)
\begin{quote}\begin{description}
\sphinxlineitem{Parameters}\begin{itemize}
\item {} 
\sphinxAtStartPar
\sphinxstyleliteralstrong{\sphinxupquote{exists\_ok}} – If set to True, do not raise exception, just ignore the appending silently

\item {} 
\sphinxAtStartPar
\sphinxstyleliteralstrong{\sphinxupquote{table\_row}} (\sphinxstyleliteralemphasis{\sphinxupquote{List}}\sphinxstyleliteralemphasis{\sphinxupquote{ of }}\sphinxstyleliteralemphasis{\sphinxupquote{values in correct order to columns}}) – A list of values as row to insert

\item {} 
\sphinxAtStartPar
\sphinxstyleliteralstrong{\sphinxupquote{attrs}} (\sphinxstyleliteralemphasis{\sphinxupquote{Customizable dictionary}}) – Optional attributes for this row

\item {} 
\sphinxAtStartPar
\sphinxstyleliteralstrong{\sphinxupquote{row\_type}} – Row type used to trigger template output

\item {} 
\sphinxAtStartPar
\sphinxstyleliteralstrong{\sphinxupquote{row\_id}} – If not set, the row\sphinxhyphen{}id is calculated from primary key values

\end{itemize}

\end{description}\end{quote}

\end{fulllineitems}\end{savenotes}

\index{commit() (eezz.database.TDatabaseTable method)@\spxentry{commit()}\spxextra{eezz.database.TDatabaseTable method}}

\begin{savenotes}\begin{fulllineitems}
\phantomsection\label{\detokenize{eezz:eezz.database.TDatabaseTable.commit}}
\pysigstartsignatures
\pysiglinewithargsret{\sphinxbfcode{\sphinxupquote{commit}}}{}{}
\pysigstopsignatures
\end{fulllineitems}\end{savenotes}

\index{db\_create() (eezz.database.TDatabaseTable method)@\spxentry{db\_create()}\spxextra{eezz.database.TDatabaseTable method}}

\begin{savenotes}\begin{fulllineitems}
\phantomsection\label{\detokenize{eezz:eezz.database.TDatabaseTable.db_create}}
\pysigstartsignatures
\pysiglinewithargsret{\sphinxbfcode{\sphinxupquote{db\_create}}}{}{{ $\rightarrow$ None}}
\pysigstopsignatures
\sphinxAtStartPar
Create the table on the database

\end{fulllineitems}\end{savenotes}

\index{do\_select() (eezz.database.TDatabaseTable method)@\spxentry{do\_select()}\spxextra{eezz.database.TDatabaseTable method}}

\begin{savenotes}\begin{fulllineitems}
\phantomsection\label{\detokenize{eezz:eezz.database.TDatabaseTable.do_select}}
\pysigstartsignatures
\pysiglinewithargsret{\sphinxbfcode{\sphinxupquote{do\_select}}}{\sphinxparam{\DUrole{n}{filters}\DUrole{p}{:}\DUrole{w}{ }\DUrole{n}{dict}}\sphinxparamcomma \sphinxparam{\DUrole{n}{get\_all}\DUrole{p}{:}\DUrole{w}{ }\DUrole{n}{bool}\DUrole{w}{ }\DUrole{o}{=}\DUrole{w}{ }\DUrole{default_value}{False}}}{{ $\rightarrow$ List\DUrole{p}{{[}}{\hyperref[\detokenize{eezz:eezz.table.TTableRow}]{\sphinxcrossref{TTableRow}}}\DUrole{p}{{]}}}}
\pysigstopsignatures
\sphinxAtStartPar
Works on local buffer, as long as the scope is not changed. If send to database the syntax of the
values have to be adjusted to
\sphinxhref{https://www.sqlitetutorial.net/sqlite-like/}{splite3\sphinxhyphen{}like}%
\begin{footnote}[1]\sphinxAtStartFootnote
\sphinxnolinkurl{https://www.sqlitetutorial.net/sqlite-like/}
%
\end{footnote}

\end{fulllineitems}\end{savenotes}

\index{get\_visible\_rows() (eezz.database.TDatabaseTable method)@\spxentry{get\_visible\_rows()}\spxextra{eezz.database.TDatabaseTable method}}

\begin{savenotes}\begin{fulllineitems}
\phantomsection\label{\detokenize{eezz:eezz.database.TDatabaseTable.get_visible_rows}}
\pysigstartsignatures
\pysiglinewithargsret{\sphinxbfcode{\sphinxupquote{get\_visible\_rows}}}{\sphinxparam{\DUrole{n}{get\_all}\DUrole{o}{=}\DUrole{default_value}{False}}}{}
\pysigstopsignatures
\sphinxAtStartPar
Works on local buffer, as long as the scope is not changed

\end{fulllineitems}\end{savenotes}

\index{navigate() (eezz.database.TDatabaseTable method)@\spxentry{navigate()}\spxextra{eezz.database.TDatabaseTable method}}

\begin{savenotes}\begin{fulllineitems}
\phantomsection\label{\detokenize{eezz:eezz.database.TDatabaseTable.navigate}}
\pysigstartsignatures
\pysiglinewithargsret{\sphinxbfcode{\sphinxupquote{navigate}}}{\sphinxparam{\DUrole{n}{where\_togo}\DUrole{p}{:}\DUrole{w}{ }\DUrole{n}{{\hyperref[\detokenize{eezz:eezz.table.TNavigation}]{\sphinxcrossref{TNavigation}}}}\DUrole{w}{ }\DUrole{o}{=}\DUrole{w}{ }\DUrole{default_value}{TNavigation.NEXT}}\sphinxparamcomma \sphinxparam{\DUrole{n}{position}\DUrole{p}{:}\DUrole{w}{ }\DUrole{n}{int}\DUrole{w}{ }\DUrole{o}{=}\DUrole{w}{ }\DUrole{default_value}{0}}}{{ $\rightarrow$ None}}
\pysigstopsignatures
\sphinxAtStartPar
Navigate in block mode
\begin{quote}\begin{description}
\sphinxlineitem{Parameters}\begin{itemize}
\item {} 
\sphinxAtStartPar
\sphinxstyleliteralstrong{\sphinxupquote{where\_togo}} ({\hyperref[\detokenize{eezz:eezz.table.TNavigation}]{\sphinxcrossref{\sphinxstyleliteralemphasis{\sphinxupquote{TNavigation}}}}}) – Navigation direction

\item {} 
\sphinxAtStartPar
\sphinxstyleliteralstrong{\sphinxupquote{position}} – Position for absolute navigation, ignored in any other case

\end{itemize}

\end{description}\end{quote}

\end{fulllineitems}\end{savenotes}

\index{prepare\_statements() (eezz.database.TDatabaseTable method)@\spxentry{prepare\_statements()}\spxextra{eezz.database.TDatabaseTable method}}

\begin{savenotes}\begin{fulllineitems}
\phantomsection\label{\detokenize{eezz:eezz.database.TDatabaseTable.prepare_statements}}
\pysigstartsignatures
\pysiglinewithargsret{\sphinxbfcode{\sphinxupquote{prepare\_statements}}}{}{}
\pysigstopsignatures
\sphinxAtStartPar
Generate a set of consistent database statements for

\end{fulllineitems}\end{savenotes}


\end{fulllineitems}\end{savenotes}



\subsection{eezz.document module}
\label{\detokenize{eezz:module-eezz.document}}\label{\detokenize{eezz:eezz-document-module}}\index{module@\spxentry{module}!eezz.document@\spxentry{eezz.document}}\index{eezz.document@\spxentry{eezz.document}!module@\spxentry{module}}
\sphinxAtStartPar
This module implements the following classes
\begin{itemize}
\item {} 
\sphinxAtStartPar
{\hyperref[\detokenize{eezz:eezz.document.TManifest}]{\sphinxcrossref{\sphinxcode{\sphinxupquote{eezz.document.TManifest}}}}}:   Document header representation. The header is a dictionary with a given     structure and a defined set of keys and sub\sphinxhyphen{}keys. The manifest defines the database table and access.     The manifest is the structure, which is signed and which is used to identify and verify the document.

\item {} 
\sphinxAtStartPar
\sphinxcode{\sphinxupquote{eezz.document.TDocument}}:  A document consists of more than one file and the manifest. Part of the     document is encrypted. The document key could be used in combination with a mobile device to decrypt the file.

\end{itemize}
\index{TDocuments (class in eezz.document)@\spxentry{TDocuments}\spxextra{class in eezz.document}}

\begin{savenotes}\begin{fulllineitems}
\phantomsection\label{\detokenize{eezz:eezz.document.TDocuments}}
\pysigstartsignatures
\pysiglinewithargsret{\sphinxbfcode{\sphinxupquote{class\DUrole{w}{ }}}\sphinxcode{\sphinxupquote{eezz.document.}}\sphinxbfcode{\sphinxupquote{TDocuments}}}{\sphinxparam{\DUrole{n}{*}}\sphinxparamcomma \sphinxparam{\DUrole{n}{column\_names: list = None}}\sphinxparamcomma \sphinxparam{\DUrole{n}{column\_names\_map: \textasciitilde{}typing.Dict{[}str}}\sphinxparamcomma \sphinxparam{\DUrole{n}{\textasciitilde{}table.TTableCell{]} | None = None}}\sphinxparamcomma \sphinxparam{\DUrole{n}{column\_names\_alias: \textasciitilde{}typing.Dict{[}str}}\sphinxparamcomma \sphinxparam{\DUrole{n}{str{]} | None = None}}\sphinxparamcomma \sphinxparam{\DUrole{n}{column\_names\_filter: \textasciitilde{}typing.List{[}int{]} | None = None}}\sphinxparamcomma \sphinxparam{\DUrole{n}{column\_descr: \textasciitilde{}typing.List{[}\textasciitilde{}database.TDatabaseColumn{]} = None}}\sphinxparamcomma \sphinxparam{\DUrole{n}{table\_index: \textasciitilde{}typing.Dict{[}str}}\sphinxparamcomma \sphinxparam{\DUrole{n}{\textasciitilde{}table.TTableRow{]} = None}}\sphinxparamcomma \sphinxparam{\DUrole{n}{title: str = 'Table'}}\sphinxparamcomma \sphinxparam{\DUrole{n}{attrs: dict = None}}\sphinxparamcomma \sphinxparam{\DUrole{n}{visible\_items: int = 20}}\sphinxparamcomma \sphinxparam{\DUrole{n}{offset: int = 0}}\sphinxparamcomma \sphinxparam{\DUrole{n}{selected\_row: \textasciitilde{}table.TTableRow = None}}\sphinxparamcomma \sphinxparam{\DUrole{n}{header\_row: \textasciitilde{}table.TTableRow = None}}\sphinxparamcomma \sphinxparam{\DUrole{n}{apply\_filter\_column: bool = False}}\sphinxparamcomma \sphinxparam{\DUrole{n}{format\_types: dict = None}}\sphinxparamcomma \sphinxparam{\DUrole{n}{async\_condition: \textasciitilde{}threading.Condition = <Condition(<unlocked \_thread.RLock object owner=0 count=0>}}\sphinxparamcomma \sphinxparam{\DUrole{n}{0)>}}\sphinxparamcomma \sphinxparam{\DUrole{n}{async\_lock: \textasciitilde{}\_thread.allocate\_lock = <unlocked \_thread.lock object>}}\sphinxparamcomma \sphinxparam{\DUrole{n}{statement\_select: str = None}}\sphinxparamcomma \sphinxparam{\DUrole{n}{statement\_count: str = None}}\sphinxparamcomma \sphinxparam{\DUrole{n}{statement\_create: str = None}}\sphinxparamcomma \sphinxparam{\DUrole{n}{statement\_insert: str = None}}\sphinxparamcomma \sphinxparam{\DUrole{n}{database\_path: \textasciitilde{}pathlib.Path = None}}\sphinxparamcomma \sphinxparam{\DUrole{n}{virtual\_len: int = 0}}\sphinxparamcomma \sphinxparam{\DUrole{n}{is\_synchron: bool = False}}\sphinxparamcomma \sphinxparam{\DUrole{n}{header: dict = None}}\sphinxparamcomma \sphinxparam{\DUrole{n}{files\_list: \textasciitilde{}typing.List{[}\textasciitilde{}filesrv.TEezzFile{]} = None}}\sphinxparamcomma \sphinxparam{\DUrole{n}{key: bytes = None}}\sphinxparamcomma \sphinxparam{\DUrole{n}{vector: bytes = None}}\sphinxparamcomma \sphinxparam{\DUrole{n}{description: dict = None}}\sphinxparamcomma \sphinxparam{\DUrole{n}{manifest: \textasciitilde{}eezz.document.TManifest = None}}\sphinxparamcomma \sphinxparam{\DUrole{n}{eezz\_path: \textasciitilde{}pathlib.Path = None}}\sphinxparamcomma \sphinxparam{\DUrole{n}{name: str = None}}}{}
\pysigstopsignatures
\sphinxAtStartPar
Bases: \sphinxcode{\sphinxupquote{TDatabaseTable}}

\sphinxAtStartPar
Manages documents
There are two ways to start the document:
\begin{enumerate}
\sphinxsetlistlabels{\arabic}{enumi}{enumii}{}{.}%
\item {} 
\sphinxAtStartPar
Create a document using prepare\_download, handle\_download and create
As a result the document is zipped in TAR format with a signed manifest. The key for decryption is stored on
the mobile device and on EEZZ

\item {} 
\sphinxAtStartPar
Open a document, reading the manifest. Noe you could check if you have the key on your mobile device, or you
buy the key from EEZZ

\end{enumerate}
\begin{quote}\begin{description}
\sphinxlineitem{Parameters}\begin{itemize}
\item {} 
\sphinxAtStartPar
\sphinxstyleliteralstrong{\sphinxupquote{files\_list}} – List of files

\item {} 
\sphinxAtStartPar
\sphinxstyleliteralstrong{\sphinxupquote{key}} – Document key

\item {} 
\sphinxAtStartPar
\sphinxstyleliteralstrong{\sphinxupquote{vector}} – Document vector

\item {} 
\sphinxAtStartPar
\sphinxstyleliteralstrong{\sphinxupquote{description}} – Document description

\item {} 
\sphinxAtStartPar
\sphinxstyleliteralstrong{\sphinxupquote{manifest}} ({\hyperref[\detokenize{eezz:eezz.document.TManifest}]{\sphinxcrossref{\sphinxstyleliteralemphasis{\sphinxupquote{TManifest}}}}}) – Manifest as document header

\item {} 
\sphinxAtStartPar
\sphinxstyleliteralstrong{\sphinxupquote{eezz\_path}} – Document file name

\item {} 
\sphinxAtStartPar
\sphinxstyleliteralstrong{\sphinxupquote{name}} – Document name

\end{itemize}

\end{description}\end{quote}
\index{add\_document\_to\_device() (eezz.document.TDocuments method)@\spxentry{add\_document\_to\_device()}\spxextra{eezz.document.TDocuments method}}

\begin{savenotes}\begin{fulllineitems}
\phantomsection\label{\detokenize{eezz:eezz.document.TDocuments.add_document_to_device}}
\pysigstartsignatures
\pysiglinewithargsret{\sphinxbfcode{\sphinxupquote{add\_document\_to\_device}}}{}{{ $\rightarrow$ dict}}
\pysigstopsignatures
\end{fulllineitems}\end{savenotes}

\index{create\_document() (eezz.document.TDocuments method)@\spxentry{create\_document()}\spxextra{eezz.document.TDocuments method}}

\begin{savenotes}\begin{fulllineitems}
\phantomsection\label{\detokenize{eezz:eezz.document.TDocuments.create_document}}
\pysigstartsignatures
\pysiglinewithargsret{\sphinxbfcode{\sphinxupquote{create\_document}}}{\sphinxparam{\DUrole{n}{name}\DUrole{p}{:}\DUrole{w}{ }\DUrole{n}{str}}\sphinxparamcomma \sphinxparam{\DUrole{n}{nr\_files}\DUrole{p}{:}\DUrole{w}{ }\DUrole{n}{int}}\sphinxparamcomma \sphinxparam{\DUrole{n}{queue}\DUrole{p}{:}\DUrole{w}{ }\DUrole{n}{Queue\DUrole{p}{{[}}{\hyperref[\detokenize{eezz:eezz.filesrv.TEezzFile}]{\sphinxcrossref{TEezzFile}}}\DUrole{p}{{]}}}}}{{ $\rightarrow$ None}}
\pysigstopsignatures
\sphinxAtStartPar
After all files downloaded, The document header is registered on eezz server and signed
All files are zipped together with this header.
\begin{quote}\begin{description}
\sphinxlineitem{Parameters}\begin{itemize}
\item {} 
\sphinxAtStartPar
\sphinxstyleliteralstrong{\sphinxupquote{name}} – Name of the document on disk

\item {} 
\sphinxAtStartPar
\sphinxstyleliteralstrong{\sphinxupquote{nr\_files}} – Number of files in the queue

\item {} 
\sphinxAtStartPar
\sphinxstyleliteralstrong{\sphinxupquote{queue}} – The process queue

\end{itemize}

\end{description}\end{quote}

\end{fulllineitems}\end{savenotes}

\index{decrypt\_key\_with\_device() (eezz.document.TDocuments method)@\spxentry{decrypt\_key\_with\_device()}\spxextra{eezz.document.TDocuments method}}

\begin{savenotes}\begin{fulllineitems}
\phantomsection\label{\detokenize{eezz:eezz.document.TDocuments.decrypt_key_with_device}}
\pysigstartsignatures
\pysiglinewithargsret{\sphinxbfcode{\sphinxupquote{decrypt\_key\_with\_device}}}{\sphinxparam{\DUrole{n}{encrypted\_key}\DUrole{p}{:}\DUrole{w}{ }\DUrole{n}{bytes}}}{{ $\rightarrow$ bytes\DUrole{w}{ }\DUrole{p}{|}\DUrole{w}{ }None}}
\pysigstopsignatures
\sphinxAtStartPar
Decrypt the document key
\begin{quote}\begin{description}
\sphinxlineitem{Parameters}
\sphinxAtStartPar
\sphinxstyleliteralstrong{\sphinxupquote{encrypted\_key}} – The encrypted key

\sphinxlineitem{Returns}
\sphinxAtStartPar
The decrypted key, vector pair

\end{description}\end{quote}

\end{fulllineitems}\end{savenotes}

\index{eezz\_buy\_document() (eezz.document.TDocuments method)@\spxentry{eezz\_buy\_document()}\spxextra{eezz.document.TDocuments method}}

\begin{savenotes}\begin{fulllineitems}
\phantomsection\label{\detokenize{eezz:eezz.document.TDocuments.eezz_buy_document}}
\pysigstartsignatures
\pysiglinewithargsret{\sphinxbfcode{\sphinxupquote{eezz\_buy\_document}}}{\sphinxparam{\DUrole{n}{transaction\_key}\DUrole{p}{:}\DUrole{w}{ }\DUrole{n}{bytes}}}{}
\pysigstopsignatures
\sphinxAtStartPar
Buy transaction to get the document key.
\begin{quote}\begin{description}
\sphinxlineitem{Parameters}
\sphinxAtStartPar
\sphinxstyleliteralstrong{\sphinxupquote{transaction\_key}} – With the method \sphinxcode{\sphinxupquote{eezz.document.TDocument.eezz\_get\_document\_key()}} you get the         key, if you are owner or you get a transaction\_key, which could be used in this method to buy the key. Once         you own the key, you could store it in local database.         The document key is encrypted with the mobile device key. Using the key requires this device.

\end{description}\end{quote}

\end{fulllineitems}\end{savenotes}

\index{eezz\_get\_document\_key() (eezz.document.TDocuments method)@\spxentry{eezz\_get\_document\_key()}\spxextra{eezz.document.TDocuments method}}

\begin{savenotes}\begin{fulllineitems}
\phantomsection\label{\detokenize{eezz:eezz.document.TDocuments.eezz_get_document_key}}
\pysigstartsignatures
\pysiglinewithargsret{\sphinxbfcode{\sphinxupquote{eezz\_get\_document\_key}}}{\sphinxparam{\DUrole{n}{buy\_request}\DUrole{o}{=}\DUrole{default_value}{False}}}{{ $\rightarrow$ dict}}
\pysigstopsignatures
\end{fulllineitems}\end{savenotes}

\index{eezz\_register\_document() (eezz.document.TDocuments method)@\spxentry{eezz\_register\_document()}\spxextra{eezz.document.TDocuments method}}

\begin{savenotes}\begin{fulllineitems}
\phantomsection\label{\detokenize{eezz:eezz.document.TDocuments.eezz_register_document}}
\pysigstartsignatures
\pysiglinewithargsret{\sphinxbfcode{\sphinxupquote{eezz\_register\_document}}}{}{{ $\rightarrow$ dict}}
\pysigstopsignatures
\sphinxAtStartPar
Registers the document header to EEZZ and returns it signed with the EEZZ key. The signed header
is stored as manifest in the final document.

\end{fulllineitems}\end{savenotes}

\index{encrypt\_key\_with\_device() (eezz.document.TDocuments method)@\spxentry{encrypt\_key\_with\_device()}\spxextra{eezz.document.TDocuments method}}

\begin{savenotes}\begin{fulllineitems}
\phantomsection\label{\detokenize{eezz:eezz.document.TDocuments.encrypt_key_with_device}}
\pysigstartsignatures
\pysiglinewithargsret{\sphinxbfcode{\sphinxupquote{encrypt\_key\_with\_device}}}{\sphinxparam{\DUrole{n}{key}\DUrole{p}{:}\DUrole{w}{ }\DUrole{n}{bytes}}\sphinxparamcomma \sphinxparam{\DUrole{n}{vector}\DUrole{p}{:}\DUrole{w}{ }\DUrole{n}{bytes}}}{{ $\rightarrow$ bytes\DUrole{w}{ }\DUrole{p}{|}\DUrole{w}{ }None}}
\pysigstopsignatures
\sphinxAtStartPar
Encrypt the document key
\begin{quote}\begin{description}
\sphinxlineitem{Parameters}\begin{itemize}
\item {} 
\sphinxAtStartPar
\sphinxstyleliteralstrong{\sphinxupquote{key}} – Document key

\item {} 
\sphinxAtStartPar
\sphinxstyleliteralstrong{\sphinxupquote{vector}} – Document vector

\end{itemize}

\sphinxlineitem{Returns}
\sphinxAtStartPar
encrypted document key

\end{description}\end{quote}

\end{fulllineitems}\end{savenotes}

\index{get\_device\_key() (eezz.document.TDocuments method)@\spxentry{get\_device\_key()}\spxextra{eezz.document.TDocuments method}}

\begin{savenotes}\begin{fulllineitems}
\phantomsection\label{\detokenize{eezz:eezz.document.TDocuments.get_device_key}}
\pysigstartsignatures
\pysiglinewithargsret{\sphinxbfcode{\sphinxupquote{get\_device\_key}}}{}{{ $\rightarrow$ bytes\DUrole{w}{ }\DUrole{p}{|}\DUrole{w}{ }None}}
\pysigstopsignatures
\end{fulllineitems}\end{savenotes}

\index{handle\_download() (eezz.document.TDocuments method)@\spxentry{handle\_download()}\spxextra{eezz.document.TDocuments method}}

\begin{savenotes}\begin{fulllineitems}
\phantomsection\label{\detokenize{eezz:eezz.document.TDocuments.handle_download}}
\pysigstartsignatures
\pysiglinewithargsret{\sphinxbfcode{\sphinxupquote{handle\_download}}}{\sphinxparam{\DUrole{n}{request}\DUrole{p}{:}\DUrole{w}{ }\DUrole{n}{dict}}\sphinxparamcomma \sphinxparam{\DUrole{n}{raw\_data}\DUrole{p}{:}\DUrole{w}{ }\DUrole{n}{Any}}}{{ $\rightarrow$ dict}}
\pysigstopsignatures
\sphinxAtStartPar
Handle file download
\begin{quote}\begin{description}
\sphinxlineitem{Parameters}\begin{itemize}
\item {} 
\sphinxAtStartPar
\sphinxstyleliteralstrong{\sphinxupquote{request}} – Download reqeust

\item {} 
\sphinxAtStartPar
\sphinxstyleliteralstrong{\sphinxupquote{raw\_data}} – Data chunk to write

\end{itemize}

\sphinxlineitem{Returns}
\sphinxAtStartPar
Update response

\end{description}\end{quote}

\end{fulllineitems}\end{savenotes}

\index{prepare\_download() (eezz.document.TDocuments method)@\spxentry{prepare\_download()}\spxextra{eezz.document.TDocuments method}}

\begin{savenotes}\begin{fulllineitems}
\phantomsection\label{\detokenize{eezz:eezz.document.TDocuments.prepare_download}}
\pysigstartsignatures
\pysiglinewithargsret{\sphinxbfcode{\sphinxupquote{prepare\_download}}}{\sphinxparam{\DUrole{n}{request}\DUrole{p}{:}\DUrole{w}{ }\DUrole{n}{dict}}}{}
\pysigstopsignatures
\sphinxAtStartPar
Prepares the download of several files to include to an EEZZ document.
The preparation puts all file descriptors into a queue and waits to all documents until the last download.
This triggers the creation of the EEZZ document
\begin{quote}\begin{description}
\sphinxlineitem{Parameters}
\sphinxAtStartPar
\sphinxstyleliteralstrong{\sphinxupquote{request}} – The json format of a WEB socket request

\end{description}\end{quote}

\end{fulllineitems}\end{savenotes}

\index{read\_document\_header() (eezz.document.TDocuments method)@\spxentry{read\_document\_header()}\spxextra{eezz.document.TDocuments method}}

\begin{savenotes}\begin{fulllineitems}
\phantomsection\label{\detokenize{eezz:eezz.document.TDocuments.read_document_header}}
\pysigstartsignatures
\pysiglinewithargsret{\sphinxbfcode{\sphinxupquote{read\_document\_header}}}{\sphinxparam{\DUrole{n}{source}\DUrole{p}{:}\DUrole{w}{ }\DUrole{n}{Path}}}{{ $\rightarrow$ bool}}
\pysigstopsignatures
\sphinxAtStartPar
If a customer finds an EEZZ document, this method opens the zipped content and verifies the header.
With the verified header, the document could be unzipped.
\begin{quote}\begin{description}
\sphinxlineitem{Parameters}
\sphinxAtStartPar
\sphinxstyleliteralstrong{\sphinxupquote{source}} – 

\sphinxlineitem{Returns}
\sphinxAtStartPar


\end{description}\end{quote}

\end{fulllineitems}\end{savenotes}

\index{unzip() (eezz.document.TDocuments method)@\spxentry{unzip()}\spxextra{eezz.document.TDocuments method}}

\begin{savenotes}\begin{fulllineitems}
\phantomsection\label{\detokenize{eezz:eezz.document.TDocuments.unzip}}
\pysigstartsignatures
\pysiglinewithargsret{\sphinxbfcode{\sphinxupquote{unzip}}}{\sphinxparam{\DUrole{n}{source}\DUrole{p}{:}\DUrole{w}{ }\DUrole{n}{Path}}\sphinxparamcomma \sphinxparam{\DUrole{n}{manifest\_only}\DUrole{o}{=}\DUrole{default_value}{False}}\sphinxparamcomma \sphinxparam{\DUrole{n}{document\_key}\DUrole{p}{:}\DUrole{w}{ }\DUrole{n}{bytes}\DUrole{w}{ }\DUrole{o}{=}\DUrole{w}{ }\DUrole{default_value}{None}}}{{ $\rightarrow$ dict}}
\pysigstopsignatures
\sphinxAtStartPar
Unzip a file and return the Manifest in JSON format. Unzip needs the document key. If the key is not
available, unzip the preview and the manifest, store the result into database for further processing
\begin{quote}\begin{description}
\sphinxlineitem{Parameters}\begin{itemize}
\item {} 
\sphinxAtStartPar
\sphinxstyleliteralstrong{\sphinxupquote{source}} – 

\item {} 
\sphinxAtStartPar
\sphinxstyleliteralstrong{\sphinxupquote{manifest\_only}} – Extract the header

\item {} 
\sphinxAtStartPar
\sphinxstyleliteralstrong{\sphinxupquote{document\_key}} – If set, try to decrypt the document on the fly

\end{itemize}

\sphinxlineitem{Returns}
\sphinxAtStartPar
The header

\end{description}\end{quote}

\end{fulllineitems}\end{savenotes}

\index{zip() (eezz.document.TDocuments method)@\spxentry{zip()}\spxextra{eezz.document.TDocuments method}}

\begin{savenotes}\begin{fulllineitems}
\phantomsection\label{\detokenize{eezz:eezz.document.TDocuments.zip}}
\pysigstartsignatures
\pysiglinewithargsret{\sphinxbfcode{\sphinxupquote{zip}}}{\sphinxparam{\DUrole{n}{destination}\DUrole{p}{:}\DUrole{w}{ }\DUrole{n}{Path}}\sphinxparamcomma \sphinxparam{\DUrole{n}{manifest}\DUrole{p}{:}\DUrole{w}{ }\DUrole{n}{dict}}\sphinxparamcomma \sphinxparam{\DUrole{n}{files}\DUrole{p}{:}\DUrole{w}{ }\DUrole{n}{List\DUrole{p}{{[}}{\hyperref[\detokenize{eezz:eezz.filesrv.TFile}]{\sphinxcrossref{TFile}}}\DUrole{p}{{]}}}}}{}
\pysigstopsignatures
\sphinxAtStartPar
Zip the given files and the manifest to an EEZZ document.
\begin{quote}\begin{description}
\sphinxlineitem{Parameters}\begin{itemize}
\item {} 
\sphinxAtStartPar
\sphinxstyleliteralstrong{\sphinxupquote{destination}} – Path to the EEZZ document. Has to be like <directory>/<filename>

\item {} 
\sphinxAtStartPar
\sphinxstyleliteralstrong{\sphinxupquote{manifest}} – Description and header

\item {} 
\sphinxAtStartPar
\sphinxstyleliteralstrong{\sphinxupquote{files}} – Files included for the document

\end{itemize}

\end{description}\end{quote}

\end{fulllineitems}\end{savenotes}


\end{fulllineitems}\end{savenotes}

\index{TManifest (class in eezz.document)@\spxentry{TManifest}\spxextra{class in eezz.document}}

\begin{savenotes}\begin{fulllineitems}
\phantomsection\label{\detokenize{eezz:eezz.document.TManifest}}
\pysigstartsignatures
\pysiglinewithargsret{\sphinxbfcode{\sphinxupquote{class\DUrole{w}{ }}}\sphinxcode{\sphinxupquote{eezz.document.}}\sphinxbfcode{\sphinxupquote{TManifest}}}{\sphinxparam{\DUrole{o}{*}}\sphinxparamcomma \sphinxparam{\DUrole{n}{document\_key}\DUrole{p}{:}\DUrole{w}{ }\DUrole{n}{bytes}\DUrole{w}{ }\DUrole{o}{=}\DUrole{w}{ }\DUrole{default_value}{None}}\sphinxparamcomma \sphinxparam{\DUrole{n}{column\_names}\DUrole{p}{:}\DUrole{w}{ }\DUrole{n}{List\DUrole{p}{{[}}str\DUrole{p}{{]}}}\DUrole{w}{ }\DUrole{o}{=}\DUrole{w}{ }\DUrole{default_value}{None}}}{}
\pysigstopsignatures
\sphinxAtStartPar
Bases: \sphinxcode{\sphinxupquote{object}}

\sphinxAtStartPar
The manifest represents the information for the EEZZ document. This structure is stored
together with all other EEZZ files.
\begin{quote}\begin{description}
\sphinxlineitem{Parameters}\begin{itemize}
\item {} 
\sphinxAtStartPar
\sphinxstyleliteralstrong{\sphinxupquote{column\_names}} – 

\item {} 
\sphinxAtStartPar
\sphinxstyleliteralstrong{\sphinxupquote{document\_key}} – 

\end{itemize}

\end{description}\end{quote}
\index{get\_key() (eezz.document.TManifest method)@\spxentry{get\_key()}\spxextra{eezz.document.TManifest method}}

\begin{savenotes}\begin{fulllineitems}
\phantomsection\label{\detokenize{eezz:eezz.document.TManifest.get_key}}
\pysigstartsignatures
\pysiglinewithargsret{\sphinxbfcode{\sphinxupquote{get\_key}}}{}{{ $\rightarrow$ str}}
\pysigstopsignatures
\end{fulllineitems}\end{savenotes}

\index{get\_values() (eezz.document.TManifest method)@\spxentry{get\_values()}\spxextra{eezz.document.TManifest method}}

\begin{savenotes}\begin{fulllineitems}
\phantomsection\label{\detokenize{eezz:eezz.document.TManifest.get_values}}
\pysigstartsignatures
\pysiglinewithargsret{\sphinxbfcode{\sphinxupquote{get\_values}}}{}{{ $\rightarrow$ list}}
\pysigstopsignatures
\end{fulllineitems}\end{savenotes}

\index{scan() (eezz.document.TManifest method)@\spxentry{scan()}\spxextra{eezz.document.TManifest method}}

\begin{savenotes}\begin{fulllineitems}
\phantomsection\label{\detokenize{eezz:eezz.document.TManifest.scan}}
\pysigstartsignatures
\pysiglinewithargsret{\sphinxbfcode{\sphinxupquote{scan}}}{\sphinxparam{\DUrole{n}{header}\DUrole{p}{:}\DUrole{w}{ }\DUrole{n}{str}}}{}
\pysigstopsignatures
\end{fulllineitems}\end{savenotes}

\index{set\_values() (eezz.document.TManifest method)@\spxentry{set\_values()}\spxextra{eezz.document.TManifest method}}

\begin{savenotes}\begin{fulllineitems}
\phantomsection\label{\detokenize{eezz:eezz.document.TManifest.set_values}}
\pysigstartsignatures
\pysiglinewithargsret{\sphinxbfcode{\sphinxupquote{set\_values}}}{\sphinxparam{\DUrole{n}{section}\DUrole{p}{:}\DUrole{w}{ }\DUrole{n}{str}}\sphinxparamcomma \sphinxparam{\DUrole{n}{proposed}\DUrole{p}{:}\DUrole{w}{ }\DUrole{n}{dict\DUrole{w}{ }\DUrole{p}{|}\DUrole{w}{ }bytes}}}{}
\pysigstopsignatures
\sphinxAtStartPar
Insert values to the manifest. There is a restricted set of keys in a fixed structure
\begin{quote}\begin{description}
\sphinxlineitem{Parameters}\begin{itemize}
\item {} 
\sphinxAtStartPar
\sphinxstyleliteralstrong{\sphinxupquote{section}} – Define a section to add info

\item {} 
\sphinxAtStartPar
\sphinxstyleliteralstrong{\sphinxupquote{proposed}} (\sphinxstyleliteralemphasis{\sphinxupquote{dict}}) – Proposed data: The program will pick only allowed keys

\end{itemize}

\end{description}\end{quote}

\end{fulllineitems}\end{savenotes}


\end{fulllineitems}\end{savenotes}



\subsection{eezz.filesrv module}
\label{\detokenize{eezz:module-eezz.filesrv}}\label{\detokenize{eezz:eezz-filesrv-module}}\index{module@\spxentry{module}!eezz.filesrv@\spxentry{eezz.filesrv}}\index{eezz.filesrv@\spxentry{eezz.filesrv}!module@\spxentry{module}}
\sphinxAtStartPar
This module implements the following classes:
\begin{itemize}
\item {} 
\sphinxAtStartPar
\sphinxstylestrong{TFile}:        Takes a chunk of data and merges it to a file

\item {} 
\sphinxAtStartPar
\sphinxstylestrong{TEezzFile}:    Extends TFile and implements encryption and decryption for transmitted data

\item {} 
\sphinxAtStartPar
\sphinxstylestrong{TFileMode}:    Enum file\sphinxhyphen{}mode for TEezzFiles

\end{itemize}

\sphinxAtStartPar
This module supports a download of big files in chunks and ensures, that the incoming fragments are
put together in the correct order again. Furthermore, a hash is calculated for each chunk, so that the
data consistency of a file could be ensured during reading.
\index{TEezzFile (class in eezz.filesrv)@\spxentry{TEezzFile}\spxextra{class in eezz.filesrv}}

\begin{savenotes}\begin{fulllineitems}
\phantomsection\label{\detokenize{eezz:eezz.filesrv.TEezzFile}}
\pysigstartsignatures
\pysiglinewithargsret{\sphinxbfcode{\sphinxupquote{class\DUrole{w}{ }}}\sphinxcode{\sphinxupquote{eezz.filesrv.}}\sphinxbfcode{\sphinxupquote{TEezzFile}}}{\sphinxparam{\DUrole{o}{*}}\sphinxparamcomma \sphinxparam{\DUrole{n}{file\_type}\DUrole{p}{:}\DUrole{w}{ }\DUrole{n}{str}}\sphinxparamcomma \sphinxparam{\DUrole{n}{destination}\DUrole{p}{:}\DUrole{w}{ }\DUrole{n}{Path}}\sphinxparamcomma \sphinxparam{\DUrole{n}{size}\DUrole{p}{:}\DUrole{w}{ }\DUrole{n}{int}}\sphinxparamcomma \sphinxparam{\DUrole{n}{chunk\_size}\DUrole{p}{:}\DUrole{w}{ }\DUrole{n}{int}}\sphinxparamcomma \sphinxparam{\DUrole{n}{key}\DUrole{p}{:}\DUrole{w}{ }\DUrole{n}{bytes}}\sphinxparamcomma \sphinxparam{\DUrole{n}{vector}\DUrole{p}{:}\DUrole{w}{ }\DUrole{n}{bytes}}\sphinxparamcomma \sphinxparam{\DUrole{n}{response}\DUrole{p}{:}\DUrole{w}{ }\DUrole{n}{Queue}\DUrole{w}{ }\DUrole{o}{=}\DUrole{w}{ }\DUrole{default_value}{None}}\sphinxparamcomma \sphinxparam{\DUrole{n}{hash\_chain}\DUrole{p}{:}\DUrole{w}{ }\DUrole{n}{dict}\DUrole{w}{ }\DUrole{o}{=}\DUrole{w}{ }\DUrole{default_value}{None}}}{}
\pysigstopsignatures
\sphinxAtStartPar
Bases: {\hyperref[\detokenize{eezz:eezz.filesrv.TFile}]{\sphinxcrossref{\sphinxcode{\sphinxupquote{TFile}}}}}

\sphinxAtStartPar
Derived from TFile, this class allows encryption and decryption using AES key.
After finishing the transfer, the instance is pushed into the response queue, which allows to implement a
supervisor thread, which blocks on the queue reading
\begin{quote}\begin{description}
\sphinxlineitem{Parameters}\begin{itemize}
\item {} 
\sphinxAtStartPar
\sphinxstyleliteralstrong{\sphinxupquote{key}} (\sphinxstyleliteralemphasis{\sphinxupquote{Crypto.Random.new}}\sphinxstyleliteralemphasis{\sphinxupquote{(}}\sphinxstyleliteralemphasis{\sphinxupquote{16}}\sphinxstyleliteralemphasis{\sphinxupquote{)}}) – AES key for cypher

\item {} 
\sphinxAtStartPar
\sphinxstyleliteralstrong{\sphinxupquote{vector}} (\sphinxstyleliteralemphasis{\sphinxupquote{Crypto.Random.new}}\sphinxstyleliteralemphasis{\sphinxupquote{(}}\sphinxstyleliteralemphasis{\sphinxupquote{16}}\sphinxstyleliteralemphasis{\sphinxupquote{)}}) – AES vector for cypher

\item {} 
\sphinxAtStartPar
\sphinxstyleliteralstrong{\sphinxupquote{response}} (\sphinxstyleliteralemphasis{\sphinxupquote{Queue.queue}}) – Queue to get the final state of an instance

\item {} 
\sphinxAtStartPar
\sphinxstyleliteralstrong{\sphinxupquote{hash\_chain}} (\sphinxstyleliteralemphasis{\sphinxupquote{List}}\sphinxstyleliteralemphasis{\sphinxupquote{{[}}}\sphinxstyleliteralemphasis{\sphinxupquote{SHA256.hexdigest}}\sphinxstyleliteralemphasis{\sphinxupquote{{]}}}) – A list of hash values for each chunk

\end{itemize}

\end{description}\end{quote}
\index{decrypt() (eezz.filesrv.TEezzFile method)@\spxentry{decrypt()}\spxextra{eezz.filesrv.TEezzFile method}}

\begin{savenotes}\begin{fulllineitems}
\phantomsection\label{\detokenize{eezz:eezz.filesrv.TEezzFile.decrypt}}
\pysigstartsignatures
\pysiglinewithargsret{\sphinxbfcode{\sphinxupquote{decrypt}}}{\sphinxparam{\DUrole{n}{raw\_data}\DUrole{p}{:}\DUrole{w}{ }\DUrole{n}{Any}}\sphinxparamcomma \sphinxparam{\DUrole{n}{sequence\_nr}\DUrole{p}{:}\DUrole{w}{ }\DUrole{n}{int}}}{{ $\rightarrow$ str}}
\pysigstopsignatures
\sphinxAtStartPar
Decrypt the incoming stream
\begin{quote}\begin{description}
\sphinxlineitem{Parameters}\begin{itemize}
\item {} 
\sphinxAtStartPar
\sphinxstyleliteralstrong{\sphinxupquote{raw\_data}} – Data chunk of the steam

\item {} 
\sphinxAtStartPar
\sphinxstyleliteralstrong{\sphinxupquote{sequence\_nr}} – Sequence number in the stream

\end{itemize}

\sphinxlineitem{Returns}
\sphinxAtStartPar
Hash value of the chunk

\sphinxlineitem{Return type}
\sphinxAtStartPar
SHA256.hexdigest

\end{description}\end{quote}

\end{fulllineitems}\end{savenotes}

\index{encrypt() (eezz.filesrv.TEezzFile method)@\spxentry{encrypt()}\spxextra{eezz.filesrv.TEezzFile method}}

\begin{savenotes}\begin{fulllineitems}
\phantomsection\label{\detokenize{eezz:eezz.filesrv.TEezzFile.encrypt}}
\pysigstartsignatures
\pysiglinewithargsret{\sphinxbfcode{\sphinxupquote{encrypt}}}{\sphinxparam{\DUrole{n}{raw\_data}\DUrole{p}{:}\DUrole{w}{ }\DUrole{n}{Any}}\sphinxparamcomma \sphinxparam{\DUrole{n}{sequence\_nr}\DUrole{p}{:}\DUrole{w}{ }\DUrole{n}{int}}}{{ $\rightarrow$ str}}
\pysigstopsignatures
\sphinxAtStartPar
Encrypt the incoming data stream
\begin{quote}\begin{description}
\sphinxlineitem{Parameters}\begin{itemize}
\item {} 
\sphinxAtStartPar
\sphinxstyleliteralstrong{\sphinxupquote{raw\_data}} – Data chunk of the stream

\item {} 
\sphinxAtStartPar
\sphinxstyleliteralstrong{\sphinxupquote{sequence\_nr}} – Sequence number in the stream

\end{itemize}

\sphinxlineitem{Returns}
\sphinxAtStartPar
Hash value of the chunk

\sphinxlineitem{Return type}
\sphinxAtStartPar
SHA256.hexdigest

\end{description}\end{quote}

\end{fulllineitems}\end{savenotes}

\index{read() (eezz.filesrv.TEezzFile method)@\spxentry{read()}\spxextra{eezz.filesrv.TEezzFile method}}

\begin{savenotes}\begin{fulllineitems}
\phantomsection\label{\detokenize{eezz:eezz.filesrv.TEezzFile.read}}
\pysigstartsignatures
\pysiglinewithargsret{\sphinxbfcode{\sphinxupquote{read}}}{\sphinxparam{\DUrole{n}{source}\DUrole{p}{:}\DUrole{w}{ }\DUrole{n}{BufferedReader}}\sphinxparamcomma \sphinxparam{\DUrole{n}{hash\_list}\DUrole{p}{:}\DUrole{w}{ }\DUrole{n}{List\DUrole{p}{{[}}str\DUrole{p}{{]}}}\DUrole{w}{ }\DUrole{o}{=}\DUrole{w}{ }\DUrole{default_value}{None}}}{{ $\rightarrow$ None}}
\pysigstopsignatures
\sphinxAtStartPar
Read an encrypted file from source input stream and create an decrypted version
\begin{quote}\begin{description}
\sphinxlineitem{Parameters}\begin{itemize}
\item {} 
\sphinxAtStartPar
\sphinxstyleliteralstrong{\sphinxupquote{source}} – Input stream

\item {} 
\sphinxAtStartPar
\sphinxstyleliteralstrong{\sphinxupquote{hash\_list}} (\sphinxstyleliteralemphasis{\sphinxupquote{List}}\sphinxstyleliteralemphasis{\sphinxupquote{{[}}}\sphinxstyleliteralemphasis{\sphinxupquote{SHA256.hexdigest}}\sphinxstyleliteralemphasis{\sphinxupquote{{]}}}) – A hash list to check the stream

\end{itemize}

\end{description}\end{quote}

\end{fulllineitems}\end{savenotes}

\index{write() (eezz.filesrv.TEezzFile method)@\spxentry{write()}\spxextra{eezz.filesrv.TEezzFile method}}

\begin{savenotes}\begin{fulllineitems}
\phantomsection\label{\detokenize{eezz:eezz.filesrv.TEezzFile.write}}
\pysigstartsignatures
\pysiglinewithargsret{\sphinxbfcode{\sphinxupquote{write}}}{\sphinxparam{\DUrole{n}{raw\_data}\DUrole{p}{:}\DUrole{w}{ }\DUrole{n}{Any}}\sphinxparamcomma \sphinxparam{\DUrole{n}{sequence\_nr}\DUrole{p}{:}\DUrole{w}{ }\DUrole{n}{int}}\sphinxparamcomma \sphinxparam{\DUrole{n}{mode}\DUrole{p}{:}\DUrole{w}{ }\DUrole{n}{{\hyperref[\detokenize{eezz:eezz.filesrv.TFileMode}]{\sphinxcrossref{TFileMode}}}}\DUrole{w}{ }\DUrole{o}{=}\DUrole{w}{ }\DUrole{default_value}{TFileMode.ENCRYPT}}}{{ $\rightarrow$ str}}
\pysigstopsignatures
\sphinxAtStartPar
Write a chunk of data
\begin{quote}\begin{description}
\sphinxlineitem{Parameters}\begin{itemize}
\item {} 
\sphinxAtStartPar
\sphinxstyleliteralstrong{\sphinxupquote{raw\_data}} – The data chunk to write

\item {} 
\sphinxAtStartPar
\sphinxstyleliteralstrong{\sphinxupquote{sequence\_nr}} – Sequence of the data chunk in the stream

\item {} 
\sphinxAtStartPar
\sphinxstyleliteralstrong{\sphinxupquote{mode}} – The mode used to en\sphinxhyphen{} or decrypt the data or pass through

\end{itemize}

\sphinxlineitem{Returns}
\sphinxAtStartPar
The hash value of the data after encryption/before decryption

\sphinxlineitem{Return type}
\sphinxAtStartPar
SHA256.hexdigest

\end{description}\end{quote}

\end{fulllineitems}\end{savenotes}


\end{fulllineitems}\end{savenotes}

\index{TFile (class in eezz.filesrv)@\spxentry{TFile}\spxextra{class in eezz.filesrv}}

\begin{savenotes}\begin{fulllineitems}
\phantomsection\label{\detokenize{eezz:eezz.filesrv.TFile}}
\pysigstartsignatures
\pysiglinewithargsret{\sphinxbfcode{\sphinxupquote{class\DUrole{w}{ }}}\sphinxcode{\sphinxupquote{eezz.filesrv.}}\sphinxbfcode{\sphinxupquote{TFile}}}{\sphinxparam{\DUrole{o}{*}}\sphinxparamcomma \sphinxparam{\DUrole{n}{file\_type}\DUrole{p}{:}\DUrole{w}{ }\DUrole{n}{str}}\sphinxparamcomma \sphinxparam{\DUrole{n}{destination}\DUrole{p}{:}\DUrole{w}{ }\DUrole{n}{Path}}\sphinxparamcomma \sphinxparam{\DUrole{n}{size}\DUrole{p}{:}\DUrole{w}{ }\DUrole{n}{int}}\sphinxparamcomma \sphinxparam{\DUrole{n}{chunk\_size}\DUrole{p}{:}\DUrole{w}{ }\DUrole{n}{int}}}{}
\pysigstopsignatures
\sphinxAtStartPar
Bases: \sphinxcode{\sphinxupquote{object}}

\sphinxAtStartPar
Class to be used as file download handler. It accepts chunks of data in separate calls
\begin{quote}\begin{description}
\sphinxlineitem{Parameters}\begin{itemize}
\item {} 
\sphinxAtStartPar
\sphinxstyleliteralstrong{\sphinxupquote{file\_type}} – User defined file type

\item {} 
\sphinxAtStartPar
\sphinxstyleliteralstrong{\sphinxupquote{destination}} – Path to store the file

\item {} 
\sphinxAtStartPar
\sphinxstyleliteralstrong{\sphinxupquote{size}} – The size of the file

\item {} 
\sphinxAtStartPar
\sphinxstyleliteralstrong{\sphinxupquote{chunk\_size}} – Fixed size for each chunk of data, except the last element

\end{itemize}

\end{description}\end{quote}
\index{write() (eezz.filesrv.TFile method)@\spxentry{write()}\spxextra{eezz.filesrv.TFile method}}

\begin{savenotes}\begin{fulllineitems}
\phantomsection\label{\detokenize{eezz:eezz.filesrv.TFile.write}}
\pysigstartsignatures
\pysiglinewithargsret{\sphinxbfcode{\sphinxupquote{write}}}{\sphinxparam{\DUrole{n}{raw\_data}\DUrole{p}{:}\DUrole{w}{ }\DUrole{n}{Any}}\sphinxparamcomma \sphinxparam{\DUrole{n}{sequence\_nr}\DUrole{p}{:}\DUrole{w}{ }\DUrole{n}{int}}\sphinxparamcomma \sphinxparam{\DUrole{n}{mode}\DUrole{p}{:}\DUrole{w}{ }\DUrole{n}{{\hyperref[\detokenize{eezz:eezz.filesrv.TFileMode}]{\sphinxcrossref{TFileMode}}}}\DUrole{w}{ }\DUrole{o}{=}\DUrole{w}{ }\DUrole{default_value}{TFileMode.NORMAL}}}{{ $\rightarrow$ str}}
\pysigstopsignatures
\sphinxAtStartPar
Write constant chunks of raw data to file. Only the last chunk might be smaller.
The sequence number is passed along, because we cannot guarantee, that elements received in the same
order as they are send.
\begin{quote}\begin{description}
\sphinxlineitem{Parameters}\begin{itemize}
\item {} 
\sphinxAtStartPar
\sphinxstyleliteralstrong{\sphinxupquote{raw\_data}} – Raw chunk of data

\item {} 
\sphinxAtStartPar
\sphinxstyleliteralstrong{\sphinxupquote{sequence\_nr}} – Sequence number to insert chunks at the right place

\item {} 
\sphinxAtStartPar
\sphinxstyleliteralstrong{\sphinxupquote{mode}} – Ignored:   set signature for derived classes

\end{itemize}

\sphinxlineitem{Returns}
\sphinxAtStartPar
Empty string:  set signature for derived classes

\end{description}\end{quote}

\end{fulllineitems}\end{savenotes}


\end{fulllineitems}\end{savenotes}



\begin{savenotes}\begin{fulllineitems}
\phantomsection\label{\detokenize{eezz:eezz.filesrv.TFileMode}}
\pysigstartsignatures
\pysiglinewithargsret{\sphinxbfcode{\sphinxupquote{enum\DUrole{w}{ }}}\sphinxcode{\sphinxupquote{eezz.filesrv.}}\sphinxbfcode{\sphinxupquote{TFileMode}}}{\sphinxparam{\DUrole{n}{value}}}{}
\pysigstopsignatures
\sphinxAtStartPar
Bases: \sphinxcode{\sphinxupquote{Enum}}

\sphinxAtStartPar
File mode: Determine how to handle incoming stream
\begin{quote}\begin{description}
\sphinxlineitem{Parameters}\begin{itemize}
\item {} 
\sphinxAtStartPar
\sphinxstyleliteralstrong{\sphinxupquote{NORMAL}} – Write through

\item {} 
\sphinxAtStartPar
\sphinxstyleliteralstrong{\sphinxupquote{ENCRYPT}} – Encrypt and write

\item {} 
\sphinxAtStartPar
\sphinxstyleliteralstrong{\sphinxupquote{ENCRYPT}} – Decrypt and write

\end{itemize}

\end{description}\end{quote}

\sphinxAtStartPar
Valid values are as follows:
\index{NORMAL (eezz.filesrv.TFileMode attribute)@\spxentry{NORMAL}\spxextra{eezz.filesrv.TFileMode attribute}}

\begin{savenotes}\begin{fulllineitems}
\phantomsection\label{\detokenize{eezz:eezz.filesrv.TFileMode.NORMAL}}
\pysigstartsignatures
\pysigline{\sphinxbfcode{\sphinxupquote{NORMAL}}\sphinxbfcode{\sphinxupquote{\DUrole{w}{ }\DUrole{p}{=}\DUrole{w}{ }<TFileMode.NORMAL: 0>}}}
\pysigstopsignatures
\end{fulllineitems}\end{savenotes}

\index{ENCRYPT (eezz.filesrv.TFileMode attribute)@\spxentry{ENCRYPT}\spxextra{eezz.filesrv.TFileMode attribute}}

\begin{savenotes}\begin{fulllineitems}
\phantomsection\label{\detokenize{eezz:eezz.filesrv.TFileMode.ENCRYPT}}
\pysigstartsignatures
\pysigline{\sphinxbfcode{\sphinxupquote{ENCRYPT}}\sphinxbfcode{\sphinxupquote{\DUrole{w}{ }\DUrole{p}{=}\DUrole{w}{ }<TFileMode.ENCRYPT: 1>}}}
\pysigstopsignatures
\end{fulllineitems}\end{savenotes}

\index{DECRYPT (eezz.filesrv.TFileMode attribute)@\spxentry{DECRYPT}\spxextra{eezz.filesrv.TFileMode attribute}}

\begin{savenotes}\begin{fulllineitems}
\phantomsection\label{\detokenize{eezz:eezz.filesrv.TFileMode.DECRYPT}}
\pysigstartsignatures
\pysigline{\sphinxbfcode{\sphinxupquote{DECRYPT}}\sphinxbfcode{\sphinxupquote{\DUrole{w}{ }\DUrole{p}{=}\DUrole{w}{ }<TFileMode.DECRYPT: 2>}}}
\pysigstopsignatures
\end{fulllineitems}\end{savenotes}


\end{fulllineitems}\end{savenotes}

\index{test\_file\_reader() (in module eezz.filesrv)@\spxentry{test\_file\_reader()}\spxextra{in module eezz.filesrv}}

\begin{savenotes}\begin{fulllineitems}
\phantomsection\label{\detokenize{eezz:eezz.filesrv.test_file_reader}}
\pysigstartsignatures
\pysiglinewithargsret{\sphinxcode{\sphinxupquote{eezz.filesrv.}}\sphinxbfcode{\sphinxupquote{test\_file\_reader}}}{}{}
\pysigstopsignatures
\sphinxAtStartPar
Test the TFile interfaces
:meta private:

\end{fulllineitems}\end{savenotes}



\subsection{eezz.http\_agent module}
\label{\detokenize{eezz:module-eezz.http_agent}}\label{\detokenize{eezz:eezz-http-agent-module}}\index{module@\spxentry{module}!eezz.http\_agent@\spxentry{eezz.http\_agent}}\index{eezz.http\_agent@\spxentry{eezz.http\_agent}!module@\spxentry{module}}\begin{itemize}
\item {} 
\sphinxAtStartPar
\sphinxstylestrong{THttpAgent}: Handle WEB\sphinxhyphen{}Socket requests

\end{itemize}

\sphinxAtStartPar
The interaction with the JavaScript via WEB\sphinxhyphen{}Socket includes generation of HTML parts for user interface updates
\index{THttpAgent (class in eezz.http\_agent)@\spxentry{THttpAgent}\spxextra{class in eezz.http\_agent}}

\begin{savenotes}\begin{fulllineitems}
\phantomsection\label{\detokenize{eezz:eezz.http_agent.THttpAgent}}
\pysigstartsignatures
\pysigline{\sphinxbfcode{\sphinxupquote{class\DUrole{w}{ }}}\sphinxcode{\sphinxupquote{eezz.http\_agent.}}\sphinxbfcode{\sphinxupquote{THttpAgent}}}
\pysigstopsignatures
\sphinxAtStartPar
Bases: \sphinxcode{\sphinxupquote{TWebSocketAgent}}

\sphinxAtStartPar
Agent handles WEB socket events
\index{compile\_data() (eezz.http\_agent.THttpAgent method)@\spxentry{compile\_data()}\spxextra{eezz.http\_agent.THttpAgent method}}

\begin{savenotes}\begin{fulllineitems}
\phantomsection\label{\detokenize{eezz:eezz.http_agent.THttpAgent.compile_data}}
\pysigstartsignatures
\pysiglinewithargsret{\sphinxbfcode{\sphinxupquote{compile\_data}}}{\sphinxparam{\DUrole{n}{a\_parser}\DUrole{p}{:}\DUrole{w}{ }\DUrole{n}{Lark}}\sphinxparamcomma \sphinxparam{\DUrole{n}{a\_tag\_list}\DUrole{p}{:}\DUrole{w}{ }\DUrole{n}{list}}\sphinxparamcomma \sphinxparam{\DUrole{n}{a\_id}\DUrole{p}{:}\DUrole{w}{ }\DUrole{n}{str}}\sphinxparamcomma \sphinxparam{\DUrole{n}{a\_query}\DUrole{p}{:}\DUrole{w}{ }\DUrole{n}{dict}\DUrole{w}{ }\DUrole{o}{=}\DUrole{w}{ }\DUrole{default_value}{None}}}{{ $\rightarrow$ None}}
\pysigstopsignatures
\sphinxAtStartPar
Compile data\sphinxhyphen{}eezz\sphinxhyphen{}json to data\sphinxhyphen{}eezz\sphinxhyphen{}compile,
create tag attributes and generate tag\sphinxhyphen{}id to manage incoming requests
\begin{quote}\begin{description}
\sphinxlineitem{Parameters}\begin{itemize}
\item {} 
\sphinxAtStartPar
\sphinxstyleliteralstrong{\sphinxupquote{a\_parser}} – The Lark parser to compile EEZZ to json

\item {} 
\sphinxAtStartPar
\sphinxstyleliteralstrong{\sphinxupquote{a\_tag\_list}} – HTML\sphinxhyphen{}Tag to compile

\item {} 
\sphinxAtStartPar
\sphinxstyleliteralstrong{\sphinxupquote{a\_id}} – The ID of the tag to be identified for update

\item {} 
\sphinxAtStartPar
\sphinxstyleliteralstrong{\sphinxupquote{a\_query}} – The query of the HTML request

\end{itemize}

\sphinxlineitem{Returns}
\sphinxAtStartPar
None

\end{description}\end{quote}

\end{fulllineitems}\end{savenotes}

\index{do\_get() (eezz.http\_agent.THttpAgent method)@\spxentry{do\_get()}\spxextra{eezz.http\_agent.THttpAgent method}}

\begin{savenotes}\begin{fulllineitems}
\phantomsection\label{\detokenize{eezz:eezz.http_agent.THttpAgent.do_get}}
\pysigstartsignatures
\pysiglinewithargsret{\sphinxbfcode{\sphinxupquote{do\_get}}}{\sphinxparam{\DUrole{n}{a\_resource}\DUrole{p}{:}\DUrole{w}{ }\DUrole{n}{Path\DUrole{w}{ }\DUrole{p}{|}\DUrole{w}{ }str}}\sphinxparamcomma \sphinxparam{\DUrole{n}{a\_query}\DUrole{p}{:}\DUrole{w}{ }\DUrole{n}{dict}}}{{ $\rightarrow$ str}}
\pysigstopsignatures
\sphinxAtStartPar
Response to an HTML GET command

\sphinxAtStartPar
The agent reads the source, compiles the data\sphinxhyphen{}eezz sections and adds the web\sphinxhyphen{}socket component
It returns the enriched document
\begin{quote}\begin{description}
\sphinxlineitem{Parameters}\begin{itemize}
\item {} 
\sphinxAtStartPar
\sphinxstyleliteralstrong{\sphinxupquote{a\_resource}} (\sphinxstyleliteralemphasis{\sphinxupquote{pathlib.Path}}) – The path to the HTML document, containing EEZZ extensions

\item {} 
\sphinxAtStartPar
\sphinxstyleliteralstrong{\sphinxupquote{a\_query}} – The query string of the URL

\end{itemize}

\sphinxlineitem{Returns}
\sphinxAtStartPar
The compiled version of the HTML file

\end{description}\end{quote}

\end{fulllineitems}\end{savenotes}

\index{format\_attributes() (eezz.http\_agent.THttpAgent method)@\spxentry{format\_attributes()}\spxextra{eezz.http\_agent.THttpAgent method}}

\begin{savenotes}\begin{fulllineitems}
\phantomsection\label{\detokenize{eezz:eezz.http_agent.THttpAgent.format_attributes}}
\pysigstartsignatures
\pysiglinewithargsret{\sphinxbfcode{\sphinxupquote{format\_attributes}}}{\sphinxparam{\DUrole{n}{a\_key}\DUrole{p}{:}\DUrole{w}{ }\DUrole{n}{str}}\sphinxparamcomma \sphinxparam{\DUrole{n}{a\_value}\DUrole{p}{:}\DUrole{w}{ }\DUrole{n}{str}}\sphinxparamcomma \sphinxparam{\DUrole{n}{a\_fmt\_funct}\DUrole{p}{:}\DUrole{w}{ }\DUrole{n}{Callable}}}{{ $\rightarrow$ str}}
\pysigstopsignatures
\sphinxAtStartPar
Eval template tag\sphinxhyphen{}attributes, diving deep into data\sphinxhyphen{}eezz\sphinxhyphen{}json
\begin{quote}\begin{description}
\sphinxlineitem{Parameters}\begin{itemize}
\item {} 
\sphinxAtStartPar
\sphinxstyleliteralstrong{\sphinxupquote{a\_key}} – Thw key string to pick the items in a HTML tag

\item {} 
\sphinxAtStartPar
\sphinxstyleliteralstrong{\sphinxupquote{a\_value}} – The dictionary in string format to be formatted

\item {} 
\sphinxAtStartPar
\sphinxstyleliteralstrong{\sphinxupquote{a\_fmt\_funct}} – The function to be called to format the values

\end{itemize}

\sphinxlineitem{Returns}
\sphinxAtStartPar
The formatted string

\end{description}\end{quote}

\end{fulllineitems}\end{savenotes}

\index{generate\_html\_cells() (eezz.http\_agent.THttpAgent method)@\spxentry{generate\_html\_cells()}\spxextra{eezz.http\_agent.THttpAgent method}}

\begin{savenotes}\begin{fulllineitems}
\phantomsection\label{\detokenize{eezz:eezz.http_agent.THttpAgent.generate_html_cells}}
\pysigstartsignatures
\pysiglinewithargsret{\sphinxbfcode{\sphinxupquote{generate\_html\_cells}}}{\sphinxparam{\DUrole{n}{a\_tag}\DUrole{p}{:}\DUrole{w}{ }\DUrole{n}{Tag}}\sphinxparamcomma \sphinxparam{\DUrole{n}{a\_cell}\DUrole{p}{:}\DUrole{w}{ }\DUrole{n}{{\hyperref[\detokenize{eezz:eezz.table.TTableCell}]{\sphinxcrossref{TTableCell}}}}}}{{ $\rightarrow$ Tag}}
\pysigstopsignatures
\sphinxAtStartPar
Generate HTML cells
Input for the lamda is a string and output is formatted according to the TTableCell object
\begin{quote}\begin{description}
\sphinxlineitem{Parameters}\begin{itemize}
\item {} 
\sphinxAtStartPar
\sphinxstyleliteralstrong{\sphinxupquote{a\_tag}} – The parent tag to generate the table cells

\item {} 
\sphinxAtStartPar
\sphinxstyleliteralstrong{\sphinxupquote{a\_cell}} – The template cell to format to HTML

\end{itemize}

\sphinxlineitem{Returns}
\sphinxAtStartPar
The formatted HTML tag

\end{description}\end{quote}

\end{fulllineitems}\end{savenotes}

\index{generate\_html\_grid() (eezz.http\_agent.THttpAgent method)@\spxentry{generate\_html\_grid()}\spxextra{eezz.http\_agent.THttpAgent method}}

\begin{savenotes}\begin{fulllineitems}
\phantomsection\label{\detokenize{eezz:eezz.http_agent.THttpAgent.generate_html_grid}}
\pysigstartsignatures
\pysiglinewithargsret{\sphinxbfcode{\sphinxupquote{generate\_html\_grid}}}{\sphinxparam{\DUrole{n}{a\_tag}\DUrole{p}{:}\DUrole{w}{ }\DUrole{n}{Tag}}}{{ $\rightarrow$ dict}}
\pysigstopsignatures
\sphinxAtStartPar
Besides the table, supported display is grid (via class clzz\_grid or select

\end{fulllineitems}\end{savenotes}

\index{generate\_html\_grid\_item() (eezz.http\_agent.THttpAgent method)@\spxentry{generate\_html\_grid\_item()}\spxextra{eezz.http\_agent.THttpAgent method}}

\begin{savenotes}\begin{fulllineitems}
\phantomsection\label{\detokenize{eezz:eezz.http_agent.THttpAgent.generate_html_grid_item}}
\pysigstartsignatures
\pysiglinewithargsret{\sphinxbfcode{\sphinxupquote{generate\_html\_grid\_item}}}{\sphinxparam{\DUrole{n}{a\_tag}\DUrole{p}{:}\DUrole{w}{ }\DUrole{n}{Tag}}\sphinxparamcomma \sphinxparam{\DUrole{n}{a\_row}\DUrole{p}{:}\DUrole{w}{ }\DUrole{n}{{\hyperref[\detokenize{eezz:eezz.table.TTableRow}]{\sphinxcrossref{TTableRow}}}}}\sphinxparamcomma \sphinxparam{\DUrole{n}{a\_header}\DUrole{p}{:}\DUrole{w}{ }\DUrole{n}{{\hyperref[\detokenize{eezz:eezz.table.TTableRow}]{\sphinxcrossref{TTableRow}}}}}}{{ $\rightarrow$ Tag}}
\pysigstopsignatures
\sphinxAtStartPar
Generates elements of the same kind, derived from a template and update content
according the row values

\end{fulllineitems}\end{savenotes}

\index{generate\_html\_rows() (eezz.http\_agent.THttpAgent method)@\spxentry{generate\_html\_rows()}\spxextra{eezz.http\_agent.THttpAgent method}}

\begin{savenotes}\begin{fulllineitems}
\phantomsection\label{\detokenize{eezz:eezz.http_agent.THttpAgent.generate_html_rows}}
\pysigstartsignatures
\pysiglinewithargsret{\sphinxbfcode{\sphinxupquote{generate\_html\_rows}}}{\sphinxparam{\DUrole{n}{a\_html\_cells}\DUrole{p}{:}\DUrole{w}{ }\DUrole{n}{list}}\sphinxparamcomma \sphinxparam{\DUrole{n}{a\_tag}\DUrole{p}{:}\DUrole{w}{ }\DUrole{n}{Tag}}\sphinxparamcomma \sphinxparam{\DUrole{n}{a\_row}\DUrole{p}{:}\DUrole{w}{ }\DUrole{n}{{\hyperref[\detokenize{eezz:eezz.table.TTableRow}]{\sphinxcrossref{TTableRow}}}}}}{{ $\rightarrow$ Tag}}
\pysigstopsignatures
\sphinxAtStartPar
This operation add fixed cells to the table.
Cells which are not included as template for table data are used to add a constant info to the row
\begin{quote}\begin{description}
\sphinxlineitem{Parameters}\begin{itemize}
\item {} 
\sphinxAtStartPar
\sphinxstyleliteralstrong{\sphinxupquote{a\_html\_cells}} – A list of cells to build up a row

\item {} 
\sphinxAtStartPar
\sphinxstyleliteralstrong{\sphinxupquote{a\_tag}} – The parent containing the templates for the row

\item {} 
\sphinxAtStartPar
\sphinxstyleliteralstrong{\sphinxupquote{a\_row}} – The table row values to insert

\end{itemize}

\sphinxlineitem{Returns}
\sphinxAtStartPar
The row with values rendered to HZML

\end{description}\end{quote}

\end{fulllineitems}\end{savenotes}

\index{generate\_html\_table() (eezz.http\_agent.THttpAgent method)@\spxentry{generate\_html\_table()}\spxextra{eezz.http\_agent.THttpAgent method}}

\begin{savenotes}\begin{fulllineitems}
\phantomsection\label{\detokenize{eezz:eezz.http_agent.THttpAgent.generate_html_table}}
\pysigstartsignatures
\pysiglinewithargsret{\sphinxbfcode{\sphinxupquote{generate\_html\_table}}}{\sphinxparam{\DUrole{n}{a\_table\_tag}\DUrole{p}{:}\DUrole{w}{ }\DUrole{n}{Tag}}}{{ $\rightarrow$ dict}}
\pysigstopsignatures
\sphinxAtStartPar
Generates a table structure in four steps
\begin{enumerate}
\sphinxsetlistlabels{\arabic}{enumi}{enumii}{}{.}%
\item {} 
\sphinxAtStartPar
Get the column order and the viewport

\item {} 
\sphinxAtStartPar
Get the row templates

\item {} 
\sphinxAtStartPar
Evaluate the table cells

\item {} 
\sphinxAtStartPar
Send the result separated by table main elements

\end{enumerate}
\begin{quote}\begin{description}
\sphinxlineitem{Parameters}
\sphinxAtStartPar
\sphinxstyleliteralstrong{\sphinxupquote{a\_table\_tag}} – The parent table tag to produce the output

\end{description}\end{quote}

\end{fulllineitems}\end{savenotes}

\index{handle\_download() (eezz.http\_agent.THttpAgent method)@\spxentry{handle\_download()}\spxextra{eezz.http\_agent.THttpAgent method}}

\begin{savenotes}\begin{fulllineitems}
\phantomsection\label{\detokenize{eezz:eezz.http_agent.THttpAgent.handle_download}}
\pysigstartsignatures
\pysiglinewithargsret{\sphinxbfcode{\sphinxupquote{handle\_download}}}{\sphinxparam{\DUrole{n}{request\_data}\DUrole{p}{:}\DUrole{w}{ }\DUrole{n}{dict}}\sphinxparamcomma \sphinxparam{\DUrole{n}{raw\_data}\DUrole{p}{:}\DUrole{w}{ }\DUrole{n}{Any}}}{{ $\rightarrow$ str}}
\pysigstopsignatures
\sphinxAtStartPar
Handle file downloads: The browser slices the file into chunks and the agent has to
re\sphinxhyphen{}arrange the stream using the file name and the sequence number
\begin{quote}\begin{description}
\sphinxlineitem{Parameters}\begin{itemize}
\item {} 
\sphinxAtStartPar
\sphinxstyleliteralstrong{\sphinxupquote{request\_data}} – The request data are encoded in dictionary format

\item {} 
\sphinxAtStartPar
\sphinxstyleliteralstrong{\sphinxupquote{raw\_data}} – The rae data chunk to download

\end{itemize}

\sphinxlineitem{Returns}
\sphinxAtStartPar
Progress information to the update destination of the event

\end{description}\end{quote}

\end{fulllineitems}\end{savenotes}

\index{handle\_request() (eezz.http\_agent.THttpAgent method)@\spxentry{handle\_request()}\spxextra{eezz.http\_agent.THttpAgent method}}

\begin{savenotes}\begin{fulllineitems}
\phantomsection\label{\detokenize{eezz:eezz.http_agent.THttpAgent.handle_request}}
\pysigstartsignatures
\pysiglinewithargsret{\sphinxbfcode{\sphinxupquote{handle\_request}}}{\sphinxparam{\DUrole{n}{request\_data}\DUrole{p}{:}\DUrole{w}{ }\DUrole{n}{dict}}}{{ $\rightarrow$ str\DUrole{w}{ }\DUrole{p}{|}\DUrole{w}{ }None}}
\pysigstopsignatures
\sphinxAtStartPar
Handle WEB socket requests
\begin{itemize}
\item {} 
\sphinxAtStartPar
\sphinxstylestrong{initialize}: The browser sends the complete HTML for analysis.

\item {} 
\sphinxAtStartPar
\sphinxstylestrong{call}: The request issues a method call and the result is send back to the browser

\end{itemize}
\begin{quote}\begin{description}
\sphinxlineitem{Parameters}
\sphinxAtStartPar
\sphinxstyleliteralstrong{\sphinxupquote{request\_data}} – The request send by the browser

\sphinxlineitem{Returns}
\sphinxAtStartPar
Response in JSON stream, containing valid HTML parts for the browser

\end{description}\end{quote}

\end{fulllineitems}\end{savenotes}

\index{setup\_download() (eezz.http\_agent.THttpAgent method)@\spxentry{setup\_download()}\spxextra{eezz.http\_agent.THttpAgent method}}

\begin{savenotes}\begin{fulllineitems}
\phantomsection\label{\detokenize{eezz:eezz.http_agent.THttpAgent.setup_download}}
\pysigstartsignatures
\pysiglinewithargsret{\sphinxbfcode{\sphinxupquote{setup\_download}}}{\sphinxparam{\DUrole{n}{request\_data}\DUrole{p}{:}\DUrole{w}{ }\DUrole{n}{dict}}}{{ $\rightarrow$ str}}
\pysigstopsignatures
\end{fulllineitems}\end{savenotes}


\end{fulllineitems}\end{savenotes}



\subsection{eezz.seccom module}
\label{\detokenize{eezz:module-eezz.seccom}}\label{\detokenize{eezz:eezz-seccom-module}}\index{module@\spxentry{module}!eezz.seccom@\spxentry{eezz.seccom}}\index{eezz.seccom@\spxentry{eezz.seccom}!module@\spxentry{module}}
\sphinxAtStartPar
Copyright (C) 2015 www.EEZZ.biz (haftungsbeschränkt)

\sphinxAtStartPar
TSecureSocket:
Implements secure communication with eezz server
using RSA and AES encryption
\index{TSecureSocket (class in eezz.seccom)@\spxentry{TSecureSocket}\spxextra{class in eezz.seccom}}

\begin{savenotes}\begin{fulllineitems}
\phantomsection\label{\detokenize{eezz:eezz.seccom.TSecureSocket}}
\pysigstartsignatures
\pysigline{\sphinxbfcode{\sphinxupquote{class\DUrole{w}{ }}}\sphinxcode{\sphinxupquote{eezz.seccom.}}\sphinxbfcode{\sphinxupquote{TSecureSocket}}}
\pysigstopsignatures
\sphinxAtStartPar
Bases: \sphinxcode{\sphinxupquote{object}}
\index{send\_request() (eezz.seccom.TSecureSocket method)@\spxentry{send\_request()}\spxextra{eezz.seccom.TSecureSocket method}}

\begin{savenotes}\begin{fulllineitems}
\phantomsection\label{\detokenize{eezz:eezz.seccom.TSecureSocket.send_request}}
\pysigstartsignatures
\pysiglinewithargsret{\sphinxbfcode{\sphinxupquote{send\_request}}}{\sphinxparam{\DUrole{n}{a\_action}}\sphinxparamcomma \sphinxparam{\DUrole{n}{a\_header}\DUrole{o}{=}\DUrole{default_value}{None}}\sphinxparamcomma \sphinxparam{\DUrole{n}{a\_data}\DUrole{o}{=}\DUrole{default_value}{None}}}{}
\pysigstopsignatures
\end{fulllineitems}\end{savenotes}


\end{fulllineitems}\end{savenotes}



\subsection{eezz.server module}
\label{\detokenize{eezz:module-eezz.server}}\label{\detokenize{eezz:eezz-server-module}}\index{module@\spxentry{module}!eezz.server@\spxentry{eezz.server}}\index{eezz.server@\spxentry{eezz.server}!module@\spxentry{module}}
\sphinxAtStartPar
EezzServer:
High speed application development and
high speed execution based on HTML5

\sphinxAtStartPar
Copyright (C) 2015  Albert Zedlitz

\sphinxAtStartPar
This program is free software: you can redistribute it and/or modify
it under the terms of the GNU General Public License as published by
the Free Software Foundation, either version 3 of the License, or
(at your option) any later version.

\sphinxAtStartPar
This program is distributed in the hope that it will be useful,
but WITHOUT ANY WARRANTY; without even the implied warranty of
MERCHANTABILITY or FITNESS FOR A PARTICULAR PURPOSE.  See the
GNU General Public License for more details.

\sphinxAtStartPar
You should have received a copy of the GNU General Public License
along with this program.  If not, see <\sphinxurl{http://www.gnu.org/licenses/}>.
\index{THttpHandler (class in eezz.server)@\spxentry{THttpHandler}\spxextra{class in eezz.server}}

\begin{savenotes}\begin{fulllineitems}
\phantomsection\label{\detokenize{eezz:eezz.server.THttpHandler}}
\pysigstartsignatures
\pysiglinewithargsret{\sphinxbfcode{\sphinxupquote{class\DUrole{w}{ }}}\sphinxcode{\sphinxupquote{eezz.server.}}\sphinxbfcode{\sphinxupquote{THttpHandler}}}{\sphinxparam{\DUrole{n}{request}}\sphinxparamcomma \sphinxparam{\DUrole{n}{client\_address}}\sphinxparamcomma \sphinxparam{\DUrole{n}{server}}}{}
\pysigstopsignatures
\sphinxAtStartPar
Bases: \sphinxcode{\sphinxupquote{SimpleHTTPRequestHandler}}

\sphinxAtStartPar
HTTP handler for incoming requests
\index{do\_GET() (eezz.server.THttpHandler method)@\spxentry{do\_GET()}\spxextra{eezz.server.THttpHandler method}}

\begin{savenotes}\begin{fulllineitems}
\phantomsection\label{\detokenize{eezz:eezz.server.THttpHandler.do_GET}}
\pysigstartsignatures
\pysiglinewithargsret{\sphinxbfcode{\sphinxupquote{do\_GET}}}{}{}
\pysigstopsignatures
\sphinxAtStartPar
handle GET request

\end{fulllineitems}\end{savenotes}

\index{do\_POST() (eezz.server.THttpHandler method)@\spxentry{do\_POST()}\spxextra{eezz.server.THttpHandler method}}

\begin{savenotes}\begin{fulllineitems}
\phantomsection\label{\detokenize{eezz:eezz.server.THttpHandler.do_POST}}
\pysigstartsignatures
\pysiglinewithargsret{\sphinxbfcode{\sphinxupquote{do\_POST}}}{}{}
\pysigstopsignatures
\sphinxAtStartPar
handle POST request

\end{fulllineitems}\end{savenotes}

\index{handle\_request() (eezz.server.THttpHandler method)@\spxentry{handle\_request()}\spxextra{eezz.server.THttpHandler method}}

\begin{savenotes}\begin{fulllineitems}
\phantomsection\label{\detokenize{eezz:eezz.server.THttpHandler.handle_request}}
\pysigstartsignatures
\pysiglinewithargsret{\sphinxbfcode{\sphinxupquote{handle\_request}}}{}{}
\pysigstopsignatures
\sphinxAtStartPar
handle GET and POST requests

\end{fulllineitems}\end{savenotes}

\index{shutdown() (eezz.server.THttpHandler method)@\spxentry{shutdown()}\spxextra{eezz.server.THttpHandler method}}

\begin{savenotes}\begin{fulllineitems}
\phantomsection\label{\detokenize{eezz:eezz.server.THttpHandler.shutdown}}
\pysigstartsignatures
\pysiglinewithargsret{\sphinxbfcode{\sphinxupquote{shutdown}}}{\sphinxparam{\DUrole{n}{args}\DUrole{p}{:}\DUrole{w}{ }\DUrole{n}{int}\DUrole{w}{ }\DUrole{o}{=}\DUrole{w}{ }\DUrole{default_value}{0}}}{}
\pysigstopsignatures
\end{fulllineitems}\end{savenotes}


\end{fulllineitems}\end{savenotes}

\index{TWebServer (class in eezz.server)@\spxentry{TWebServer}\spxextra{class in eezz.server}}

\begin{savenotes}\begin{fulllineitems}
\phantomsection\label{\detokenize{eezz:eezz.server.TWebServer}}
\pysigstartsignatures
\pysiglinewithargsret{\sphinxbfcode{\sphinxupquote{class\DUrole{w}{ }}}\sphinxcode{\sphinxupquote{eezz.server.}}\sphinxbfcode{\sphinxupquote{TWebServer}}}{\sphinxparam{\DUrole{n}{a\_server\_address}}\sphinxparamcomma \sphinxparam{\DUrole{n}{a\_http\_handler}}\sphinxparamcomma \sphinxparam{\DUrole{n}{a\_web\_socket}}}{}
\pysigstopsignatures
\sphinxAtStartPar
Bases: \sphinxcode{\sphinxupquote{HTTPServer}}

\sphinxAtStartPar
WEB Server encapsulate the WEB socket implementation
\index{shutdown() (eezz.server.TWebServer method)@\spxentry{shutdown()}\spxextra{eezz.server.TWebServer method}}

\begin{savenotes}\begin{fulllineitems}
\phantomsection\label{\detokenize{eezz:eezz.server.TWebServer.shutdown}}
\pysigstartsignatures
\pysiglinewithargsret{\sphinxbfcode{\sphinxupquote{shutdown}}}{}{}
\pysigstopsignatures
\sphinxAtStartPar
Stops the serve\_forever loop.

\sphinxAtStartPar
Blocks until the loop has finished. This must be called while
serve\_forever() is running in another thread, or it will
deadlock.

\end{fulllineitems}\end{savenotes}


\end{fulllineitems}\end{savenotes}

\index{shutdown\_function() (in module eezz.server)@\spxentry{shutdown\_function()}\spxextra{in module eezz.server}}

\begin{savenotes}\begin{fulllineitems}
\phantomsection\label{\detokenize{eezz:eezz.server.shutdown_function}}
\pysigstartsignatures
\pysiglinewithargsret{\sphinxcode{\sphinxupquote{eezz.server.}}\sphinxbfcode{\sphinxupquote{shutdown\_function}}}{\sphinxparam{\DUrole{n}{handler}\DUrole{p}{:}\DUrole{w}{ }\DUrole{n}{{\hyperref[\detokenize{eezz:eezz.server.THttpHandler}]{\sphinxcrossref{THttpHandler}}}}}}{}
\pysigstopsignatures
\end{fulllineitems}\end{savenotes}



\subsection{eezz.service module}
\label{\detokenize{eezz:module-eezz.service}}\label{\detokenize{eezz:eezz-service-module}}\index{module@\spxentry{module}!eezz.service@\spxentry{eezz.service}}\index{eezz.service@\spxentry{eezz.service}!module@\spxentry{module}}
\sphinxAtStartPar
This module implements the following classes:
\begin{itemize}
\item {} 
\sphinxAtStartPar
\sphinxstylestrong{TGlobalService}: Container for global environment

\item {} 
\sphinxAtStartPar
\sphinxstylestrong{TService}: A singleton for TGlobalService

\item {} 
\sphinxAtStartPar
\sphinxstylestrong{TServiceCompiler}: A Lark compiler for HTML EEZZ extensions

\item {} 
\sphinxAtStartPar
\sphinxstylestrong{TTranslate}: Extract translation info from HTML to create a POT file

\item {} 
\sphinxAtStartPar
\sphinxstylestrong{TQuery}: Class representing the query of an HTML request

\end{itemize}
\index{TGlobal (class in eezz.service)@\spxentry{TGlobal}\spxextra{class in eezz.service}}

\begin{savenotes}\begin{fulllineitems}
\phantomsection\label{\detokenize{eezz:eezz.service.TGlobal}}
\pysigstartsignatures
\pysigline{\sphinxbfcode{\sphinxupquote{class\DUrole{w}{ }}}\sphinxcode{\sphinxupquote{eezz.service.}}\sphinxbfcode{\sphinxupquote{TGlobal}}}
\pysigstopsignatures
\sphinxAtStartPar
Bases: \sphinxcode{\sphinxupquote{object}}
\index{get\_instance() (eezz.service.TGlobal class method)@\spxentry{get\_instance()}\spxextra{eezz.service.TGlobal class method}}

\begin{savenotes}\begin{fulllineitems}
\phantomsection\label{\detokenize{eezz:eezz.service.TGlobal.get_instance}}
\pysigstartsignatures
\pysiglinewithargsret{\sphinxbfcode{\sphinxupquote{classmethod\DUrole{w}{ }}}\sphinxbfcode{\sphinxupquote{get\_instance}}}{\sphinxparam{\DUrole{n}{cls\_type}}}{}
\pysigstopsignatures
\end{fulllineitems}\end{savenotes}

\index{instances (eezz.service.TGlobal attribute)@\spxentry{instances}\spxextra{eezz.service.TGlobal attribute}}

\begin{savenotes}\begin{fulllineitems}
\phantomsection\label{\detokenize{eezz:eezz.service.TGlobal.instances}}
\pysigstartsignatures
\pysigline{\sphinxbfcode{\sphinxupquote{instances}}\sphinxbfcode{\sphinxupquote{\DUrole{p}{:}\DUrole{w}{ }dict}}\sphinxbfcode{\sphinxupquote{\DUrole{w}{ }\DUrole{p}{=}\DUrole{w}{ }\{\}}}}
\pysigstopsignatures
\end{fulllineitems}\end{savenotes}


\end{fulllineitems}\end{savenotes}

\index{TQuery (class in eezz.service)@\spxentry{TQuery}\spxextra{class in eezz.service}}

\begin{savenotes}\begin{fulllineitems}
\phantomsection\label{\detokenize{eezz:eezz.service.TQuery}}
\pysigstartsignatures
\pysiglinewithargsret{\sphinxbfcode{\sphinxupquote{class\DUrole{w}{ }}}\sphinxcode{\sphinxupquote{eezz.service.}}\sphinxbfcode{\sphinxupquote{TQuery}}}{\sphinxparam{\DUrole{n}{query}\DUrole{p}{:}\DUrole{w}{ }\DUrole{n}{dict}}}{}
\pysigstopsignatures
\sphinxAtStartPar
Bases: \sphinxcode{\sphinxupquote{object}}

\sphinxAtStartPar
Transfer the HTTP query to class attributes
\begin{quote}\begin{description}
\sphinxlineitem{Parameters}
\sphinxAtStartPar
\sphinxstyleliteralstrong{\sphinxupquote{query}} – The query string in dictionary format

\end{description}\end{quote}

\end{fulllineitems}\end{savenotes}

\index{TService (class in eezz.service)@\spxentry{TService}\spxextra{class in eezz.service}}

\begin{savenotes}\begin{fulllineitems}
\phantomsection\label{\detokenize{eezz:eezz.service.TService}}
\pysigstartsignatures
\pysiglinewithargsret{\sphinxbfcode{\sphinxupquote{class\DUrole{w}{ }}}\sphinxcode{\sphinxupquote{eezz.service.}}\sphinxbfcode{\sphinxupquote{TService}}}{\sphinxparam{\DUrole{o}{*}}\sphinxparamcomma \sphinxparam{\DUrole{n}{root\_path}\DUrole{p}{:}\DUrole{w}{ }\DUrole{n}{Path}\DUrole{w}{ }\DUrole{o}{=}\DUrole{w}{ }\DUrole{default_value}{None}}\sphinxparamcomma \sphinxparam{\DUrole{n}{document\_path}\DUrole{p}{:}\DUrole{w}{ }\DUrole{n}{Path}\DUrole{w}{ }\DUrole{o}{=}\DUrole{w}{ }\DUrole{default_value}{None}}\sphinxparamcomma \sphinxparam{\DUrole{n}{application\_path}\DUrole{p}{:}\DUrole{w}{ }\DUrole{n}{Path}\DUrole{w}{ }\DUrole{o}{=}\DUrole{w}{ }\DUrole{default_value}{None}}\sphinxparamcomma \sphinxparam{\DUrole{n}{public\_path}\DUrole{p}{:}\DUrole{w}{ }\DUrole{n}{Path}\DUrole{w}{ }\DUrole{o}{=}\DUrole{w}{ }\DUrole{default_value}{None}}\sphinxparamcomma \sphinxparam{\DUrole{n}{resource\_path}\DUrole{p}{:}\DUrole{w}{ }\DUrole{n}{Path}\DUrole{w}{ }\DUrole{o}{=}\DUrole{w}{ }\DUrole{default_value}{None}}\sphinxparamcomma \sphinxparam{\DUrole{n}{locales\_path}\DUrole{p}{:}\DUrole{w}{ }\DUrole{n}{Path}\DUrole{w}{ }\DUrole{o}{=}\DUrole{w}{ }\DUrole{default_value}{None}}\sphinxparamcomma \sphinxparam{\DUrole{n}{host}\DUrole{p}{:}\DUrole{w}{ }\DUrole{n}{str}\DUrole{w}{ }\DUrole{o}{=}\DUrole{w}{ }\DUrole{default_value}{'localhost'}}\sphinxparamcomma \sphinxparam{\DUrole{n}{websocket\_addr}\DUrole{p}{:}\DUrole{w}{ }\DUrole{n}{int}\DUrole{w}{ }\DUrole{o}{=}\DUrole{w}{ }\DUrole{default_value}{8100}}\sphinxparamcomma \sphinxparam{\DUrole{n}{global\_objects}\DUrole{p}{:}\DUrole{w}{ }\DUrole{n}{dict}\DUrole{w}{ }\DUrole{o}{=}\DUrole{w}{ }\DUrole{default_value}{None}}\sphinxparamcomma \sphinxparam{\DUrole{n}{translate}\DUrole{p}{:}\DUrole{w}{ }\DUrole{n}{bool}\DUrole{w}{ }\DUrole{o}{=}\DUrole{w}{ }\DUrole{default_value}{False}}\sphinxparamcomma \sphinxparam{\DUrole{n}{async\_methods}\DUrole{p}{:}\DUrole{w}{ }\DUrole{n}{Dict\DUrole{p}{{[}}Callable\DUrole{p}{,}\DUrole{w}{ }Thread\DUrole{p}{{]}}}\DUrole{w}{ }\DUrole{o}{=}\DUrole{w}{ }\DUrole{default_value}{None}}\sphinxparamcomma \sphinxparam{\DUrole{n}{private\_key}\DUrole{p}{:}\DUrole{w}{ }\DUrole{n}{RsaKey}\DUrole{w}{ }\DUrole{o}{=}\DUrole{w}{ }\DUrole{default_value}{None}}\sphinxparamcomma \sphinxparam{\DUrole{n}{public\_key}\DUrole{p}{:}\DUrole{w}{ }\DUrole{n}{RsaKey}\DUrole{w}{ }\DUrole{o}{=}\DUrole{w}{ }\DUrole{default_value}{None}}\sphinxparamcomma \sphinxparam{\DUrole{n}{database\_path}\DUrole{p}{:}\DUrole{w}{ }\DUrole{n}{Path}\DUrole{w}{ }\DUrole{o}{=}\DUrole{w}{ }\DUrole{default_value}{None}}\sphinxparamcomma \sphinxparam{\DUrole{n}{eezz\_service\_id}\DUrole{p}{:}\DUrole{w}{ }\DUrole{n}{str}\DUrole{w}{ }\DUrole{o}{=}\DUrole{w}{ }\DUrole{default_value}{None}}}{}
\pysigstopsignatures
\sphinxAtStartPar
Bases: \sphinxcode{\sphinxupquote{object}}

\sphinxAtStartPar
Container for global environment
\index{application\_path (eezz.service.TService attribute)@\spxentry{application\_path}\spxextra{eezz.service.TService attribute}}

\begin{savenotes}\begin{fulllineitems}
\phantomsection\label{\detokenize{eezz:eezz.service.TService.application_path}}
\pysigstartsignatures
\pysigline{\sphinxbfcode{\sphinxupquote{application\_path}}\sphinxbfcode{\sphinxupquote{\DUrole{p}{:}\DUrole{w}{ }Path}}\sphinxbfcode{\sphinxupquote{\DUrole{w}{ }\DUrole{p}{=}\DUrole{w}{ }None}}}
\pysigstopsignatures
\sphinxAtStartPar
Path to applications using the browser interface

\end{fulllineitems}\end{savenotes}

\index{assign\_object() (eezz.service.TService method)@\spxentry{assign\_object()}\spxextra{eezz.service.TService method}}

\begin{savenotes}\begin{fulllineitems}
\phantomsection\label{\detokenize{eezz:eezz.service.TService.assign_object}}
\pysigstartsignatures
\pysiglinewithargsret{\sphinxbfcode{\sphinxupquote{assign\_object}}}{\sphinxparam{\DUrole{n}{obj\_id}\DUrole{p}{:}\DUrole{w}{ }\DUrole{n}{str}}\sphinxparamcomma \sphinxparam{\DUrole{n}{description}\DUrole{p}{:}\DUrole{w}{ }\DUrole{n}{str}}\sphinxparamcomma \sphinxparam{\DUrole{n}{attrs}\DUrole{p}{:}\DUrole{w}{ }\DUrole{n}{dict}}\sphinxparamcomma \sphinxparam{\DUrole{n}{a\_tag}\DUrole{p}{:}\DUrole{w}{ }\DUrole{n}{Tag}\DUrole{w}{ }\DUrole{o}{=}\DUrole{w}{ }\DUrole{default_value}{None}}}{{ $\rightarrow$ None}}
\pysigstopsignatures
\sphinxAtStartPar
\phantomsection\label{\detokenize{eezz:assign-object}}assign\_object Assigns an object to an HTML tag
\begin{quote}\begin{description}
\sphinxlineitem{Raises}\begin{itemize}
\item {} 
\sphinxAtStartPar
\sphinxstyleliteralstrong{\sphinxupquote{IndexError}} – description systax does not match

\item {} 
\sphinxAtStartPar
\sphinxstyleliteralstrong{\sphinxupquote{AttributeError}} – Class not found

\end{itemize}

\sphinxlineitem{Parameters}\begin{itemize}
\item {} 
\sphinxAtStartPar
\sphinxstyleliteralstrong{\sphinxupquote{obj\_id}} – Unique object\sphinxhyphen{}id

\item {} 
\sphinxAtStartPar
\sphinxstyleliteralstrong{\sphinxupquote{description}} – Path to the class: <directory>.<module>.<class>

\item {} 
\sphinxAtStartPar
\sphinxstyleliteralstrong{\sphinxupquote{attrs}} – Attributes for the constructor

\item {} 
\sphinxAtStartPar
\sphinxstyleliteralstrong{\sphinxupquote{a\_tag}} – Parent tag which handles an instance of this object

\end{itemize}

\end{description}\end{quote}

\end{fulllineitems}\end{savenotes}

\index{async\_methods (eezz.service.TService attribute)@\spxentry{async\_methods}\spxextra{eezz.service.TService attribute}}

\begin{savenotes}\begin{fulllineitems}
\phantomsection\label{\detokenize{eezz:eezz.service.TService.async_methods}}
\pysigstartsignatures
\pysigline{\sphinxbfcode{\sphinxupquote{async\_methods}}\sphinxbfcode{\sphinxupquote{\DUrole{p}{:}\DUrole{w}{ }Dict\DUrole{p}{{[}}Callable\DUrole{p}{,}\DUrole{w}{ }Thread\DUrole{p}{{]}}}}\sphinxbfcode{\sphinxupquote{\DUrole{w}{ }\DUrole{p}{=}\DUrole{w}{ }None}}}
\pysigstopsignatures
\end{fulllineitems}\end{savenotes}

\index{database\_path (eezz.service.TService attribute)@\spxentry{database\_path}\spxextra{eezz.service.TService attribute}}

\begin{savenotes}\begin{fulllineitems}
\phantomsection\label{\detokenize{eezz:eezz.service.TService.database_path}}
\pysigstartsignatures
\pysigline{\sphinxbfcode{\sphinxupquote{database\_path}}\sphinxbfcode{\sphinxupquote{\DUrole{p}{:}\DUrole{w}{ }Path}}\sphinxbfcode{\sphinxupquote{\DUrole{w}{ }\DUrole{p}{=}\DUrole{w}{ }None}}}
\pysigstopsignatures
\end{fulllineitems}\end{savenotes}

\index{document\_path (eezz.service.TService attribute)@\spxentry{document\_path}\spxextra{eezz.service.TService attribute}}

\begin{savenotes}\begin{fulllineitems}
\phantomsection\label{\detokenize{eezz:eezz.service.TService.document_path}}
\pysigstartsignatures
\pysigline{\sphinxbfcode{\sphinxupquote{document\_path}}\sphinxbfcode{\sphinxupquote{\DUrole{p}{:}\DUrole{w}{ }Path}}\sphinxbfcode{\sphinxupquote{\DUrole{w}{ }\DUrole{p}{=}\DUrole{w}{ }None}}}
\pysigstopsignatures
\sphinxAtStartPar
Path to EEZZ documents

\end{fulllineitems}\end{savenotes}

\index{eezz\_service\_id (eezz.service.TService attribute)@\spxentry{eezz\_service\_id}\spxextra{eezz.service.TService attribute}}

\begin{savenotes}\begin{fulllineitems}
\phantomsection\label{\detokenize{eezz:eezz.service.TService.eezz_service_id}}
\pysigstartsignatures
\pysigline{\sphinxbfcode{\sphinxupquote{eezz\_service\_id}}\sphinxbfcode{\sphinxupquote{\DUrole{p}{:}\DUrole{w}{ }str}}\sphinxbfcode{\sphinxupquote{\DUrole{w}{ }\DUrole{p}{=}\DUrole{w}{ }None}}}
\pysigstopsignatures
\end{fulllineitems}\end{savenotes}

\index{get\_instance() (eezz.service.TService class method)@\spxentry{get\_instance()}\spxextra{eezz.service.TService class method}}

\begin{savenotes}\begin{fulllineitems}
\phantomsection\label{\detokenize{eezz:eezz.service.TService.get_instance}}
\pysigstartsignatures
\pysiglinewithargsret{\sphinxbfcode{\sphinxupquote{classmethod\DUrole{w}{ }}}\sphinxbfcode{\sphinxupquote{get\_instance}}}{\sphinxparam{\DUrole{n}{class\_type}\DUrole{o}{=}\DUrole{default_value}{None}}}{}
\pysigstopsignatures
\end{fulllineitems}\end{savenotes}

\index{get\_method() (eezz.service.TService method)@\spxentry{get\_method()}\spxextra{eezz.service.TService method}}

\begin{savenotes}\begin{fulllineitems}
\phantomsection\label{\detokenize{eezz:eezz.service.TService.get_method}}
\pysigstartsignatures
\pysiglinewithargsret{\sphinxbfcode{\sphinxupquote{get\_method}}}{\sphinxparam{\DUrole{n}{obj\_id}\DUrole{p}{:}\DUrole{w}{ }\DUrole{n}{str}}\sphinxparamcomma \sphinxparam{\DUrole{n}{a\_method\_name}\DUrole{p}{:}\DUrole{w}{ }\DUrole{n}{str}}}{{ $\rightarrow$ tuple}}
\pysigstopsignatures
\sphinxAtStartPar
Get a method by name for a given object
\begin{quote}\begin{description}
\sphinxlineitem{Raises}
\sphinxAtStartPar
\sphinxstyleliteralstrong{\sphinxupquote{AttributeError}} – Class has no method with the given name

\sphinxlineitem{Parameters}\begin{itemize}
\item {} 
\sphinxAtStartPar
\sphinxstyleliteralstrong{\sphinxupquote{obj\_id}} – Unique hash\sphinxhyphen{}ID for object as stored in {\hyperref[\detokenize{eezz:eezz.service.TService.assign_object}]{\sphinxcrossref{\sphinxcode{\sphinxupquote{eezz.service.TService.assign\_object()}}}}}

\item {} 
\sphinxAtStartPar
\sphinxstyleliteralstrong{\sphinxupquote{a\_method\_name}} – 

\end{itemize}

\sphinxlineitem{Returns}
\sphinxAtStartPar
tuple(object, method, parent\sphinxhyphen{}tag)

\end{description}\end{quote}

\end{fulllineitems}\end{savenotes}

\index{get\_object() (eezz.service.TService method)@\spxentry{get\_object()}\spxextra{eezz.service.TService method}}

\begin{savenotes}\begin{fulllineitems}
\phantomsection\label{\detokenize{eezz:eezz.service.TService.get_object}}
\pysigstartsignatures
\pysiglinewithargsret{\sphinxbfcode{\sphinxupquote{get\_object}}}{\sphinxparam{\DUrole{n}{obj\_id}\DUrole{p}{:}\DUrole{w}{ }\DUrole{n}{str}}}{{ $\rightarrow$ Any}}
\pysigstopsignatures
\sphinxAtStartPar
Get the object for a given ID
\begin{quote}\begin{description}
\sphinxlineitem{Parameters}
\sphinxAtStartPar
\sphinxstyleliteralstrong{\sphinxupquote{obj\_id}} – Unique hash\sphinxhyphen{}ID for object as stored in \sphinxcode{\sphinxupquote{eezz.service.TGlobalService.assign\_object()}}

\sphinxlineitem{Returns}
\sphinxAtStartPar
The assigned object

\end{description}\end{quote}

\end{fulllineitems}\end{savenotes}

\index{global\_objects (eezz.service.TService attribute)@\spxentry{global\_objects}\spxextra{eezz.service.TService attribute}}

\begin{savenotes}\begin{fulllineitems}
\phantomsection\label{\detokenize{eezz:eezz.service.TService.global_objects}}
\pysigstartsignatures
\pysigline{\sphinxbfcode{\sphinxupquote{global\_objects}}\sphinxbfcode{\sphinxupquote{\DUrole{p}{:}\DUrole{w}{ }dict}}\sphinxbfcode{\sphinxupquote{\DUrole{w}{ }\DUrole{p}{=}\DUrole{w}{ }None}}}
\pysigstopsignatures
\end{fulllineitems}\end{savenotes}

\index{host (eezz.service.TService attribute)@\spxentry{host}\spxextra{eezz.service.TService attribute}}

\begin{savenotes}\begin{fulllineitems}
\phantomsection\label{\detokenize{eezz:eezz.service.TService.host}}
\pysigstartsignatures
\pysigline{\sphinxbfcode{\sphinxupquote{host}}\sphinxbfcode{\sphinxupquote{\DUrole{p}{:}\DUrole{w}{ }str}}\sphinxbfcode{\sphinxupquote{\DUrole{w}{ }\DUrole{p}{=}\DUrole{w}{ }'localhost'}}}
\pysigstopsignatures
\end{fulllineitems}\end{savenotes}

\index{locales\_path (eezz.service.TService attribute)@\spxentry{locales\_path}\spxextra{eezz.service.TService attribute}}

\begin{savenotes}\begin{fulllineitems}
\phantomsection\label{\detokenize{eezz:eezz.service.TService.locales_path}}
\pysigstartsignatures
\pysigline{\sphinxbfcode{\sphinxupquote{locales\_path}}\sphinxbfcode{\sphinxupquote{\DUrole{p}{:}\DUrole{w}{ }Path}}\sphinxbfcode{\sphinxupquote{\DUrole{w}{ }\DUrole{p}{=}\DUrole{w}{ }None}}}
\pysigstopsignatures
\end{fulllineitems}\end{savenotes}

\index{private\_key (eezz.service.TService attribute)@\spxentry{private\_key}\spxextra{eezz.service.TService attribute}}

\begin{savenotes}\begin{fulllineitems}
\phantomsection\label{\detokenize{eezz:eezz.service.TService.private_key}}
\pysigstartsignatures
\pysigline{\sphinxbfcode{\sphinxupquote{private\_key}}\sphinxbfcode{\sphinxupquote{\DUrole{p}{:}\DUrole{w}{ }RsaKey}}\sphinxbfcode{\sphinxupquote{\DUrole{w}{ }\DUrole{p}{=}\DUrole{w}{ }None}}}
\pysigstopsignatures
\end{fulllineitems}\end{savenotes}

\index{public\_key (eezz.service.TService attribute)@\spxentry{public\_key}\spxextra{eezz.service.TService attribute}}

\begin{savenotes}\begin{fulllineitems}
\phantomsection\label{\detokenize{eezz:eezz.service.TService.public_key}}
\pysigstartsignatures
\pysigline{\sphinxbfcode{\sphinxupquote{public\_key}}\sphinxbfcode{\sphinxupquote{\DUrole{p}{:}\DUrole{w}{ }RsaKey}}\sphinxbfcode{\sphinxupquote{\DUrole{w}{ }\DUrole{p}{=}\DUrole{w}{ }None}}}
\pysigstopsignatures
\end{fulllineitems}\end{savenotes}

\index{public\_path (eezz.service.TService attribute)@\spxentry{public\_path}\spxextra{eezz.service.TService attribute}}

\begin{savenotes}\begin{fulllineitems}
\phantomsection\label{\detokenize{eezz:eezz.service.TService.public_path}}
\pysigstartsignatures
\pysigline{\sphinxbfcode{\sphinxupquote{public\_path}}\sphinxbfcode{\sphinxupquote{\DUrole{p}{:}\DUrole{w}{ }Path}}\sphinxbfcode{\sphinxupquote{\DUrole{w}{ }\DUrole{p}{=}\DUrole{w}{ }None}}}
\pysigstopsignatures
\end{fulllineitems}\end{savenotes}

\index{resource\_path (eezz.service.TService attribute)@\spxentry{resource\_path}\spxextra{eezz.service.TService attribute}}

\begin{savenotes}\begin{fulllineitems}
\phantomsection\label{\detokenize{eezz:eezz.service.TService.resource_path}}
\pysigstartsignatures
\pysigline{\sphinxbfcode{\sphinxupquote{resource\_path}}\sphinxbfcode{\sphinxupquote{\DUrole{p}{:}\DUrole{w}{ }Path}}\sphinxbfcode{\sphinxupquote{\DUrole{w}{ }\DUrole{p}{=}\DUrole{w}{ }None}}}
\pysigstopsignatures
\end{fulllineitems}\end{savenotes}

\index{root\_path (eezz.service.TService attribute)@\spxentry{root\_path}\spxextra{eezz.service.TService attribute}}

\begin{savenotes}\begin{fulllineitems}
\phantomsection\label{\detokenize{eezz:eezz.service.TService.root_path}}
\pysigstartsignatures
\pysigline{\sphinxbfcode{\sphinxupquote{root\_path}}\sphinxbfcode{\sphinxupquote{\DUrole{p}{:}\DUrole{w}{ }Path}}\sphinxbfcode{\sphinxupquote{\DUrole{w}{ }\DUrole{p}{=}\DUrole{w}{ }None}}}
\pysigstopsignatures
\sphinxAtStartPar
Root path for the HTTP server

\end{fulllineitems}\end{savenotes}

\index{set\_instance() (eezz.service.TService class method)@\spxentry{set\_instance()}\spxextra{eezz.service.TService class method}}

\begin{savenotes}\begin{fulllineitems}
\phantomsection\label{\detokenize{eezz:eezz.service.TService.set_instance}}
\pysigstartsignatures
\pysiglinewithargsret{\sphinxbfcode{\sphinxupquote{classmethod\DUrole{w}{ }}}\sphinxbfcode{\sphinxupquote{set\_instance}}}{\sphinxparam{\DUrole{n}{instance}}}{}
\pysigstopsignatures
\end{fulllineitems}\end{savenotes}

\index{singletons (eezz.service.TService attribute)@\spxentry{singletons}\spxextra{eezz.service.TService attribute}}

\begin{savenotes}\begin{fulllineitems}
\phantomsection\label{\detokenize{eezz:eezz.service.TService.singletons}}
\pysigstartsignatures
\pysigline{\sphinxbfcode{\sphinxupquote{singletons}}\sphinxbfcode{\sphinxupquote{\DUrole{p}{:}\DUrole{w}{ }ClassVar\DUrole{p}{{[}}dict\DUrole{p}{{]}}}}\sphinxbfcode{\sphinxupquote{\DUrole{w}{ }\DUrole{p}{=}\DUrole{w}{ }\{\}}}}
\pysigstopsignatures
\end{fulllineitems}\end{savenotes}

\index{translate (eezz.service.TService attribute)@\spxentry{translate}\spxextra{eezz.service.TService attribute}}

\begin{savenotes}\begin{fulllineitems}
\phantomsection\label{\detokenize{eezz:eezz.service.TService.translate}}
\pysigstartsignatures
\pysigline{\sphinxbfcode{\sphinxupquote{translate}}\sphinxbfcode{\sphinxupquote{\DUrole{p}{:}\DUrole{w}{ }bool}}\sphinxbfcode{\sphinxupquote{\DUrole{w}{ }\DUrole{p}{=}\DUrole{w}{ }False}}}
\pysigstopsignatures
\end{fulllineitems}\end{savenotes}

\index{websocket\_addr (eezz.service.TService attribute)@\spxentry{websocket\_addr}\spxextra{eezz.service.TService attribute}}

\begin{savenotes}\begin{fulllineitems}
\phantomsection\label{\detokenize{eezz:eezz.service.TService.websocket_addr}}
\pysigstartsignatures
\pysigline{\sphinxbfcode{\sphinxupquote{websocket\_addr}}\sphinxbfcode{\sphinxupquote{\DUrole{p}{:}\DUrole{w}{ }int}}\sphinxbfcode{\sphinxupquote{\DUrole{w}{ }\DUrole{p}{=}\DUrole{w}{ }8100}}}
\pysigstopsignatures
\end{fulllineitems}\end{savenotes}


\end{fulllineitems}\end{savenotes}

\index{TServiceCompiler (class in eezz.service)@\spxentry{TServiceCompiler}\spxextra{class in eezz.service}}

\begin{savenotes}\begin{fulllineitems}
\phantomsection\label{\detokenize{eezz:eezz.service.TServiceCompiler}}
\pysigstartsignatures
\pysiglinewithargsret{\sphinxbfcode{\sphinxupquote{class\DUrole{w}{ }}}\sphinxcode{\sphinxupquote{eezz.service.}}\sphinxbfcode{\sphinxupquote{TServiceCompiler}}}{\sphinxparam{\DUrole{n}{a\_tag}\DUrole{p}{:}\DUrole{w}{ }\DUrole{n}{Tag}}\sphinxparamcomma \sphinxparam{\DUrole{n}{a\_id}\DUrole{p}{:}\DUrole{w}{ }\DUrole{n}{str}\DUrole{w}{ }\DUrole{o}{=}\DUrole{w}{ }\DUrole{default_value}{''}}\sphinxparamcomma \sphinxparam{\DUrole{n}{a\_query}\DUrole{p}{:}\DUrole{w}{ }\DUrole{n}{dict}\DUrole{w}{ }\DUrole{o}{=}\DUrole{w}{ }\DUrole{default_value}{None}}}{}
\pysigstopsignatures
\sphinxAtStartPar
Bases: \sphinxcode{\sphinxupquote{Transformer}}

\sphinxAtStartPar
Transforms the parser tree into a list of dictionaries
The transformer output is in json format
\begin{quote}\begin{description}
\sphinxlineitem{Parameters}\begin{itemize}
\item {} 
\sphinxAtStartPar
\sphinxstyleliteralstrong{\sphinxupquote{a\_tag}} (\sphinxstyleliteralemphasis{\sphinxupquote{BeautifulSoup4.Tag}}) – The parent tag

\item {} 
\sphinxAtStartPar
\sphinxstyleliteralstrong{\sphinxupquote{a\_id}} – A unique object id

\item {} 
\sphinxAtStartPar
\sphinxstyleliteralstrong{\sphinxupquote{a\_query}} – The URL query part

\end{itemize}

\end{description}\end{quote}
\index{assignment() (eezz.service.TServiceCompiler static method)@\spxentry{assignment()}\spxextra{eezz.service.TServiceCompiler static method}}

\begin{savenotes}\begin{fulllineitems}
\phantomsection\label{\detokenize{eezz:eezz.service.TServiceCompiler.assignment}}
\pysigstartsignatures
\pysiglinewithargsret{\sphinxbfcode{\sphinxupquote{static\DUrole{w}{ }}}\sphinxbfcode{\sphinxupquote{assignment}}}{\sphinxparam{\DUrole{n}{item}}}{}
\pysigstopsignatures
\sphinxAtStartPar
Parse ‘assignment’ expression: \sphinxcode{\sphinxupquote{variable = value}}

\end{fulllineitems}\end{savenotes}

\index{download() (eezz.service.TServiceCompiler static method)@\spxentry{download()}\spxextra{eezz.service.TServiceCompiler static method}}

\begin{savenotes}\begin{fulllineitems}
\phantomsection\label{\detokenize{eezz:eezz.service.TServiceCompiler.download}}
\pysigstartsignatures
\pysiglinewithargsret{\sphinxbfcode{\sphinxupquote{static\DUrole{w}{ }}}\sphinxbfcode{\sphinxupquote{download}}}{\sphinxparam{\DUrole{n}{item}}}{}
\pysigstopsignatures
\sphinxAtStartPar
Parse ‘download’ section

\end{fulllineitems}\end{savenotes}

\index{escaped\_str() (eezz.service.TServiceCompiler static method)@\spxentry{escaped\_str()}\spxextra{eezz.service.TServiceCompiler static method}}

\begin{savenotes}\begin{fulllineitems}
\phantomsection\label{\detokenize{eezz:eezz.service.TServiceCompiler.escaped_str}}
\pysigstartsignatures
\pysiglinewithargsret{\sphinxbfcode{\sphinxupquote{static\DUrole{w}{ }}}\sphinxbfcode{\sphinxupquote{escaped\_str}}}{\sphinxparam{\DUrole{n}{item}}}{}
\pysigstopsignatures
\sphinxAtStartPar
Parse an escaped string

\end{fulllineitems}\end{savenotes}

\index{format\_string() (eezz.service.TServiceCompiler static method)@\spxentry{format\_string()}\spxextra{eezz.service.TServiceCompiler static method}}

\begin{savenotes}\begin{fulllineitems}
\phantomsection\label{\detokenize{eezz:eezz.service.TServiceCompiler.format_string}}
\pysigstartsignatures
\pysiglinewithargsret{\sphinxbfcode{\sphinxupquote{static\DUrole{w}{ }}}\sphinxbfcode{\sphinxupquote{format\_string}}}{\sphinxparam{\DUrole{n}{item}}}{}
\pysigstopsignatures
\sphinxAtStartPar
Create a format string: \sphinxcode{\sphinxupquote{\{value\}}}

\end{fulllineitems}\end{savenotes}

\index{format\_value() (eezz.service.TServiceCompiler static method)@\spxentry{format\_value()}\spxextra{eezz.service.TServiceCompiler static method}}

\begin{savenotes}\begin{fulllineitems}
\phantomsection\label{\detokenize{eezz:eezz.service.TServiceCompiler.format_value}}
\pysigstartsignatures
\pysiglinewithargsret{\sphinxbfcode{\sphinxupquote{static\DUrole{w}{ }}}\sphinxbfcode{\sphinxupquote{format\_value}}}{\sphinxparam{\DUrole{n}{item}}}{}
\pysigstopsignatures
\sphinxAtStartPar
Create a format string: \sphinxcode{\sphinxupquote{\{key.value\}}}

\end{fulllineitems}\end{savenotes}

\index{funct\_assignment() (eezz.service.TServiceCompiler method)@\spxentry{funct\_assignment()}\spxextra{eezz.service.TServiceCompiler method}}

\begin{savenotes}\begin{fulllineitems}
\phantomsection\label{\detokenize{eezz:eezz.service.TServiceCompiler.funct_assignment}}
\pysigstartsignatures
\pysiglinewithargsret{\sphinxbfcode{\sphinxupquote{funct\_assignment}}}{\sphinxparam{\DUrole{n}{item}}}{}
\pysigstopsignatures
\sphinxAtStartPar
Parse ‘function’ section

\end{fulllineitems}\end{savenotes}

\index{list\_arguments() (eezz.service.TServiceCompiler static method)@\spxentry{list\_arguments()}\spxextra{eezz.service.TServiceCompiler static method}}

\begin{savenotes}\begin{fulllineitems}
\phantomsection\label{\detokenize{eezz:eezz.service.TServiceCompiler.list_arguments}}
\pysigstartsignatures
\pysiglinewithargsret{\sphinxbfcode{\sphinxupquote{static\DUrole{w}{ }}}\sphinxbfcode{\sphinxupquote{list\_arguments}}}{\sphinxparam{\DUrole{n}{item}}}{}
\pysigstopsignatures
\sphinxAtStartPar
Accumulate arguments for function call

\end{fulllineitems}\end{savenotes}

\index{list\_updates() (eezz.service.TServiceCompiler static method)@\spxentry{list\_updates()}\spxextra{eezz.service.TServiceCompiler static method}}

\begin{savenotes}\begin{fulllineitems}
\phantomsection\label{\detokenize{eezz:eezz.service.TServiceCompiler.list_updates}}
\pysigstartsignatures
\pysiglinewithargsret{\sphinxbfcode{\sphinxupquote{static\DUrole{w}{ }}}\sphinxbfcode{\sphinxupquote{list\_updates}}}{\sphinxparam{\DUrole{n}{item}}}{}
\pysigstopsignatures
\sphinxAtStartPar
Accumulate ‘update’ statements

\end{fulllineitems}\end{savenotes}

\index{post\_init() (eezz.service.TServiceCompiler method)@\spxentry{post\_init()}\spxextra{eezz.service.TServiceCompiler method}}

\begin{savenotes}\begin{fulllineitems}
\phantomsection\label{\detokenize{eezz:eezz.service.TServiceCompiler.post_init}}
\pysigstartsignatures
\pysiglinewithargsret{\sphinxbfcode{\sphinxupquote{post\_init}}}{\sphinxparam{\DUrole{n}{item}}}{}
\pysigstopsignatures
\sphinxAtStartPar
Parse ‘post\sphinxhyphen{}init’ section for function assignment

\end{fulllineitems}\end{savenotes}

\index{qualified\_string() (eezz.service.TServiceCompiler static method)@\spxentry{qualified\_string()}\spxextra{eezz.service.TServiceCompiler static method}}

\begin{savenotes}\begin{fulllineitems}
\phantomsection\label{\detokenize{eezz:eezz.service.TServiceCompiler.qualified_string}}
\pysigstartsignatures
\pysiglinewithargsret{\sphinxbfcode{\sphinxupquote{static\DUrole{w}{ }}}\sphinxbfcode{\sphinxupquote{qualified\_string}}}{\sphinxparam{\DUrole{n}{item}}}{}
\pysigstopsignatures
\sphinxAtStartPar
Parse a qualified string: \sphinxcode{\sphinxupquote{part1.part2.part3}}

\end{fulllineitems}\end{savenotes}

\index{simple\_str() (eezz.service.TServiceCompiler static method)@\spxentry{simple\_str()}\spxextra{eezz.service.TServiceCompiler static method}}

\begin{savenotes}\begin{fulllineitems}
\phantomsection\label{\detokenize{eezz:eezz.service.TServiceCompiler.simple_str}}
\pysigstartsignatures
\pysiglinewithargsret{\sphinxbfcode{\sphinxupquote{static\DUrole{w}{ }}}\sphinxbfcode{\sphinxupquote{simple\_str}}}{\sphinxparam{\DUrole{n}{item}}}{}
\pysigstopsignatures
\sphinxAtStartPar
Parse a string token

\end{fulllineitems}\end{savenotes}

\index{table\_assignment() (eezz.service.TServiceCompiler method)@\spxentry{table\_assignment()}\spxextra{eezz.service.TServiceCompiler method}}

\begin{savenotes}\begin{fulllineitems}
\phantomsection\label{\detokenize{eezz:eezz.service.TServiceCompiler.table_assignment}}
\pysigstartsignatures
\pysiglinewithargsret{\sphinxbfcode{\sphinxupquote{table\_assignment}}}{\sphinxparam{\DUrole{n}{item}}}{}
\pysigstopsignatures
\sphinxAtStartPar
Parse ‘assign’ section, assigning a Python object to an HTML\sphinxhyphen{}Tag
The table assignment uses TQuery to format arguments
In case the arguments are not all present, the format is broken and process continues with default

\end{fulllineitems}\end{savenotes}

\index{template\_section() (eezz.service.TServiceCompiler method)@\spxentry{template\_section()}\spxextra{eezz.service.TServiceCompiler method}}

\begin{savenotes}\begin{fulllineitems}
\phantomsection\label{\detokenize{eezz:eezz.service.TServiceCompiler.template_section}}
\pysigstartsignatures
\pysiglinewithargsret{\sphinxbfcode{\sphinxupquote{template\_section}}}{\sphinxparam{\DUrole{n}{item}}}{}
\pysigstopsignatures
\sphinxAtStartPar
Create tag attributes

\end{fulllineitems}\end{savenotes}

\index{update\_item() (eezz.service.TServiceCompiler static method)@\spxentry{update\_item()}\spxextra{eezz.service.TServiceCompiler static method}}

\begin{savenotes}\begin{fulllineitems}
\phantomsection\label{\detokenize{eezz:eezz.service.TServiceCompiler.update_item}}
\pysigstartsignatures
\pysiglinewithargsret{\sphinxbfcode{\sphinxupquote{static\DUrole{w}{ }}}\sphinxbfcode{\sphinxupquote{update\_item}}}{\sphinxparam{\DUrole{n}{item}}}{}
\pysigstopsignatures
\sphinxAtStartPar
Parse ‘update’ expression

\end{fulllineitems}\end{savenotes}

\index{update\_section() (eezz.service.TServiceCompiler static method)@\spxentry{update\_section()}\spxextra{eezz.service.TServiceCompiler static method}}

\begin{savenotes}\begin{fulllineitems}
\phantomsection\label{\detokenize{eezz:eezz.service.TServiceCompiler.update_section}}
\pysigstartsignatures
\pysiglinewithargsret{\sphinxbfcode{\sphinxupquote{static\DUrole{w}{ }}}\sphinxbfcode{\sphinxupquote{update\_section}}}{\sphinxparam{\DUrole{n}{item}}}{}
\pysigstopsignatures
\sphinxAtStartPar
Parse ‘update’ section

\end{fulllineitems}\end{savenotes}


\end{fulllineitems}\end{savenotes}

\index{TTranslate (class in eezz.service)@\spxentry{TTranslate}\spxextra{class in eezz.service}}

\begin{savenotes}\begin{fulllineitems}
\phantomsection\label{\detokenize{eezz:eezz.service.TTranslate}}
\pysigstartsignatures
\pysigline{\sphinxbfcode{\sphinxupquote{class\DUrole{w}{ }}}\sphinxcode{\sphinxupquote{eezz.service.}}\sphinxbfcode{\sphinxupquote{TTranslate}}}
\pysigstopsignatures
\sphinxAtStartPar
Bases: \sphinxcode{\sphinxupquote{object}}
\index{generate\_pot() (eezz.service.TTranslate static method)@\spxentry{generate\_pot()}\spxextra{eezz.service.TTranslate static method}}

\begin{savenotes}\begin{fulllineitems}
\phantomsection\label{\detokenize{eezz:eezz.service.TTranslate.generate_pot}}
\pysigstartsignatures
\pysiglinewithargsret{\sphinxbfcode{\sphinxupquote{static\DUrole{w}{ }}}\sphinxbfcode{\sphinxupquote{generate\_pot}}}{\sphinxparam{\DUrole{n}{a\_soup}}\sphinxparamcomma \sphinxparam{\DUrole{n}{a\_title}}}{}
\pysigstopsignatures
\sphinxAtStartPar
Generate a POT file from HTML file
\begin{quote}\begin{description}
\sphinxlineitem{Parameters}\begin{itemize}
\item {} 
\sphinxAtStartPar
\sphinxstyleliteralstrong{\sphinxupquote{a\_soup}} – The HTML page for translation

\item {} 
\sphinxAtStartPar
\sphinxstyleliteralstrong{\sphinxupquote{a\_title}} – The file name for the POT file

\end{itemize}

\end{description}\end{quote}

\end{fulllineitems}\end{savenotes}


\end{fulllineitems}\end{savenotes}

\index{test\_parser() (in module eezz.service)@\spxentry{test\_parser()}\spxextra{in module eezz.service}}

\begin{savenotes}\begin{fulllineitems}
\phantomsection\label{\detokenize{eezz:eezz.service.test_parser}}
\pysigstartsignatures
\pysiglinewithargsret{\sphinxcode{\sphinxupquote{eezz.service.}}\sphinxbfcode{\sphinxupquote{test\_parser}}}{\sphinxparam{\DUrole{n}{source}\DUrole{p}{:}\DUrole{w}{ }\DUrole{n}{str}}}{}
\pysigstopsignatures
\end{fulllineitems}\end{savenotes}



\subsection{eezz.session module}
\label{\detokenize{eezz:module-eezz.session}}\label{\detokenize{eezz:eezz-session-module}}\index{module@\spxentry{module}!eezz.session@\spxentry{eezz.session}}\index{eezz.session@\spxentry{eezz.session}!module@\spxentry{module}}
\sphinxAtStartPar
This module implements the following classes
\begin{itemize}
\item {} 
\sphinxAtStartPar
\sphinxstylestrong{TSession}: User session management

\end{itemize}
\index{TSession (class in eezz.session)@\spxentry{TSession}\spxextra{class in eezz.session}}

\begin{savenotes}\begin{fulllineitems}
\phantomsection\label{\detokenize{eezz:eezz.session.TSession}}
\pysigstartsignatures
\pysiglinewithargsret{\sphinxbfcode{\sphinxupquote{class\DUrole{w}{ }}}\sphinxcode{\sphinxupquote{eezz.session.}}\sphinxbfcode{\sphinxupquote{TSession}}}{\sphinxparam{*, column\_names: \textasciitilde{}typing.List{[}str{]}, column\_names\_map: \textasciitilde{}typing.Dict{[}str, \textasciitilde{}table.TTableCell{]} | None = None, column\_names\_alias: \textasciitilde{}typing.Dict{[}str, str{]} | None = None, column\_names\_filter: \textasciitilde{}typing.List{[}int{]} | None = None, column\_descr: \textasciitilde{}typing.List{[}\textasciitilde{}table.TTableColumn{]} = None, table\_index: \textasciitilde{}typing.Dict{[}str, \textasciitilde{}table.TTableRow{]} = None, title: str = 'Table', attrs: dict = None, visible\_items: int = 20, offset: int = 0, selected\_row: \textasciitilde{}table.TTableRow = None, header\_row: \textasciitilde{}table.TTableRow = None, apply\_filter\_column: bool = False, format\_types: dict = None, async\_condition: \textasciitilde{}threading.Condition = <Condition(<unlocked \_thread.RLock object owner=0 count=0>, 0)>, async\_lock: \textasciitilde{}\_thread.allocate\_lock = <unlocked \_thread.lock object>, sid: str = None, address: str = None, name: str = None, desktop\_connected: bool = False, device\_connected: bool = False, paired\_device: \textasciitilde{}table.TTableRow | None = None, bt\_service: \textasciitilde{}blueserv.TBluetoothService | None = None, bt\_devices: \textasciitilde{}blueserv.TBluetooth = None, mb\_devices: \textasciitilde{}mobile.TMobileDevices = None}}{}
\pysigstopsignatures
\sphinxAtStartPar
Bases: \sphinxcode{\sphinxupquote{TTable}}

\sphinxAtStartPar
TSession implements the interface to Windows users. In a first step the user connects to the HTTP server with
SID and NAME.
Within the connect call a thread is started to sync with bluetooth device search, with the intention to connect
to the EEZZ\sphinxhyphen{}App on this device.
Pairing is supported with standard UI interfaces to select the device from GUI and register the user.
After this, the user could choose to store the password on the device to allow automatic lock and unlock feature
of the EEZZ Windows installation.
\begin{quote}\begin{description}
\sphinxlineitem{Parameters}\begin{itemize}
\item {} 
\sphinxAtStartPar
\sphinxstyleliteralstrong{\sphinxupquote{sid}} – Windows user SID

\item {} 
\sphinxAtStartPar
\sphinxstyleliteralstrong{\sphinxupquote{name}} – Windows username

\item {} 
\sphinxAtStartPar
\sphinxstyleliteralstrong{\sphinxupquote{desktop\_connected}} (\sphinxstyleliteralemphasis{\sphinxupquote{bool}}) – Connected to the desktop user

\item {} 
\sphinxAtStartPar
\sphinxstyleliteralstrong{\sphinxupquote{device\_connected}} (\sphinxstyleliteralemphasis{\sphinxupquote{bool}}) – Connected device and desktop user

\item {} 
\sphinxAtStartPar
\sphinxstyleliteralstrong{\sphinxupquote{paired\_device}} ({\hyperref[\detokenize{eezz:eezz.table.TTableRow}]{\sphinxcrossref{\sphinxstyleliteralemphasis{\sphinxupquote{TTableRow}}}}}) – Data of connected device

\item {} 
\sphinxAtStartPar
\sphinxstyleliteralstrong{\sphinxupquote{bt\_service}} ({\hyperref[\detokenize{eezz:eezz.blueserv.TBluetoothService}]{\sphinxcrossref{\sphinxstyleliteralemphasis{\sphinxupquote{TBluetoothService}}}}}) – Bluetooth communication protocol

\item {} 
\sphinxAtStartPar
\sphinxstyleliteralstrong{\sphinxupquote{bt\_devices}} ({\hyperref[\detokenize{eezz:eezz.table.TTable}]{\sphinxcrossref{\sphinxstyleliteralemphasis{\sphinxupquote{TTable}}}}}) – Table listing bluetooth devices in range

\item {} 
\sphinxAtStartPar
\sphinxstyleliteralstrong{\sphinxupquote{mb\_devices}} ({\hyperref[\detokenize{eezz:eezz.table.TTable}]{\sphinxcrossref{\sphinxstyleliteralemphasis{\sphinxupquote{TTable}}}}}) – Table with paired devices

\end{itemize}

\end{description}\end{quote}
\index{connect() (eezz.session.TSession method)@\spxentry{connect()}\spxextra{eezz.session.TSession method}}

\begin{savenotes}\begin{fulllineitems}
\phantomsection\label{\detokenize{eezz:eezz.session.TSession.connect}}
\pysigstartsignatures
\pysiglinewithargsret{\sphinxbfcode{\sphinxupquote{connect}}}{\sphinxparam{\DUrole{n}{local\_user}\DUrole{p}{:}\DUrole{w}{ }\DUrole{n}{dict}}}{}
\pysigstopsignatures
\sphinxAtStartPar
Connect a Windows user to EEZZ interface. This method is called using html:
\sphinxurl{http://localhost}:<port>/eezzyfree?sid=<user\sphinxhyphen{}sid>,name=<user\sphinxhyphen{}name>
\begin{quote}\begin{description}
\sphinxlineitem{Parameters}
\sphinxAtStartPar
\sphinxstyleliteralstrong{\sphinxupquote{local\_user}} – The user to connect to GUI

\end{description}\end{quote}

\end{fulllineitems}\end{savenotes}

\index{get\_user\_pwd() (eezz.session.TSession method)@\spxentry{get\_user\_pwd()}\spxextra{eezz.session.TSession method}}

\begin{savenotes}\begin{fulllineitems}
\phantomsection\label{\detokenize{eezz:eezz.session.TSession.get_user_pwd}}
\pysigstartsignatures
\pysiglinewithargsret{\sphinxbfcode{\sphinxupquote{get\_user\_pwd}}}{}{{ $\rightarrow$ dict}}
\pysigstopsignatures
\sphinxAtStartPar
Called by external process to unlock workstation
\begin{quote}\begin{description}
\sphinxlineitem{Returns}
\sphinxAtStartPar
The password to unlock the workstation

\end{description}\end{quote}

\end{fulllineitems}\end{savenotes}

\index{handle\_bt\_devices() (eezz.session.TSession method)@\spxentry{handle\_bt\_devices()}\spxextra{eezz.session.TSession method}}

\begin{savenotes}\begin{fulllineitems}
\phantomsection\label{\detokenize{eezz:eezz.session.TSession.handle_bt_devices}}
\pysigstartsignatures
\pysiglinewithargsret{\sphinxbfcode{\sphinxupquote{handle\_bt\_devices}}}{}{}
\pysigstopsignatures
\sphinxAtStartPar
Interact with the bluetooth search {\hyperref[\detokenize{eezz:eezz.blueserv.TBluetooth.find_devices}]{\sphinxcrossref{\sphinxcode{\sphinxupquote{eezz.blueserv.TBluetooth.find\_devices()}}}}}. This
method is called as thread target in :py:meth:eezz.session.TSession.connect` and keeps loop
as long as the connection to desktop user is established

\end{fulllineitems}\end{savenotes}

\index{pair\_device() (eezz.session.TSession method)@\spxentry{pair\_device()}\spxextra{eezz.session.TSession method}}

\begin{savenotes}\begin{fulllineitems}
\phantomsection\label{\detokenize{eezz:eezz.session.TSession.pair_device}}
\pysigstartsignatures
\pysiglinewithargsret{\sphinxbfcode{\sphinxupquote{pair\_device}}}{\sphinxparam{\DUrole{n}{address}\DUrole{p}{:}\DUrole{w}{ }\DUrole{n}{str}}\sphinxparamcomma \sphinxparam{\DUrole{n}{password}\DUrole{p}{:}\DUrole{w}{ }\DUrole{n}{str}}}{{ $\rightarrow$ bool}}
\pysigstopsignatures
\sphinxAtStartPar
Stores the user password on mobile device. The password is encrypted and the key is stored in the
Windows registry. This method is called by the user interface
\sphinxhyphen{} The user has to be connected to eezz, which is automatically done using the TaskBar tool
\sphinxhyphen{} The address has to be selected via user interface
\begin{quote}\begin{description}
\sphinxlineitem{Parameters}\begin{itemize}
\item {} 
\sphinxAtStartPar
\sphinxstyleliteralstrong{\sphinxupquote{address}} – 

\item {} 
\sphinxAtStartPar
\sphinxstyleliteralstrong{\sphinxupquote{password}} – Password will be encrypted and stored on device for unlock workstation

\end{itemize}

\sphinxlineitem{Returns}
\sphinxAtStartPar
EEZZ Confirmation message as dict

\end{description}\end{quote}

\end{fulllineitems}\end{savenotes}

\index{read\_windows\_registry() (eezz.session.TSession method)@\spxentry{read\_windows\_registry()}\spxextra{eezz.session.TSession method}}

\begin{savenotes}\begin{fulllineitems}
\phantomsection\label{\detokenize{eezz:eezz.session.TSession.read_windows_registry}}
\pysigstartsignatures
\pysiglinewithargsret{\sphinxbfcode{\sphinxupquote{read\_windows\_registry}}}{}{}
\pysigstopsignatures
\sphinxAtStartPar
Read user data from windows registry

\end{fulllineitems}\end{savenotes}

\index{register\_user() (eezz.session.TSession method)@\spxentry{register\_user()}\spxextra{eezz.session.TSession method}}

\begin{savenotes}\begin{fulllineitems}
\phantomsection\label{\detokenize{eezz:eezz.session.TSession.register_user}}
\pysigstartsignatures
\pysiglinewithargsret{\sphinxbfcode{\sphinxupquote{register\_user}}}{\sphinxparam{\DUrole{n}{password}\DUrole{p}{:}\DUrole{w}{ }\DUrole{n}{str}}\sphinxparamcomma \sphinxparam{\DUrole{n}{alias}\DUrole{p}{:}\DUrole{w}{ }\DUrole{n}{str}}\sphinxparamcomma \sphinxparam{\DUrole{n}{fname}\DUrole{p}{:}\DUrole{w}{ }\DUrole{n}{str}}\sphinxparamcomma \sphinxparam{\DUrole{n}{lname}\DUrole{p}{:}\DUrole{w}{ }\DUrole{n}{str}}\sphinxparamcomma \sphinxparam{\DUrole{n}{email}\DUrole{p}{:}\DUrole{w}{ }\DUrole{n}{str}}\sphinxparamcomma \sphinxparam{\DUrole{n}{iban}\DUrole{p}{:}\DUrole{w}{ }\DUrole{n}{str}\DUrole{w}{ }\DUrole{o}{=}\DUrole{w}{ }\DUrole{default_value}{''}}}{{ $\rightarrow$ dict}}
\pysigstopsignatures
\sphinxAtStartPar
Register user on EEZZ server. The request is send to the mobile device, which enriches the data and then
forwards it to the eezz server page.
\begin{quote}\begin{description}
\sphinxlineitem{Parameters}\begin{itemize}
\item {} 
\sphinxAtStartPar
\sphinxstyleliteralstrong{\sphinxupquote{alias}} – Display name of the user

\item {} 
\sphinxAtStartPar
\sphinxstyleliteralstrong{\sphinxupquote{fname}} – First name

\item {} 
\sphinxAtStartPar
\sphinxstyleliteralstrong{\sphinxupquote{lname}} – Last name

\item {} 
\sphinxAtStartPar
\sphinxstyleliteralstrong{\sphinxupquote{email}} – E\sphinxhyphen{}Mail address

\item {} 
\sphinxAtStartPar
\sphinxstyleliteralstrong{\sphinxupquote{iban}} – Payment account

\item {} 
\sphinxAtStartPar
\sphinxstyleliteralstrong{\sphinxupquote{password}} – Password for the service. Only the hash value is stored, not the password itself

\end{itemize}

\sphinxlineitem{Returns}
\sphinxAtStartPar
Status message

\end{description}\end{quote}

\end{fulllineitems}\end{savenotes}

\index{send\_bt\_request() (eezz.session.TSession method)@\spxentry{send\_bt\_request()}\spxextra{eezz.session.TSession method}}

\begin{savenotes}\begin{fulllineitems}
\phantomsection\label{\detokenize{eezz:eezz.session.TSession.send_bt_request}}
\pysigstartsignatures
\pysiglinewithargsret{\sphinxbfcode{\sphinxupquote{send\_bt\_request}}}{\sphinxparam{\DUrole{n}{command}\DUrole{p}{:}\DUrole{w}{ }\DUrole{n}{str}}\sphinxparamcomma \sphinxparam{\DUrole{n}{args}\DUrole{p}{:}\DUrole{w}{ }\DUrole{n}{list}}}{{ $\rightarrow$ dict}}
\pysigstopsignatures
\end{fulllineitems}\end{savenotes}


\end{fulllineitems}\end{savenotes}



\subsection{eezz.table module}
\label{\detokenize{eezz:module-eezz.table}}\label{\detokenize{eezz:eezz-table-module}}\index{module@\spxentry{module}!eezz.table@\spxentry{eezz.table}}\index{eezz.table@\spxentry{eezz.table}!module@\spxentry{module}}
\sphinxAtStartPar
This module implements the following classes:
\begin{itemize}
\item {} 
\sphinxAtStartPar
\sphinxstylestrong{TTableCell}:   Defines properties of a table cell

\item {} 
\sphinxAtStartPar
\sphinxstylestrong{TTableRow}:    Defines properties of a table row, containing a list of TTableCells

\item {} 
\sphinxAtStartPar
\sphinxstylestrong{TTable}:       Defines properties of a table, containing a list of TTableRows

\item {} 
\sphinxAtStartPar
\sphinxstylestrong{TTableColumn}: Defines properties of a table column

\end{itemize}

\sphinxAtStartPar
TTable is used for formatted ASCII output of a table structure.
It allows to access the table data for further processing e.g. for HTML output.
It could also be used to access a SQL database table

\sphinxAtStartPar
TTable is a list of TTableRow objects, each of which is a list of TCell objects.
The TColumn holds the column names and is used to organize sort and filter.
A TCell object could hold TTable objects for recursive tree structures.


\begin{savenotes}\begin{fulllineitems}
\phantomsection\label{\detokenize{eezz:eezz.table.TNavigation}}
\pysigstartsignatures
\pysiglinewithargsret{\sphinxbfcode{\sphinxupquote{enum\DUrole{w}{ }}}\sphinxcode{\sphinxupquote{eezz.table.}}\sphinxbfcode{\sphinxupquote{TNavigation}}}{\sphinxparam{\DUrole{n}{value}}}{}
\pysigstopsignatures
\sphinxAtStartPar
Bases: \sphinxcode{\sphinxupquote{Enum}}

\sphinxAtStartPar
Elements to describe navigation events for method {\hyperref[\detokenize{eezz:eezz.table.TTable.navigate}]{\sphinxcrossref{\sphinxcode{\sphinxupquote{eezz.table.TTable.navigate()}}}}}. The navigation is
organized in chunks of rows given by property
{\hyperref[\detokenize{eezz:ttable-parameter-list}]{\sphinxcrossref{\DUrole{std,std-ref}{TTable.visible\_items}}}}:

\sphinxAtStartPar
Valid values are as follows:
\index{ABS (eezz.table.TNavigation attribute)@\spxentry{ABS}\spxextra{eezz.table.TNavigation attribute}}

\begin{savenotes}\begin{fulllineitems}
\phantomsection\label{\detokenize{eezz:eezz.table.TNavigation.ABS}}
\pysigstartsignatures
\pysigline{\sphinxbfcode{\sphinxupquote{ABS}}\sphinxbfcode{\sphinxupquote{\DUrole{w}{ }\DUrole{p}{=}\DUrole{w}{ }<TNavigation.ABS: (0, 'Request an absolute position in the dataset')>}}}
\pysigstopsignatures
\end{fulllineitems}\end{savenotes}

\index{NEXT (eezz.table.TNavigation attribute)@\spxentry{NEXT}\spxextra{eezz.table.TNavigation attribute}}

\begin{savenotes}\begin{fulllineitems}
\phantomsection\label{\detokenize{eezz:eezz.table.TNavigation.NEXT}}
\pysigstartsignatures
\pysigline{\sphinxbfcode{\sphinxupquote{NEXT}}\sphinxbfcode{\sphinxupquote{\DUrole{w}{ }\DUrole{p}{=}\DUrole{w}{ }<TNavigation.NEXT: (1, 'Set the cursor to show the next chunk of rows')>}}}
\pysigstopsignatures
\end{fulllineitems}\end{savenotes}

\index{PREV (eezz.table.TNavigation attribute)@\spxentry{PREV}\spxextra{eezz.table.TNavigation attribute}}

\begin{savenotes}\begin{fulllineitems}
\phantomsection\label{\detokenize{eezz:eezz.table.TNavigation.PREV}}
\pysigstartsignatures
\pysigline{\sphinxbfcode{\sphinxupquote{PREV}}\sphinxbfcode{\sphinxupquote{\DUrole{w}{ }\DUrole{p}{=}\DUrole{w}{ }<TNavigation.PREV: (2, 'Set the cursor to show the previous chunk of rows')>}}}
\pysigstopsignatures
\end{fulllineitems}\end{savenotes}

\index{TOP (eezz.table.TNavigation attribute)@\spxentry{TOP}\spxextra{eezz.table.TNavigation attribute}}

\begin{savenotes}\begin{fulllineitems}
\phantomsection\label{\detokenize{eezz:eezz.table.TNavigation.TOP}}
\pysigstartsignatures
\pysigline{\sphinxbfcode{\sphinxupquote{TOP}}\sphinxbfcode{\sphinxupquote{\DUrole{w}{ }\DUrole{p}{=}\DUrole{w}{ }<TNavigation.TOP: (3, 'Set the cursor to the first row')>}}}
\pysigstopsignatures
\end{fulllineitems}\end{savenotes}

\index{LAST (eezz.table.TNavigation attribute)@\spxentry{LAST}\spxextra{eezz.table.TNavigation attribute}}

\begin{savenotes}\begin{fulllineitems}
\phantomsection\label{\detokenize{eezz:eezz.table.TNavigation.LAST}}
\pysigstartsignatures
\pysigline{\sphinxbfcode{\sphinxupquote{LAST}}\sphinxbfcode{\sphinxupquote{\DUrole{w}{ }\DUrole{p}{=}\DUrole{w}{ }<TNavigation.LAST: (4, 'Set the cursor to show the last chunk of rows')>}}}
\pysigstopsignatures
\end{fulllineitems}\end{savenotes}


\end{fulllineitems}\end{savenotes}



\begin{savenotes}\begin{fulllineitems}
\phantomsection\label{\detokenize{eezz:eezz.table.TSort}}
\pysigstartsignatures
\pysiglinewithargsret{\sphinxbfcode{\sphinxupquote{enum\DUrole{w}{ }}}\sphinxcode{\sphinxupquote{eezz.table.}}\sphinxbfcode{\sphinxupquote{TSort}}}{\sphinxparam{\DUrole{n}{value}}}{}
\pysigstopsignatures
\sphinxAtStartPar
Bases: \sphinxcode{\sphinxupquote{Enum}}

\sphinxAtStartPar
Sorting control enum to define sort on columns

\sphinxAtStartPar
Valid values are as follows:
\index{NONE (eezz.table.TSort attribute)@\spxentry{NONE}\spxextra{eezz.table.TSort attribute}}

\begin{savenotes}\begin{fulllineitems}
\phantomsection\label{\detokenize{eezz:eezz.table.TSort.NONE}}
\pysigstartsignatures
\pysigline{\sphinxbfcode{\sphinxupquote{NONE}}\sphinxbfcode{\sphinxupquote{\DUrole{w}{ }\DUrole{p}{=}\DUrole{w}{ }<TSort.NONE: 0>}}}
\pysigstopsignatures
\end{fulllineitems}\end{savenotes}

\index{ASCENDING (eezz.table.TSort attribute)@\spxentry{ASCENDING}\spxextra{eezz.table.TSort attribute}}

\begin{savenotes}\begin{fulllineitems}
\phantomsection\label{\detokenize{eezz:eezz.table.TSort.ASCENDING}}
\pysigstartsignatures
\pysigline{\sphinxbfcode{\sphinxupquote{ASCENDING}}\sphinxbfcode{\sphinxupquote{\DUrole{w}{ }\DUrole{p}{=}\DUrole{w}{ }<TSort.ASCENDING: 1>}}}
\pysigstopsignatures
\end{fulllineitems}\end{savenotes}

\index{DESCENDING (eezz.table.TSort attribute)@\spxentry{DESCENDING}\spxextra{eezz.table.TSort attribute}}

\begin{savenotes}\begin{fulllineitems}
\phantomsection\label{\detokenize{eezz:eezz.table.TSort.DESCENDING}}
\pysigstartsignatures
\pysigline{\sphinxbfcode{\sphinxupquote{DESCENDING}}\sphinxbfcode{\sphinxupquote{\DUrole{w}{ }\DUrole{p}{=}\DUrole{w}{ }<TSort.DESCENDING: 2>}}}
\pysigstopsignatures
\end{fulllineitems}\end{savenotes}


\end{fulllineitems}\end{savenotes}

\index{TTable (class in eezz.table)@\spxentry{TTable}\spxextra{class in eezz.table}}

\begin{savenotes}\begin{fulllineitems}
\phantomsection\label{\detokenize{eezz:eezz.table.TTable}}
\pysigstartsignatures
\pysiglinewithargsret{\sphinxbfcode{\sphinxupquote{class\DUrole{w}{ }}}\sphinxcode{\sphinxupquote{eezz.table.}}\sphinxbfcode{\sphinxupquote{TTable}}}{\sphinxparam{*, column\_names: \textasciitilde{}typing.List{[}str{]}, column\_names\_map: \textasciitilde{}typing.Dict{[}str, \textasciitilde{}eezz.table.TTableCell{]} | None = None, column\_names\_alias: \textasciitilde{}typing.Dict{[}str, str{]} | None = None, column\_names\_filter: \textasciitilde{}typing.List{[}int{]} | None = None, column\_descr: \textasciitilde{}typing.List{[}\textasciitilde{}eezz.table.TTableColumn{]} = None, table\_index: \textasciitilde{}typing.Dict{[}str, \textasciitilde{}eezz.table.TTableRow{]} = None, title: str = 'Table', attrs: dict = None, visible\_items: int = 20, offset: int = 0, selected\_row: \textasciitilde{}eezz.table.TTableRow = None, header\_row: \textasciitilde{}eezz.table.TTableRow = None, apply\_filter\_column: bool = False, format\_types: dict = None, async\_condition: \textasciitilde{}threading.Condition = <Condition(<unlocked \_thread.RLock object owner=0 count=0>, 0)>, async\_lock: \textasciitilde{}\_thread.allocate\_lock = <unlocked \_thread.lock object>}}{}
\pysigstopsignatures
\sphinxAtStartPar
Bases: \sphinxcode{\sphinxupquote{UserList}}

\sphinxAtStartPar
The table is derived from User\sphinxhyphen{}list to enable sort and list management

\phantomsection\label{\detokenize{eezz:ttable-parameter-list}}\begin{quote}\begin{description}
\sphinxlineitem{Parameters}\begin{itemize}
\item {} 
\sphinxAtStartPar
\sphinxstyleliteralstrong{\sphinxupquote{column\_names}} – List names of the columns

\item {} 
\sphinxAtStartPar
\sphinxstyleliteralstrong{\sphinxupquote{column\_names\_map}} (\sphinxstyleliteralemphasis{\sphinxupquote{list}}\sphinxstyleliteralemphasis{\sphinxupquote{, }}\sphinxstyleliteralemphasis{\sphinxupquote{optional}}) – Map names for output to re\sphinxhyphen{}arrange order

\item {} 
\sphinxAtStartPar
\sphinxstyleliteralstrong{\sphinxupquote{column\_names\_alias}} – Map alias names to column names. This could be used to translate the table header    without changing the select statements.

\item {} 
\sphinxAtStartPar
\sphinxstyleliteralstrong{\sphinxupquote{column\_names\_filter}} – Map columns for output, allows selecting a subset and rearanging, without touching    the internal structure of the table

\item {} 
\sphinxAtStartPar
\sphinxstyleliteralstrong{\sphinxupquote{column\_descr}} – Contains all attributes of a column like type and width

\item {} 
\sphinxAtStartPar
\sphinxstyleliteralstrong{\sphinxupquote{table\_index}} – Managing an index for row\sphinxhyphen{}id

\item {} 
\sphinxAtStartPar
\sphinxstyleliteralstrong{\sphinxupquote{title}} – Table title

\item {} 
\sphinxAtStartPar
\sphinxstyleliteralstrong{\sphinxupquote{attrs}} – User defined attributes for the table

\item {} 
\sphinxAtStartPar
\sphinxstyleliteralstrong{\sphinxupquote{visible\_items}} – Number of visible items: default is 20

\item {} 
\sphinxAtStartPar
\sphinxstyleliteralstrong{\sphinxupquote{offset}} – Cursor position in data set

\item {} 
\sphinxAtStartPar
\sphinxstyleliteralstrong{\sphinxupquote{selected\_row}} – Current selected row

\item {} 
\sphinxAtStartPar
\sphinxstyleliteralstrong{\sphinxupquote{header\_row}} – Header row

\item {} 
\sphinxAtStartPar
\sphinxstyleliteralstrong{\sphinxupquote{apply\_filter\_column}} – Choose between a filtered setup or the original header

\item {} 
\sphinxAtStartPar
\sphinxstyleliteralstrong{\sphinxupquote{format\_types}} (\sphinxstyleliteralemphasis{\sphinxupquote{Dict}}\sphinxstyleliteralemphasis{\sphinxupquote{{[}}}\sphinxstyleliteralemphasis{\sphinxupquote{type: Callable}}\sphinxstyleliteralemphasis{\sphinxupquote{{[}}}\sphinxstyleliteralemphasis{\sphinxupquote{length}}\sphinxstyleliteralemphasis{\sphinxupquote{,}}\sphinxstyleliteralemphasis{\sphinxupquote{value}}\sphinxstyleliteralemphasis{\sphinxupquote{{]}}}\sphinxstyleliteralemphasis{\sphinxupquote{\sphinxhyphen{}>format\sphinxhyphen{}str}}\sphinxstyleliteralemphasis{\sphinxupquote{{]}}}) – Maps a column type to a formatter for ASCII output

\end{itemize}

\end{description}\end{quote}
\subsubsection*{Examples}

\sphinxAtStartPar
This is a possible extension of a format for type iban, breaking the string into chunks of 4:

\begin{sphinxVerbatim}[commandchars=\\\{\}]
\PYG{g+gp}{\PYGZgt{}\PYGZgt{}\PYGZgt{} }\PYG{n}{iban} \PYG{o}{=} \PYG{l+s+s1}{\PYGZsq{}}\PYG{l+s+s1}{de1212341234123412}\PYG{l+s+s1}{\PYGZsq{}}
\PYG{g+gp}{\PYGZgt{}\PYGZgt{}\PYGZgt{} }\PYG{n}{format\PYGZus{}types}\PYG{p}{[}\PYG{l+s+s1}{\PYGZsq{}}\PYG{l+s+s1}{iban}\PYG{l+s+s1}{\PYGZsq{}}\PYG{p}{]} \PYG{o}{=} \PYG{k}{lambda} \PYG{n}{x\PYGZus{}size}\PYG{p}{,} \PYG{n}{x\PYGZus{}val}\PYG{p}{:} \PYG{l+s+s1}{\PYGZsq{}}\PYG{l+s+s1}{ }\PYG{l+s+s1}{\PYGZsq{}}\PYG{o}{.}\PYG{n}{join}\PYG{p}{(}\PYG{p}{[}\PYG{l+s+s1}{\PYGZsq{}}\PYG{l+s+si}{\PYGZob{}\PYGZcb{}}\PYG{l+s+s1}{\PYGZsq{}} \PYG{k}{for} \PYG{n}{x} \PYG{o+ow}{in} \PYG{n+nb}{range}\PYG{p}{(}\PYG{l+m+mi}{6}\PYG{p}{)}\PYG{p}{]}\PYG{p}{)}\PYG{o}{.}\PYG{n}{format}\PYG{p}{(}\PYG{o}{*} \PYG{n}{re}\PYG{o}{.}\PYG{n}{findall}\PYG{p}{(}\PYG{l+s+s1}{\PYGZsq{}}\PYG{l+s+s1}{.}\PYG{l+s+s1}{\PYGZob{}}\PYG{l+s+s1}{1,4\PYGZcb{}}\PYG{l+s+s1}{\PYGZsq{}}\PYG{p}{,} \PYG{n}{iban}\PYG{p}{)}\PYG{p}{\PYGZcb{}}\PYG{p}{)}
\PYG{g+go}{de12 1234 1234 1234 1234 12}
\end{sphinxVerbatim}
\index{append() (eezz.table.TTable method)@\spxentry{append()}\spxextra{eezz.table.TTable method}}

\begin{savenotes}\begin{fulllineitems}
\phantomsection\label{\detokenize{eezz:eezz.table.TTable.append}}
\pysigstartsignatures
\pysiglinewithargsret{\sphinxbfcode{\sphinxupquote{append}}}{\sphinxparam{\DUrole{n}{table\_row}\DUrole{p}{:}\DUrole{w}{ }\DUrole{n}{list}}\sphinxparamcomma \sphinxparam{\DUrole{n}{attrs}\DUrole{p}{:}\DUrole{w}{ }\DUrole{n}{dict}\DUrole{w}{ }\DUrole{o}{=}\DUrole{w}{ }\DUrole{default_value}{None}}\sphinxparamcomma \sphinxparam{\DUrole{n}{row\_type}\DUrole{p}{:}\DUrole{w}{ }\DUrole{n}{str}\DUrole{w}{ }\DUrole{o}{=}\DUrole{w}{ }\DUrole{default_value}{'body'}}\sphinxparamcomma \sphinxparam{\DUrole{n}{row\_id}\DUrole{p}{:}\DUrole{w}{ }\DUrole{n}{str}\DUrole{w}{ }\DUrole{o}{=}\DUrole{w}{ }\DUrole{default_value}{''}}\sphinxparamcomma \sphinxparam{\DUrole{n}{exists\_ok}\DUrole{o}{=}\DUrole{default_value}{True}}}{{ $\rightarrow$ {\hyperref[\detokenize{eezz:eezz.table.TTableRow}]{\sphinxcrossref{TTableRow}}}}}
\pysigstopsignatures
\sphinxAtStartPar
Append a row into the table
This procedure also defines the column type and the width
\begin{quote}\begin{description}
\sphinxlineitem{Parameters}\begin{itemize}
\item {} 
\sphinxAtStartPar
\sphinxstyleliteralstrong{\sphinxupquote{exists\_ok}} – Try to append, but do not throw exception, if key exists

\item {} 
\sphinxAtStartPar
\sphinxstyleliteralstrong{\sphinxupquote{table\_row}} – List of values

\item {} 
\sphinxAtStartPar
\sphinxstyleliteralstrong{\sphinxupquote{attrs}} – Customizable attributes

\item {} 
\sphinxAtStartPar
\sphinxstyleliteralstrong{\sphinxupquote{row\_type}} – Row type used for output filter

\item {} 
\sphinxAtStartPar
\sphinxstyleliteralstrong{\sphinxupquote{row\_id}} – Unique row id

\end{itemize}

\sphinxlineitem{Raises}
\sphinxAtStartPar
\sphinxstyleliteralstrong{\sphinxupquote{TableInsertException}} – Exception if row\sphinxhyphen{}id already exists

\end{description}\end{quote}

\end{fulllineitems}\end{savenotes}

\index{do\_select() (eezz.table.TTable method)@\spxentry{do\_select()}\spxextra{eezz.table.TTable method}}

\begin{savenotes}\begin{fulllineitems}
\phantomsection\label{\detokenize{eezz:eezz.table.TTable.do_select}}
\pysigstartsignatures
\pysiglinewithargsret{\sphinxbfcode{\sphinxupquote{do\_select}}}{\sphinxparam{\DUrole{n}{filters}\DUrole{p}{:}\DUrole{w}{ }\DUrole{n}{dict\DUrole{w}{ }\DUrole{p}{|}\DUrole{w}{ }str}}\sphinxparamcomma \sphinxparam{\DUrole{n}{get\_all}\DUrole{p}{:}\DUrole{w}{ }\DUrole{n}{bool}\DUrole{w}{ }\DUrole{o}{=}\DUrole{w}{ }\DUrole{default_value}{False}}}{{ $\rightarrow$ List\DUrole{p}{{[}}{\hyperref[\detokenize{eezz:eezz.table.TTableRow}]{\sphinxcrossref{TTableRow}}}\DUrole{p}{{]}}}}
\pysigstopsignatures
\sphinxAtStartPar
Select table rows using column values pairs, return at maximum visible\_items.
The value could be any valid regular expression.
\begin{quote}\begin{description}
\sphinxlineitem{Parameters}\begin{itemize}
\item {} 
\sphinxAtStartPar
\sphinxstyleliteralstrong{\sphinxupquote{filters}} (\sphinxstyleliteralemphasis{\sphinxupquote{dict}}\sphinxstyleliteralemphasis{\sphinxupquote{{[}}}\sphinxstyleliteralemphasis{\sphinxupquote{column\_name}}\sphinxstyleliteralemphasis{\sphinxupquote{, }}\sphinxstyleliteralemphasis{\sphinxupquote{value}}\sphinxstyleliteralemphasis{\sphinxupquote{{]}}}) – dictionary with column\sphinxhyphen{}name \sphinxhyphen{} column\sphinxhyphen{}value pairs

\item {} 
\sphinxAtStartPar
\sphinxstyleliteralstrong{\sphinxupquote{get\_all}} – If True select more than visible\_items

\end{itemize}

\sphinxlineitem{Returns}
\sphinxAtStartPar
List of selected rows

\sphinxlineitem{Return type}
\sphinxAtStartPar
List{[}{\hyperref[\detokenize{eezz:eezz.table.TTableRow}]{\sphinxcrossref{TTableRow}}}{]}

\end{description}\end{quote}
\begin{description}
\sphinxlineitem{Example}
\begin{sphinxVerbatim}[commandchars=\\\{\}]
\PYG{g+gp}{\PYGZgt{}\PYGZgt{}\PYGZgt{} }\PYG{n}{users} \PYG{o}{=} \PYG{n}{TTable}\PYG{p}{(}\PYG{n}{column\PYGZus{}names}\PYG{o}{=}\PYG{p}{[}\PYG{l+s+s1}{\PYGZsq{}}\PYG{l+s+s1}{CUser}\PYG{l+s+s1}{\PYGZsq{}}\PYG{p}{]}\PYG{p}{,} \PYG{n}{title}\PYG{o}{=}\PYG{l+s+s1}{\PYGZsq{}}\PYG{l+s+s1}{users}\PYG{l+s+s1}{\PYGZsq{}}\PYG{p}{)}
\PYG{g+gp}{\PYGZgt{}\PYGZgt{}\PYGZgt{} }\PYG{n}{users}\PYG{o}{.}\PYG{n}{append}\PYG{p}{(}\PYG{p}{[}\PYG{l+s+s1}{\PYGZsq{}}\PYG{l+s+s1}{paul}\PYG{l+s+s1}{\PYGZsq{}}\PYG{p}{]}\PYG{p}{)}
\PYG{g+gp}{\PYGZgt{}\PYGZgt{}\PYGZgt{} }\PYG{n}{rows}  \PYG{o}{=} \PYG{n}{users}\PYG{o}{.}\PYG{n}{do\PYGZus{}select}\PYG{p}{(}\PYG{p}{\PYGZob{}}\PYG{l+s+s1}{\PYGZsq{}}\PYG{l+s+s1}{CUser}\PYG{l+s+s1}{\PYGZsq{}}\PYG{p}{:} \PYG{l+s+s1}{\PYGZsq{}}\PYG{l+s+s1}{p*}\PYG{l+s+s1}{\PYGZsq{}}\PYG{p}{\PYGZcb{}}\PYG{p}{)}
\PYG{g+gp}{\PYGZgt{}\PYGZgt{}\PYGZgt{} }\PYG{n}{rows}\PYG{p}{[}\PYG{l+m+mi}{0}\PYG{p}{]}\PYG{p}{[}\PYG{l+s+s1}{\PYGZsq{}}\PYG{l+s+s1}{CUser}\PYG{l+s+s1}{\PYGZsq{}}\PYG{p}{]}
\PYG{g+go}{\PYGZsq{}paul\PYGZsq{}}
\end{sphinxVerbatim}

\end{description}

\end{fulllineitems}\end{savenotes}

\index{do\_sort() (eezz.table.TTable method)@\spxentry{do\_sort()}\spxextra{eezz.table.TTable method}}

\begin{savenotes}\begin{fulllineitems}
\phantomsection\label{\detokenize{eezz:eezz.table.TTable.do_sort}}
\pysigstartsignatures
\pysiglinewithargsret{\sphinxbfcode{\sphinxupquote{do\_sort}}}{\sphinxparam{\DUrole{n}{column}\DUrole{p}{:}\DUrole{w}{ }\DUrole{n}{int\DUrole{w}{ }\DUrole{p}{|}\DUrole{w}{ }str}}\sphinxparamcomma \sphinxparam{\DUrole{n}{reverse}\DUrole{p}{:}\DUrole{w}{ }\DUrole{n}{bool}\DUrole{w}{ }\DUrole{o}{=}\DUrole{w}{ }\DUrole{default_value}{False}}}{{ $\rightarrow$ None}}
\pysigstopsignatures
\sphinxAtStartPar
Toggle sort on a given column index
\begin{quote}\begin{description}
\sphinxlineitem{Parameters}\begin{itemize}
\item {} 
\sphinxAtStartPar
\sphinxstyleliteralstrong{\sphinxupquote{column}} – The column to sort for

\item {} 
\sphinxAtStartPar
\sphinxstyleliteralstrong{\sphinxupquote{reverse}} – Sort direction reversed

\end{itemize}

\end{description}\end{quote}

\end{fulllineitems}\end{savenotes}

\index{filter\_clear() (eezz.table.TTable method)@\spxentry{filter\_clear()}\spxextra{eezz.table.TTable method}}

\begin{savenotes}\begin{fulllineitems}
\phantomsection\label{\detokenize{eezz:eezz.table.TTable.filter_clear}}
\pysigstartsignatures
\pysiglinewithargsret{\sphinxbfcode{\sphinxupquote{filter\_clear}}}{}{}
\pysigstopsignatures
\sphinxAtStartPar
Clear the filters and return to original output

\end{fulllineitems}\end{savenotes}

\index{filter\_columns() (eezz.table.TTable method)@\spxentry{filter\_columns()}\spxextra{eezz.table.TTable method}}

\begin{savenotes}\begin{fulllineitems}
\phantomsection\label{\detokenize{eezz:eezz.table.TTable.filter_columns}}
\pysigstartsignatures
\pysiglinewithargsret{\sphinxbfcode{\sphinxupquote{filter\_columns}}}{\sphinxparam{\DUrole{n}{column\_names}\DUrole{p}{:}\DUrole{w}{ }\DUrole{n}{Dict\DUrole{p}{{[}}str\DUrole{p}{,}\DUrole{w}{ }str\DUrole{p}{{]}}}}}{{ $\rightarrow$ None}}
\pysigstopsignatures
\sphinxAtStartPar
First tuple value is the column name at any position, second tuple value is the new column display name
The filter is used to generate customized output. This function could also be used to reduce the number of
visible columns
\begin{quote}\begin{description}
\sphinxlineitem{Parameters}
\sphinxAtStartPar
\sphinxstyleliteralstrong{\sphinxupquote{column\_names}} (\sphinxstyleliteralemphasis{\sphinxupquote{Dict}}\sphinxstyleliteralemphasis{\sphinxupquote{{[}}}\sphinxstyleliteralemphasis{\sphinxupquote{column\_name: alias\_name}}\sphinxstyleliteralemphasis{\sphinxupquote{{]}}}) – Map new names to a column, e.g. after translation

\end{description}\end{quote}

\end{fulllineitems}\end{savenotes}

\index{get\_child() (eezz.table.TTable method)@\spxentry{get\_child()}\spxextra{eezz.table.TTable method}}

\begin{savenotes}\begin{fulllineitems}
\phantomsection\label{\detokenize{eezz:eezz.table.TTable.get_child}}
\pysigstartsignatures
\pysiglinewithargsret{\sphinxbfcode{\sphinxupquote{get\_child}}}{}{{ $\rightarrow$ {\hyperref[\detokenize{eezz:eezz.table.TTableRow}]{\sphinxcrossref{TTableRow}}}\DUrole{w}{ }\DUrole{p}{|}\DUrole{w}{ }None}}
\pysigstopsignatures
\sphinxAtStartPar
Returns the child table, if exists, else None
\begin{quote}\begin{description}
\sphinxlineitem{Returns}
\sphinxAtStartPar
The child if exists

\sphinxlineitem{Return type}
\sphinxAtStartPar
{\hyperref[\detokenize{eezz:eezz.table.TTableRow}]{\sphinxcrossref{TTableRow}}} | None

\end{description}\end{quote}

\end{fulllineitems}\end{savenotes}

\index{get\_header\_row() (eezz.table.TTable method)@\spxentry{get\_header\_row()}\spxextra{eezz.table.TTable method}}

\begin{savenotes}\begin{fulllineitems}
\phantomsection\label{\detokenize{eezz:eezz.table.TTable.get_header_row}}
\pysigstartsignatures
\pysiglinewithargsret{\sphinxbfcode{\sphinxupquote{get\_header\_row}}}{}{{ $\rightarrow$ {\hyperref[\detokenize{eezz:eezz.table.TTableRow}]{\sphinxcrossref{TTableRow}}}}}
\pysigstopsignatures
\sphinxAtStartPar
Returns the header row.
\begin{quote}\begin{description}
\sphinxlineitem{Returns}
\sphinxAtStartPar
The header of the table

\sphinxlineitem{Return type}
\sphinxAtStartPar
{\hyperref[\detokenize{eezz:eezz.table.TTableRow}]{\sphinxcrossref{TTableRow}}}

\end{description}\end{quote}

\end{fulllineitems}\end{savenotes}

\index{get\_visible\_rows() (eezz.table.TTable method)@\spxentry{get\_visible\_rows()}\spxextra{eezz.table.TTable method}}

\begin{savenotes}\begin{fulllineitems}
\phantomsection\label{\detokenize{eezz:eezz.table.TTable.get_visible_rows}}
\pysigstartsignatures
\pysiglinewithargsret{\sphinxbfcode{\sphinxupquote{get\_visible\_rows}}}{\sphinxparam{\DUrole{n}{get\_all}\DUrole{p}{:}\DUrole{w}{ }\DUrole{n}{bool}\DUrole{w}{ }\DUrole{o}{=}\DUrole{w}{ }\DUrole{default_value}{False}}}{{ $\rightarrow$ List\DUrole{p}{{[}}{\hyperref[\detokenize{eezz:eezz.table.TTableRow}]{\sphinxcrossref{TTableRow}}}\DUrole{p}{{]}}}}
\pysigstopsignatures
\sphinxAtStartPar
Return the visible rows at the current cursor
\begin{quote}\begin{description}
\sphinxlineitem{Parameters}
\sphinxAtStartPar
\sphinxstyleliteralstrong{\sphinxupquote{get\_all}} – A bool value to overwrite the visible\_items for the current call

\sphinxlineitem{Returns}
\sphinxAtStartPar
A list of visible row items

\end{description}\end{quote}

\end{fulllineitems}\end{savenotes}

\index{navigate() (eezz.table.TTable method)@\spxentry{navigate()}\spxextra{eezz.table.TTable method}}

\begin{savenotes}\begin{fulllineitems}
\phantomsection\label{\detokenize{eezz:eezz.table.TTable.navigate}}
\pysigstartsignatures
\pysiglinewithargsret{\sphinxbfcode{\sphinxupquote{navigate}}}{\sphinxparam{\DUrole{n}{where\_togo}\DUrole{p}{:}\DUrole{w}{ }\DUrole{n}{{\hyperref[\detokenize{eezz:eezz.table.TNavigation}]{\sphinxcrossref{TNavigation}}}}\DUrole{w}{ }\DUrole{o}{=}\DUrole{w}{ }\DUrole{default_value}{TNavigation.NEXT}}\sphinxparamcomma \sphinxparam{\DUrole{n}{position}\DUrole{p}{:}\DUrole{w}{ }\DUrole{n}{int}\DUrole{w}{ }\DUrole{o}{=}\DUrole{w}{ }\DUrole{default_value}{0}}}{{ $\rightarrow$ None}}
\pysigstopsignatures
\sphinxAtStartPar
Navigate in block mode
\begin{quote}\begin{description}
\sphinxlineitem{Parameters}\begin{itemize}
\item {} 
\sphinxAtStartPar
\sphinxstyleliteralstrong{\sphinxupquote{where\_togo}} ({\hyperref[\detokenize{eezz:eezz.table.TNavigation}]{\sphinxcrossref{\sphinxstyleliteralemphasis{\sphinxupquote{TNavigation}}}}}) – Navigation direction

\item {} 
\sphinxAtStartPar
\sphinxstyleliteralstrong{\sphinxupquote{position}} – Position for absolute navigation, ignored in any other case

\end{itemize}

\end{description}\end{quote}

\end{fulllineitems}\end{savenotes}

\index{print() (eezz.table.TTable method)@\spxentry{print()}\spxextra{eezz.table.TTable method}}

\begin{savenotes}\begin{fulllineitems}
\phantomsection\label{\detokenize{eezz:eezz.table.TTable.print}}
\pysigstartsignatures
\pysiglinewithargsret{\sphinxbfcode{\sphinxupquote{print}}}{}{{ $\rightarrow$ None}}
\pysigstopsignatures
\sphinxAtStartPar
Print ASCII formatted table

\end{fulllineitems}\end{savenotes}


\end{fulllineitems}\end{savenotes}

\index{TTableCell (class in eezz.table)@\spxentry{TTableCell}\spxextra{class in eezz.table}}

\begin{savenotes}\begin{fulllineitems}
\phantomsection\label{\detokenize{eezz:eezz.table.TTableCell}}
\pysigstartsignatures
\pysiglinewithargsret{\sphinxbfcode{\sphinxupquote{class\DUrole{w}{ }}}\sphinxcode{\sphinxupquote{eezz.table.}}\sphinxbfcode{\sphinxupquote{TTableCell}}}{\sphinxparam{\DUrole{o}{*}}\sphinxparamcomma \sphinxparam{\DUrole{n}{name}\DUrole{p}{:}\DUrole{w}{ }\DUrole{n}{str}\DUrole{w}{ }\DUrole{o}{=}\DUrole{w}{ }\DUrole{default_value}{None}}\sphinxparamcomma \sphinxparam{\DUrole{n}{width}\DUrole{p}{:}\DUrole{w}{ }\DUrole{n}{int}\DUrole{w}{ }\DUrole{o}{=}\DUrole{w}{ }\DUrole{default_value}{10}}\sphinxparamcomma \sphinxparam{\DUrole{n}{value}\DUrole{p}{:}\DUrole{w}{ }\DUrole{n}{Any}\DUrole{w}{ }\DUrole{o}{=}\DUrole{w}{ }\DUrole{default_value}{None}}\sphinxparamcomma \sphinxparam{\DUrole{n}{index}\DUrole{p}{:}\DUrole{w}{ }\DUrole{n}{int}\DUrole{w}{ }\DUrole{o}{=}\DUrole{w}{ }\DUrole{default_value}{0}}\sphinxparamcomma \sphinxparam{\DUrole{n}{type}\DUrole{p}{:}\DUrole{w}{ }\DUrole{n}{str}\DUrole{w}{ }\DUrole{o}{=}\DUrole{w}{ }\DUrole{default_value}{'str'}}\sphinxparamcomma \sphinxparam{\DUrole{n}{attrs}\DUrole{p}{:}\DUrole{w}{ }\DUrole{n}{dict}\DUrole{w}{ }\DUrole{o}{=}\DUrole{w}{ }\DUrole{default_value}{None}}}{}
\pysigstopsignatures
\sphinxAtStartPar
Bases: \sphinxcode{\sphinxupquote{object}}

\sphinxAtStartPar
The cell is the smallest unit of a table
\index{attrs (eezz.table.TTableCell attribute)@\spxentry{attrs}\spxextra{eezz.table.TTableCell attribute}}

\begin{savenotes}\begin{fulllineitems}
\phantomsection\label{\detokenize{eezz:eezz.table.TTableCell.attrs}}
\pysigstartsignatures
\pysigline{\sphinxbfcode{\sphinxupquote{attrs}}\sphinxbfcode{\sphinxupquote{\DUrole{p}{:}\DUrole{w}{ }dict}}\sphinxbfcode{\sphinxupquote{\DUrole{w}{ }\DUrole{p}{=}\DUrole{w}{ }None}}}
\pysigstopsignatures
\sphinxAtStartPar
Meta data to describe the data, the in\sphinxhyphen{} and output format

\end{fulllineitems}\end{savenotes}

\index{index (eezz.table.TTableCell attribute)@\spxentry{index}\spxextra{eezz.table.TTableCell attribute}}

\begin{savenotes}\begin{fulllineitems}
\phantomsection\label{\detokenize{eezz:eezz.table.TTableCell.index}}
\pysigstartsignatures
\pysigline{\sphinxbfcode{\sphinxupquote{index}}\sphinxbfcode{\sphinxupquote{\DUrole{p}{:}\DUrole{w}{ }int}}\sphinxbfcode{\sphinxupquote{\DUrole{w}{ }\DUrole{p}{=}\DUrole{w}{ }0}}}
\pysigstopsignatures
\sphinxAtStartPar
Unique index of the column

\end{fulllineitems}\end{savenotes}

\index{name (eezz.table.TTableCell attribute)@\spxentry{name}\spxextra{eezz.table.TTableCell attribute}}

\begin{savenotes}\begin{fulllineitems}
\phantomsection\label{\detokenize{eezz:eezz.table.TTableCell.name}}
\pysigstartsignatures
\pysigline{\sphinxbfcode{\sphinxupquote{name}}\sphinxbfcode{\sphinxupquote{\DUrole{p}{:}\DUrole{w}{ }str}}\sphinxbfcode{\sphinxupquote{\DUrole{w}{ }\DUrole{p}{=}\DUrole{w}{ }None}}}
\pysigstopsignatures
\sphinxAtStartPar
Name of the column

\end{fulllineitems}\end{savenotes}

\index{type (eezz.table.TTableCell attribute)@\spxentry{type}\spxextra{eezz.table.TTableCell attribute}}

\begin{savenotes}\begin{fulllineitems}
\phantomsection\label{\detokenize{eezz:eezz.table.TTableCell.type}}
\pysigstartsignatures
\pysigline{\sphinxbfcode{\sphinxupquote{type}}\sphinxbfcode{\sphinxupquote{\DUrole{p}{:}\DUrole{w}{ }str}}\sphinxbfcode{\sphinxupquote{\DUrole{w}{ }\DUrole{p}{=}\DUrole{w}{ }'str'}}}
\pysigstopsignatures
\sphinxAtStartPar
Value type to specify the output format. Could be user defined

\end{fulllineitems}\end{savenotes}

\index{value (eezz.table.TTableCell attribute)@\spxentry{value}\spxextra{eezz.table.TTableCell attribute}}

\begin{savenotes}\begin{fulllineitems}
\phantomsection\label{\detokenize{eezz:eezz.table.TTableCell.value}}
\pysigstartsignatures
\pysigline{\sphinxbfcode{\sphinxupquote{value}}\sphinxbfcode{\sphinxupquote{\DUrole{p}{:}\DUrole{w}{ }Any}}\sphinxbfcode{\sphinxupquote{\DUrole{w}{ }\DUrole{p}{=}\DUrole{w}{ }None}}}
\pysigstopsignatures
\sphinxAtStartPar
Display value of the cell

\end{fulllineitems}\end{savenotes}

\index{width (eezz.table.TTableCell attribute)@\spxentry{width}\spxextra{eezz.table.TTableCell attribute}}

\begin{savenotes}\begin{fulllineitems}
\phantomsection\label{\detokenize{eezz:eezz.table.TTableCell.width}}
\pysigstartsignatures
\pysigline{\sphinxbfcode{\sphinxupquote{width}}\sphinxbfcode{\sphinxupquote{\DUrole{p}{:}\DUrole{w}{ }int}}\sphinxbfcode{\sphinxupquote{\DUrole{w}{ }\DUrole{p}{=}\DUrole{w}{ }10}}}
\pysigstopsignatures
\sphinxAtStartPar
Width of the cell content

\end{fulllineitems}\end{savenotes}


\end{fulllineitems}\end{savenotes}

\index{TTableColumn (class in eezz.table)@\spxentry{TTableColumn}\spxextra{class in eezz.table}}

\begin{savenotes}\begin{fulllineitems}
\phantomsection\label{\detokenize{eezz:eezz.table.TTableColumn}}
\pysigstartsignatures
\pysiglinewithargsret{\sphinxbfcode{\sphinxupquote{class\DUrole{w}{ }}}\sphinxcode{\sphinxupquote{eezz.table.}}\sphinxbfcode{\sphinxupquote{TTableColumn}}}{\sphinxparam{\DUrole{o}{*}}\sphinxparamcomma \sphinxparam{\DUrole{n}{index}\DUrole{p}{:}\DUrole{w}{ }\DUrole{n}{int}}\sphinxparamcomma \sphinxparam{\DUrole{n}{header}\DUrole{p}{:}\DUrole{w}{ }\DUrole{n}{str}}\sphinxparamcomma \sphinxparam{\DUrole{n}{width}\DUrole{p}{:}\DUrole{w}{ }\DUrole{n}{int}\DUrole{w}{ }\DUrole{o}{=}\DUrole{w}{ }\DUrole{default_value}{10}}\sphinxparamcomma \sphinxparam{\DUrole{n}{filter}\DUrole{p}{:}\DUrole{w}{ }\DUrole{n}{str}\DUrole{w}{ }\DUrole{o}{=}\DUrole{w}{ }\DUrole{default_value}{''}}\sphinxparamcomma \sphinxparam{\DUrole{n}{sort}\DUrole{p}{:}\DUrole{w}{ }\DUrole{n}{bool}\DUrole{w}{ }\DUrole{o}{=}\DUrole{w}{ }\DUrole{default_value}{True}}\sphinxparamcomma \sphinxparam{\DUrole{n}{type}\DUrole{p}{:}\DUrole{w}{ }\DUrole{n}{str}\DUrole{w}{ }\DUrole{o}{=}\DUrole{w}{ }\DUrole{default_value}{''}}\sphinxparamcomma \sphinxparam{\DUrole{n}{attrs}\DUrole{p}{:}\DUrole{w}{ }\DUrole{n}{dict}\DUrole{w}{ }\DUrole{o}{=}\DUrole{w}{ }\DUrole{default_value}{None}}}{}
\pysigstopsignatures
\sphinxAtStartPar
Bases: \sphinxcode{\sphinxupquote{object}}

\sphinxAtStartPar
Summarize the cell properties in a column, which includes sorting and formatting
\begin{quote}\begin{description}
\sphinxlineitem{Parameters}\begin{itemize}
\item {} 
\sphinxAtStartPar
\sphinxstyleliteralstrong{\sphinxupquote{index}} – Stable address the column, even if filtered or translated

\item {} 
\sphinxAtStartPar
\sphinxstyleliteralstrong{\sphinxupquote{header}} – Name of the column

\item {} 
\sphinxAtStartPar
\sphinxstyleliteralstrong{\sphinxupquote{width}} – Width to fit the largest element in the column

\item {} 
\sphinxAtStartPar
\sphinxstyleliteralstrong{\sphinxupquote{filter}} – Visible name for output

\item {} 
\sphinxAtStartPar
\sphinxstyleliteralstrong{\sphinxupquote{sort}} – Sort direction

\item {} 
\sphinxAtStartPar
\sphinxstyleliteralstrong{\sphinxupquote{type}} – Customizable type

\item {} 
\sphinxAtStartPar
\sphinxstyleliteralstrong{\sphinxupquote{attrs}} – Customizable attributes

\end{itemize}

\end{description}\end{quote}

\end{fulllineitems}\end{savenotes}

\index{TTableInsertException@\spxentry{TTableInsertException}}

\begin{savenotes}\begin{fulllineitems}
\phantomsection\label{\detokenize{eezz:eezz.table.TTableInsertException}}
\pysigstartsignatures
\pysiglinewithargsret{\sphinxbfcode{\sphinxupquote{exception\DUrole{w}{ }}}\sphinxcode{\sphinxupquote{eezz.table.}}\sphinxbfcode{\sphinxupquote{TTableInsertException}}}{\sphinxparam{\DUrole{n}{message}\DUrole{p}{:}\DUrole{w}{ }\DUrole{n}{str}\DUrole{w}{ }\DUrole{o}{=}\DUrole{w}{ }\DUrole{default_value}{'entry already exists, row\sphinxhyphen{}id has to be unique'}}}{}
\pysigstopsignatures
\sphinxAtStartPar
Bases: \sphinxcode{\sphinxupquote{Exception}}

\sphinxAtStartPar
The table exception: trying to insert a double row\sphinxhyphen{}id

\end{fulllineitems}\end{savenotes}

\index{TTableRow (class in eezz.table)@\spxentry{TTableRow}\spxextra{class in eezz.table}}

\begin{savenotes}\begin{fulllineitems}
\phantomsection\label{\detokenize{eezz:eezz.table.TTableRow}}
\pysigstartsignatures
\pysiglinewithargsret{\sphinxbfcode{\sphinxupquote{class\DUrole{w}{ }}}\sphinxcode{\sphinxupquote{eezz.table.}}\sphinxbfcode{\sphinxupquote{TTableRow}}}{\sphinxparam{\DUrole{o}{*}}\sphinxparamcomma \sphinxparam{\DUrole{n}{cells}\DUrole{p}{:}\DUrole{w}{ }\DUrole{n}{List\DUrole{p}{{[}}{\hyperref[\detokenize{eezz:eezz.table.TTableCell}]{\sphinxcrossref{TTableCell}}}\DUrole{p}{{]}}\DUrole{w}{ }\DUrole{p}{|}\DUrole{w}{ }List\DUrole{p}{{[}}str\DUrole{p}{{]}}}}\sphinxparamcomma \sphinxparam{\DUrole{n}{cells\_filter}\DUrole{p}{:}\DUrole{w}{ }\DUrole{n}{List\DUrole{p}{{[}}{\hyperref[\detokenize{eezz:eezz.table.TTableCell}]{\sphinxcrossref{TTableCell}}}\DUrole{p}{{]}}\DUrole{w}{ }\DUrole{p}{|}\DUrole{w}{ }None}\DUrole{w}{ }\DUrole{o}{=}\DUrole{w}{ }\DUrole{default_value}{None}}\sphinxparamcomma \sphinxparam{\DUrole{n}{column\_descr}\DUrole{p}{:}\DUrole{w}{ }\DUrole{n}{List\DUrole{p}{{[}}str\DUrole{p}{{]}}}\DUrole{w}{ }\DUrole{o}{=}\DUrole{w}{ }\DUrole{default_value}{None}}\sphinxparamcomma \sphinxparam{\DUrole{n}{index}\DUrole{p}{:}\DUrole{w}{ }\DUrole{n}{int}\DUrole{w}{ }\DUrole{o}{=}\DUrole{w}{ }\DUrole{default_value}{None}}\sphinxparamcomma \sphinxparam{\DUrole{n}{row\_id}\DUrole{p}{:}\DUrole{w}{ }\DUrole{n}{str}\DUrole{w}{ }\DUrole{o}{=}\DUrole{w}{ }\DUrole{default_value}{None}}\sphinxparamcomma \sphinxparam{\DUrole{n}{child}\DUrole{p}{:}\DUrole{w}{ }\DUrole{n}{{\hyperref[\detokenize{eezz:eezz.table.TTable}]{\sphinxcrossref{TTable}}}}\DUrole{w}{ }\DUrole{o}{=}\DUrole{w}{ }\DUrole{default_value}{None}}\sphinxparamcomma \sphinxparam{\DUrole{n}{type}\DUrole{p}{:}\DUrole{w}{ }\DUrole{n}{str}\DUrole{w}{ }\DUrole{o}{=}\DUrole{w}{ }\DUrole{default_value}{'body'}}\sphinxparamcomma \sphinxparam{\DUrole{n}{attrs}\DUrole{p}{:}\DUrole{w}{ }\DUrole{n}{dict}\DUrole{w}{ }\DUrole{o}{=}\DUrole{w}{ }\DUrole{default_value}{None}}}{}
\pysigstopsignatures
\sphinxAtStartPar
Bases: \sphinxcode{\sphinxupquote{object}}

\sphinxAtStartPar
This structure is created for each row in a table. It allows also to specify a sub\sphinxhyphen{}structure table
\begin{quote}\begin{description}
\sphinxlineitem{Parameters}\begin{itemize}
\item {} 
\sphinxAtStartPar
\sphinxstyleliteralstrong{\sphinxupquote{cells}} (\sphinxstyleliteralemphasis{\sphinxupquote{List}}\sphinxstyleliteralemphasis{\sphinxupquote{{[}}}{\hyperref[\detokenize{eezz:eezz.table.TTableCell}]{\sphinxcrossref{\sphinxstyleliteralemphasis{\sphinxupquote{TTableCell}}}}}\sphinxstyleliteralemphasis{\sphinxupquote{{]} }}\sphinxstyleliteralemphasis{\sphinxupquote{| }}\sphinxstyleliteralemphasis{\sphinxupquote{List}}\sphinxstyleliteralemphasis{\sphinxupquote{{[}}}\sphinxstyleliteralemphasis{\sphinxupquote{str}}\sphinxstyleliteralemphasis{\sphinxupquote{{]}}}) – A list of strings are converted to a list of TTableCells, the class which holds the cell attributes.

\item {} 
\sphinxAtStartPar
\sphinxstyleliteralstrong{\sphinxupquote{cells\_filter}} – A list of cells with filtered attributes. Used for example for translation or re\sphinxhyphen{}ordering.

\item {} 
\sphinxAtStartPar
\sphinxstyleliteralstrong{\sphinxupquote{column\_descr}} – The column descriptor holds the name of the column

\item {} 
\sphinxAtStartPar
\sphinxstyleliteralstrong{\sphinxupquote{index}} – Unique address for the column

\item {} 
\sphinxAtStartPar
\sphinxstyleliteralstrong{\sphinxupquote{row\_id}} – Unique row id for the entire table

\item {} 
\sphinxAtStartPar
\sphinxstyleliteralstrong{\sphinxupquote{child}} – The row could handle recursive data structures

\item {} 
\sphinxAtStartPar
\sphinxstyleliteralstrong{\sphinxupquote{type}} – Customizable type used for triggering template output

\item {} 
\sphinxAtStartPar
\sphinxstyleliteralstrong{\sphinxupquote{attrs}} – Customizable row attributes

\end{itemize}

\end{description}\end{quote}
\index{get\_values\_list() (eezz.table.TTableRow method)@\spxentry{get\_values\_list()}\spxextra{eezz.table.TTableRow method}}

\begin{savenotes}\begin{fulllineitems}
\phantomsection\label{\detokenize{eezz:eezz.table.TTableRow.get_values_list}}
\pysigstartsignatures
\pysiglinewithargsret{\sphinxbfcode{\sphinxupquote{get\_values\_list}}}{}{{ $\rightarrow$ list}}
\pysigstopsignatures
\sphinxAtStartPar
Get row values as a list
\begin{quote}\begin{description}
\sphinxlineitem{Returns}
\sphinxAtStartPar
value of each cell of the row in a list

\end{description}\end{quote}

\end{fulllineitems}\end{savenotes}


\end{fulllineitems}\end{savenotes}



\subsection{eezz.websocket module}
\label{\detokenize{eezz:module-eezz.websocket}}\label{\detokenize{eezz:eezz-websocket-module}}\index{module@\spxentry{module}!eezz.websocket@\spxentry{eezz.websocket}}\index{eezz.websocket@\spxentry{eezz.websocket}!module@\spxentry{module}}\begin{quote}

\sphinxAtStartPar
EezzServer:
High speed application development and
high speed execution based on HTML5

\sphinxAtStartPar
Copyright (C) 2015  Albert Zedlitz

\sphinxAtStartPar
This program is free software: you can redistribute it and/or modify
it under the terms of the GNU General Public License as published by
the Free Software Foundation, either version 3 of the License, or
(at your option) any later version.

\sphinxAtStartPar
This program is distributed in the hope that it will be useful,
but WITHOUT ANY WARRANTY; without even the implied warranty of
MERCHANTABILITY or FITNESS FOR A PARTICULAR PURPOSE.  See the
GNU General Public License for more details.

\sphinxAtStartPar
You should have received a copy of the GNU General Public License
along with this program.  If not, see <\sphinxurl{http://www.gnu.org/licenses/}>.
\end{quote}
\begin{description}
\sphinxlineitem{Description:}
\sphinxAtStartPar
Implements websocket protocol according to rfc 6455
\sphinxurl{https://tools.ietf.org/html/rfc6455}

\end{description}
\index{TAsyncHandler (class in eezz.websocket)@\spxentry{TAsyncHandler}\spxextra{class in eezz.websocket}}

\begin{savenotes}\begin{fulllineitems}
\phantomsection\label{\detokenize{eezz:eezz.websocket.TAsyncHandler}}
\pysigstartsignatures
\pysiglinewithargsret{\sphinxbfcode{\sphinxupquote{class\DUrole{w}{ }}}\sphinxcode{\sphinxupquote{eezz.websocket.}}\sphinxbfcode{\sphinxupquote{TAsyncHandler}}}{\sphinxparam{\DUrole{n}{method}\DUrole{p}{:}\DUrole{w}{ }\DUrole{n}{Callable}}\sphinxparamcomma \sphinxparam{\DUrole{n}{args}\DUrole{p}{:}\DUrole{w}{ }\DUrole{n}{dict}}\sphinxparamcomma \sphinxparam{\DUrole{n}{socket\_server}\DUrole{p}{:}\DUrole{w}{ }\DUrole{n}{{\hyperref[\detokenize{eezz:eezz.websocket.TWebSocketClient}]{\sphinxcrossref{TWebSocketClient}}}}}\sphinxparamcomma \sphinxparam{\DUrole{n}{request}\DUrole{p}{:}\DUrole{w}{ }\DUrole{n}{dict}}\sphinxparamcomma \sphinxparam{\DUrole{n}{description}\DUrole{p}{:}\DUrole{w}{ }\DUrole{n}{str}}}{}
\pysigstopsignatures
\sphinxAtStartPar
Bases: \sphinxcode{\sphinxupquote{Thread}}

\sphinxAtStartPar
Execute method in background task
\index{run() (eezz.websocket.TAsyncHandler method)@\spxentry{run()}\spxextra{eezz.websocket.TAsyncHandler method}}

\begin{savenotes}\begin{fulllineitems}
\phantomsection\label{\detokenize{eezz:eezz.websocket.TAsyncHandler.run}}
\pysigstartsignatures
\pysiglinewithargsret{\sphinxbfcode{\sphinxupquote{run}}}{}{}
\pysigstopsignatures
\sphinxAtStartPar
Method representing the thread’s activity.

\sphinxAtStartPar
You may override this method in a subclass. The standard run() method
invokes the callable object passed to the object’s constructor as the
target argument, if any, with sequential and keyword arguments taken
from the args and kwargs arguments, respectively.

\end{fulllineitems}\end{savenotes}


\end{fulllineitems}\end{savenotes}

\index{TWebSocket (class in eezz.websocket)@\spxentry{TWebSocket}\spxextra{class in eezz.websocket}}

\begin{savenotes}\begin{fulllineitems}
\phantomsection\label{\detokenize{eezz:eezz.websocket.TWebSocket}}
\pysigstartsignatures
\pysiglinewithargsret{\sphinxbfcode{\sphinxupquote{class\DUrole{w}{ }}}\sphinxcode{\sphinxupquote{eezz.websocket.}}\sphinxbfcode{\sphinxupquote{TWebSocket}}}{\sphinxparam{\DUrole{n}{a\_web\_address}}\sphinxparamcomma \sphinxparam{\DUrole{n}{a\_agent\_class}\DUrole{p}{:}\DUrole{w}{ }\DUrole{n}{type\DUrole{p}{{[}}{\hyperref[\detokenize{eezz:eezz.websocket.TWebSocketAgent}]{\sphinxcrossref{TWebSocketAgent}}}\DUrole{p}{{]}}}}}{}
\pysigstopsignatures
\sphinxAtStartPar
Bases: \sphinxcode{\sphinxupquote{Thread}}

\sphinxAtStartPar
Manage connections to the WEB socket interface
\index{run() (eezz.websocket.TWebSocket method)@\spxentry{run()}\spxextra{eezz.websocket.TWebSocket method}}

\begin{savenotes}\begin{fulllineitems}
\phantomsection\label{\detokenize{eezz:eezz.websocket.TWebSocket.run}}
\pysigstartsignatures
\pysiglinewithargsret{\sphinxbfcode{\sphinxupquote{run}}}{}{}
\pysigstopsignatures
\sphinxAtStartPar
Wait for incoming requests

\end{fulllineitems}\end{savenotes}

\index{shutdown() (eezz.websocket.TWebSocket method)@\spxentry{shutdown()}\spxextra{eezz.websocket.TWebSocket method}}

\begin{savenotes}\begin{fulllineitems}
\phantomsection\label{\detokenize{eezz:eezz.websocket.TWebSocket.shutdown}}
\pysigstartsignatures
\pysiglinewithargsret{\sphinxbfcode{\sphinxupquote{shutdown}}}{}{}
\pysigstopsignatures
\sphinxAtStartPar
Shutdown closes all sockets

\end{fulllineitems}\end{savenotes}


\end{fulllineitems}\end{savenotes}

\index{TWebSocketAgent (class in eezz.websocket)@\spxentry{TWebSocketAgent}\spxextra{class in eezz.websocket}}

\begin{savenotes}\begin{fulllineitems}
\phantomsection\label{\detokenize{eezz:eezz.websocket.TWebSocketAgent}}
\pysigstartsignatures
\pysigline{\sphinxbfcode{\sphinxupquote{class\DUrole{w}{ }}}\sphinxcode{\sphinxupquote{eezz.websocket.}}\sphinxbfcode{\sphinxupquote{TWebSocketAgent}}}
\pysigstopsignatures
\sphinxAtStartPar
Bases: \sphinxcode{\sphinxupquote{object}}

\sphinxAtStartPar
User has to implement this class to receive data.
TWebSocketClient is called with the class type, leaving the TWebSocketClient to generate an instance
\index{handle\_download() (eezz.websocket.TWebSocketAgent method)@\spxentry{handle\_download()}\spxextra{eezz.websocket.TWebSocketAgent method}}

\begin{savenotes}\begin{fulllineitems}
\phantomsection\label{\detokenize{eezz:eezz.websocket.TWebSocketAgent.handle_download}}
\pysigstartsignatures
\pysiglinewithargsret{\sphinxbfcode{\sphinxupquote{abstract\DUrole{w}{ }}}\sphinxbfcode{\sphinxupquote{handle\_download}}}{\sphinxparam{\DUrole{n}{description}\DUrole{p}{:}\DUrole{w}{ }\DUrole{n}{str}}\sphinxparamcomma \sphinxparam{\DUrole{n}{raw\_data}\DUrole{p}{:}\DUrole{w}{ }\DUrole{n}{Any}}}{{ $\rightarrow$ str}}
\pysigstopsignatures
\sphinxAtStartPar
handle download expects a json structure, describing the file and the data

\end{fulllineitems}\end{savenotes}

\index{handle\_request() (eezz.websocket.TWebSocketAgent method)@\spxentry{handle\_request()}\spxextra{eezz.websocket.TWebSocketAgent method}}

\begin{savenotes}\begin{fulllineitems}
\phantomsection\label{\detokenize{eezz:eezz.websocket.TWebSocketAgent.handle_request}}
\pysigstartsignatures
\pysiglinewithargsret{\sphinxbfcode{\sphinxupquote{abstract\DUrole{w}{ }}}\sphinxbfcode{\sphinxupquote{handle\_request}}}{\sphinxparam{\DUrole{n}{request\_data}\DUrole{p}{:}\DUrole{w}{ }\DUrole{n}{Any}}}{{ $\rightarrow$ str}}
\pysigstopsignatures
\sphinxAtStartPar
handle request expects a json structure

\end{fulllineitems}\end{savenotes}

\index{setup\_download() (eezz.websocket.TWebSocketAgent method)@\spxentry{setup\_download()}\spxextra{eezz.websocket.TWebSocketAgent method}}

\begin{savenotes}\begin{fulllineitems}
\phantomsection\label{\detokenize{eezz:eezz.websocket.TWebSocketAgent.setup_download}}
\pysigstartsignatures
\pysiglinewithargsret{\sphinxbfcode{\sphinxupquote{abstract\DUrole{w}{ }}}\sphinxbfcode{\sphinxupquote{setup\_download}}}{\sphinxparam{\DUrole{n}{request\_data}\DUrole{p}{:}\DUrole{w}{ }\DUrole{n}{dict}}}{{ $\rightarrow$ str}}
\pysigstopsignatures
\end{fulllineitems}\end{savenotes}

\index{shutdown() (eezz.websocket.TWebSocketAgent method)@\spxentry{shutdown()}\spxextra{eezz.websocket.TWebSocketAgent method}}

\begin{savenotes}\begin{fulllineitems}
\phantomsection\label{\detokenize{eezz:eezz.websocket.TWebSocketAgent.shutdown}}
\pysigstartsignatures
\pysiglinewithargsret{\sphinxbfcode{\sphinxupquote{shutdown}}}{}{}
\pysigstopsignatures
\sphinxAtStartPar
Implement shutdown to release allocated resources

\end{fulllineitems}\end{savenotes}


\end{fulllineitems}\end{savenotes}

\index{TWebSocketClient (class in eezz.websocket)@\spxentry{TWebSocketClient}\spxextra{class in eezz.websocket}}

\begin{savenotes}\begin{fulllineitems}
\phantomsection\label{\detokenize{eezz:eezz.websocket.TWebSocketClient}}
\pysigstartsignatures
\pysiglinewithargsret{\sphinxbfcode{\sphinxupquote{class\DUrole{w}{ }}}\sphinxcode{\sphinxupquote{eezz.websocket.}}\sphinxbfcode{\sphinxupquote{TWebSocketClient}}}{\sphinxparam{\DUrole{n}{a\_client\_addr}\DUrole{p}{:}\DUrole{w}{ }\DUrole{n}{tuple}}\sphinxparamcomma \sphinxparam{\DUrole{n}{a\_web\_addr}\DUrole{p}{:}\DUrole{w}{ }\DUrole{n}{int}}\sphinxparamcomma \sphinxparam{\DUrole{n}{a\_agent}\DUrole{p}{:}\DUrole{w}{ }\DUrole{n}{type\DUrole{p}{{[}}{\hyperref[\detokenize{eezz:eezz.websocket.TWebSocketAgent}]{\sphinxcrossref{TWebSocketAgent}}}\DUrole{p}{{]}}}}}{}
\pysigstopsignatures
\sphinxAtStartPar
Bases: \sphinxcode{\sphinxupquote{object}}

\sphinxAtStartPar
Implements a WEB socket service thread
\index{gen\_handshake() (eezz.websocket.TWebSocketClient method)@\spxentry{gen\_handshake()}\spxextra{eezz.websocket.TWebSocketClient method}}

\begin{savenotes}\begin{fulllineitems}
\phantomsection\label{\detokenize{eezz:eezz.websocket.TWebSocketClient.gen_handshake}}
\pysigstartsignatures
\pysiglinewithargsret{\sphinxbfcode{\sphinxupquote{gen\_handshake}}}{\sphinxparam{\DUrole{n}{a\_data}\DUrole{p}{:}\DUrole{w}{ }\DUrole{n}{str}}}{}
\pysigstopsignatures
\sphinxAtStartPar
Upgrade HTTP connection to WEB\sphinxhyphen{}socket
\begin{quote}\begin{description}
\sphinxlineitem{Parameters}
\sphinxAtStartPar
\sphinxstyleliteralstrong{\sphinxupquote{a\_data}} – Upgrade request data

\sphinxlineitem{Returns}
\sphinxAtStartPar


\end{description}\end{quote}

\end{fulllineitems}\end{savenotes}

\index{gen\_key() (eezz.websocket.TWebSocketClient method)@\spxentry{gen\_key()}\spxextra{eezz.websocket.TWebSocketClient method}}

\begin{savenotes}\begin{fulllineitems}
\phantomsection\label{\detokenize{eezz:eezz.websocket.TWebSocketClient.gen_key}}
\pysigstartsignatures
\pysiglinewithargsret{\sphinxbfcode{\sphinxupquote{gen\_key}}}{}{}
\pysigstopsignatures
\sphinxAtStartPar
Generates a key to establish a secure connection
\begin{quote}\begin{description}
\sphinxlineitem{Returns}
\sphinxAtStartPar
Base64 representation of the calculated hash

\end{description}\end{quote}

\end{fulllineitems}\end{savenotes}

\index{handle\_aync\_request() (eezz.websocket.TWebSocketClient method)@\spxentry{handle\_aync\_request()}\spxextra{eezz.websocket.TWebSocketClient method}}

\begin{savenotes}\begin{fulllineitems}
\phantomsection\label{\detokenize{eezz:eezz.websocket.TWebSocketClient.handle_aync_request}}
\pysigstartsignatures
\pysiglinewithargsret{\sphinxbfcode{\sphinxupquote{handle\_aync\_request}}}{\sphinxparam{\DUrole{n}{request}\DUrole{p}{:}\DUrole{w}{ }\DUrole{n}{dict}}}{{ $\rightarrow$ None}}
\pysigstopsignatures
\sphinxAtStartPar
This method is called after each method call request by user interface. The idea of an async call is,
that a user method is unpredictable long\sphinxhyphen{}lasting and could block the entire communication channel.
The environment takes care, that the same method is not executed as long as prior execution lasts.
\begin{quote}\begin{description}
\sphinxlineitem{Parameters}
\sphinxAtStartPar
\sphinxstyleliteralstrong{\sphinxupquote{request}} (\sphinxstyleliteralemphasis{\sphinxupquote{dict}}) – The original request to execute after EEZZ function call

\end{description}\end{quote}

\end{fulllineitems}\end{savenotes}

\index{handle\_request() (eezz.websocket.TWebSocketClient method)@\spxentry{handle\_request()}\spxextra{eezz.websocket.TWebSocketClient method}}

\begin{savenotes}\begin{fulllineitems}
\phantomsection\label{\detokenize{eezz:eezz.websocket.TWebSocketClient.handle_request}}
\pysigstartsignatures
\pysiglinewithargsret{\sphinxbfcode{\sphinxupquote{handle\_request}}}{}{{ $\rightarrow$ None}}
\pysigstopsignatures
\sphinxAtStartPar
Receives an request and send a response
The given method is executed async, so there will be no blocking calls. After the call the result
is collected.

\end{fulllineitems}\end{savenotes}

\index{read\_frame() (eezz.websocket.TWebSocketClient method)@\spxentry{read\_frame()}\spxextra{eezz.websocket.TWebSocketClient method}}

\begin{savenotes}\begin{fulllineitems}
\phantomsection\label{\detokenize{eezz:eezz.websocket.TWebSocketClient.read_frame}}
\pysigstartsignatures
\pysiglinewithargsret{\sphinxbfcode{\sphinxupquote{read\_frame}}}{\sphinxparam{\DUrole{n}{x\_opcode}}\sphinxparamcomma \sphinxparam{\DUrole{n}{a\_mask\_vector}}\sphinxparamcomma \sphinxparam{\DUrole{n}{a\_payload\_len}}}{}
\pysigstopsignatures
\sphinxAtStartPar
Read one frame
\begin{quote}\begin{description}
\sphinxlineitem{Parameters}\begin{itemize}
\item {} 
\sphinxAtStartPar
\sphinxstyleliteralstrong{\sphinxupquote{x\_opcode}} – The opcode describes the data type

\item {} 
\sphinxAtStartPar
\sphinxstyleliteralstrong{\sphinxupquote{a\_mask\_vector}} – The mask is used to decrypt and encrypt the data stream

\item {} 
\sphinxAtStartPar
\sphinxstyleliteralstrong{\sphinxupquote{a\_payload\_len}} – The length of the data block

\end{itemize}

\sphinxlineitem{Returns}
\sphinxAtStartPar
The buffer with the data

\end{description}\end{quote}

\end{fulllineitems}\end{savenotes}

\index{read\_frame\_header() (eezz.websocket.TWebSocketClient method)@\spxentry{read\_frame\_header()}\spxextra{eezz.websocket.TWebSocketClient method}}

\begin{savenotes}\begin{fulllineitems}
\phantomsection\label{\detokenize{eezz:eezz.websocket.TWebSocketClient.read_frame_header}}
\pysigstartsignatures
\pysiglinewithargsret{\sphinxbfcode{\sphinxupquote{read\_frame\_header}}}{}{}
\pysigstopsignatures
\sphinxAtStartPar
Interpret the incoming data stream, starting with analysis of the first bytes
\begin{quote}\begin{description}
\sphinxlineitem{Returns}
\sphinxAtStartPar
A tuple of all attributes, which enable the program to read the pay\sphinxhyphen{}load

\end{description}\end{quote}

\end{fulllineitems}\end{savenotes}

\index{read\_websocket() (eezz.websocket.TWebSocketClient method)@\spxentry{read\_websocket()}\spxextra{eezz.websocket.TWebSocketClient method}}

\begin{savenotes}\begin{fulllineitems}
\phantomsection\label{\detokenize{eezz:eezz.websocket.TWebSocketClient.read_websocket}}
\pysigstartsignatures
\pysiglinewithargsret{\sphinxbfcode{\sphinxupquote{read\_websocket}}}{}{{ $\rightarrow$ bytes}}
\pysigstopsignatures
\sphinxAtStartPar
Read a chunk of data from stream
\begin{quote}\begin{description}
\sphinxlineitem{Returns}
\sphinxAtStartPar
The chunk of data coming from browser

\end{description}\end{quote}

\end{fulllineitems}\end{savenotes}

\index{shutdown() (eezz.websocket.TWebSocketClient method)@\spxentry{shutdown()}\spxextra{eezz.websocket.TWebSocketClient method}}

\begin{savenotes}\begin{fulllineitems}
\phantomsection\label{\detokenize{eezz:eezz.websocket.TWebSocketClient.shutdown}}
\pysigstartsignatures
\pysiglinewithargsret{\sphinxbfcode{\sphinxupquote{shutdown}}}{}{}
\pysigstopsignatures
\end{fulllineitems}\end{savenotes}

\index{upgrade() (eezz.websocket.TWebSocketClient method)@\spxentry{upgrade()}\spxextra{eezz.websocket.TWebSocketClient method}}

\begin{savenotes}\begin{fulllineitems}
\phantomsection\label{\detokenize{eezz:eezz.websocket.TWebSocketClient.upgrade}}
\pysigstartsignatures
\pysiglinewithargsret{\sphinxbfcode{\sphinxupquote{upgrade}}}{}{}
\pysigstopsignatures
\sphinxAtStartPar
Upgrade HTTP connection to WEB socket

\end{fulllineitems}\end{savenotes}

\index{write\_frame() (eezz.websocket.TWebSocketClient method)@\spxentry{write\_frame()}\spxextra{eezz.websocket.TWebSocketClient method}}

\begin{savenotes}\begin{fulllineitems}
\phantomsection\label{\detokenize{eezz:eezz.websocket.TWebSocketClient.write_frame}}
\pysigstartsignatures
\pysiglinewithargsret{\sphinxbfcode{\sphinxupquote{write\_frame}}}{\sphinxparam{\DUrole{n}{a\_data}\DUrole{p}{:}\DUrole{w}{ }\DUrole{n}{bytes}}\sphinxparamcomma \sphinxparam{\DUrole{n}{a\_opcode}\DUrole{p}{:}\DUrole{w}{ }\DUrole{n}{hex}\DUrole{w}{ }\DUrole{o}{=}\DUrole{w}{ }\DUrole{default_value}{1}}\sphinxparamcomma \sphinxparam{\DUrole{n}{a\_final}\DUrole{p}{:}\DUrole{w}{ }\DUrole{n}{hex}\DUrole{w}{ }\DUrole{o}{=}\DUrole{w}{ }\DUrole{default_value}{128}}\sphinxparamcomma \sphinxparam{\DUrole{n}{a\_mask\_vector}\DUrole{p}{:}\DUrole{w}{ }\DUrole{n}{list}\DUrole{w}{ }\DUrole{o}{=}\DUrole{w}{ }\DUrole{default_value}{None}}}{{ $\rightarrow$ None}}
\pysigstopsignatures
\sphinxAtStartPar
Write single frame
\begin{quote}\begin{description}
\sphinxlineitem{Parameters}\begin{itemize}
\item {} 
\sphinxAtStartPar
\sphinxstyleliteralstrong{\sphinxupquote{a\_data}} – Data to send to browser

\item {} 
\sphinxAtStartPar
\sphinxstyleliteralstrong{\sphinxupquote{a\_opcode}} – Opcode defines the kind of data

\item {} 
\sphinxAtStartPar
\sphinxstyleliteralstrong{\sphinxupquote{a\_final}} – Bool set to True, if all data are written to stream

\item {} 
\sphinxAtStartPar
\sphinxstyleliteralstrong{\sphinxupquote{a\_mask\_vector}} – Mask to use for secure communication

\end{itemize}

\sphinxlineitem{Returns}
\sphinxAtStartPar
None

\end{description}\end{quote}

\end{fulllineitems}\end{savenotes}


\end{fulllineitems}\end{savenotes}

\index{TWebSocketException@\spxentry{TWebSocketException}}

\begin{savenotes}\begin{fulllineitems}
\phantomsection\label{\detokenize{eezz:eezz.websocket.TWebSocketException}}
\pysigstartsignatures
\pysiglinewithargsret{\sphinxbfcode{\sphinxupquote{exception\DUrole{w}{ }}}\sphinxcode{\sphinxupquote{eezz.websocket.}}\sphinxbfcode{\sphinxupquote{TWebSocketException}}}{\sphinxparam{\DUrole{n}{a\_value}}}{}
\pysigstopsignatures
\sphinxAtStartPar
Bases: \sphinxcode{\sphinxupquote{Exception}}

\sphinxAtStartPar
Exception class for this module

\end{fulllineitems}\end{savenotes}

\index{TestSocketServer (class in eezz.websocket)@\spxentry{TestSocketServer}\spxextra{class in eezz.websocket}}

\begin{savenotes}\begin{fulllineitems}
\phantomsection\label{\detokenize{eezz:eezz.websocket.TestSocketServer}}
\pysigstartsignatures
\pysiglinewithargsret{\sphinxbfcode{\sphinxupquote{class\DUrole{w}{ }}}\sphinxcode{\sphinxupquote{eezz.websocket.}}\sphinxbfcode{\sphinxupquote{TestSocketServer}}}{\sphinxparam{\DUrole{n}{a\_client\_addr}\DUrole{p}{:}\DUrole{w}{ }\DUrole{n}{tuple}}\sphinxparamcomma \sphinxparam{\DUrole{n}{a\_web\_addr}\DUrole{p}{:}\DUrole{w}{ }\DUrole{n}{int}}\sphinxparamcomma \sphinxparam{\DUrole{n}{a\_agent}\DUrole{p}{:}\DUrole{w}{ }\DUrole{n}{type\DUrole{p}{{[}}{\hyperref[\detokenize{eezz:eezz.websocket.TWebSocketAgent}]{\sphinxcrossref{TWebSocketAgent}}}\DUrole{p}{{]}}}}}{}
\pysigstopsignatures
\sphinxAtStartPar
Bases: {\hyperref[\detokenize{eezz:eezz.websocket.TWebSocketClient}]{\sphinxcrossref{\sphinxcode{\sphinxupquote{TWebSocketClient}}}}}

\sphinxAtStartPar
Simulate a request handler, waiting for a method to finish
\index{handle\_aync\_request() (eezz.websocket.TestSocketServer method)@\spxentry{handle\_aync\_request()}\spxextra{eezz.websocket.TestSocketServer method}}

\begin{savenotes}\begin{fulllineitems}
\phantomsection\label{\detokenize{eezz:eezz.websocket.TestSocketServer.handle_aync_request}}
\pysigstartsignatures
\pysiglinewithargsret{\sphinxbfcode{\sphinxupquote{handle\_aync\_request}}}{\sphinxparam{\DUrole{n}{request}\DUrole{p}{:}\DUrole{w}{ }\DUrole{n}{dict}}}{}
\pysigstopsignatures
\sphinxAtStartPar
This method is called after each method call request by user interface. The idea of an async call is,
that a user method is unpredictable long\sphinxhyphen{}lasting and could block the entire communication channel.
The environment takes care, that the same method is not executed as long as prior execution lasts.
\begin{quote}\begin{description}
\sphinxlineitem{Parameters}
\sphinxAtStartPar
\sphinxstyleliteralstrong{\sphinxupquote{request}} (\sphinxstyleliteralemphasis{\sphinxupquote{dict}}) – The original request to execute after EEZZ function call

\end{description}\end{quote}

\end{fulllineitems}\end{savenotes}


\end{fulllineitems}\end{savenotes}

\index{test\_async\_hadler() (in module eezz.websocket)@\spxentry{test\_async\_hadler()}\spxextra{in module eezz.websocket}}

\begin{savenotes}\begin{fulllineitems}
\phantomsection\label{\detokenize{eezz:eezz.websocket.test_async_hadler}}
\pysigstartsignatures
\pysiglinewithargsret{\sphinxcode{\sphinxupquote{eezz.websocket.}}\sphinxbfcode{\sphinxupquote{test\_async\_hadler}}}{}{}
\pysigstopsignatures
\sphinxAtStartPar
Test for TAsyncHandler thread

\end{fulllineitems}\end{savenotes}

\index{test\_tcm() (in module eezz.websocket)@\spxentry{test\_tcm()}\spxextra{in module eezz.websocket}}

\begin{savenotes}\begin{fulllineitems}
\phantomsection\label{\detokenize{eezz:eezz.websocket.test_tcm}}
\pysigstartsignatures
\pysiglinewithargsret{\sphinxcode{\sphinxupquote{eezz.websocket.}}\sphinxbfcode{\sphinxupquote{test\_tcm}}}{}{}
\pysigstopsignatures
\sphinxAtStartPar
simulate a time consuming method (tcm)

\end{fulllineitems}\end{savenotes}



\subsection{Module contents}
\label{\detokenize{eezz:module-eezz}}\label{\detokenize{eezz:module-contents}}\index{module@\spxentry{module}!eezz@\spxentry{eezz}}\index{eezz@\spxentry{eezz}!module@\spxentry{module}}

\chapter{Indices and tables}
\label{\detokenize{index:indices-and-tables}}\begin{itemize}
\item {} 
\sphinxAtStartPar
\DUrole{xref,std,std-ref}{genindex}

\item {} 
\sphinxAtStartPar
\DUrole{xref,std,std-ref}{modindex}

\item {} 
\sphinxAtStartPar
\DUrole{xref,std,std-ref}{search}

\end{itemize}


\renewcommand{\indexname}{Python Module Index}
\begin{sphinxtheindex}
\let\bigletter\sphinxstyleindexlettergroup
\bigletter{e}
\item\relax\sphinxstyleindexentry{eezz}\sphinxstyleindexpageref{eezz:\detokenize{module-eezz}}
\item\relax\sphinxstyleindexentry{eezz.blueserv}\sphinxstyleindexpageref{eezz:\detokenize{module-eezz.blueserv}}
\item\relax\sphinxstyleindexentry{eezz.database}\sphinxstyleindexpageref{eezz:\detokenize{module-eezz.database}}
\item\relax\sphinxstyleindexentry{eezz.document}\sphinxstyleindexpageref{eezz:\detokenize{module-eezz.document}}
\item\relax\sphinxstyleindexentry{eezz.filesrv}\sphinxstyleindexpageref{eezz:\detokenize{module-eezz.filesrv}}
\item\relax\sphinxstyleindexentry{eezz.http\_agent}\sphinxstyleindexpageref{eezz:\detokenize{module-eezz.http_agent}}
\item\relax\sphinxstyleindexentry{eezz.seccom}\sphinxstyleindexpageref{eezz:\detokenize{module-eezz.seccom}}
\item\relax\sphinxstyleindexentry{eezz.server}\sphinxstyleindexpageref{eezz:\detokenize{module-eezz.server}}
\item\relax\sphinxstyleindexentry{eezz.service}\sphinxstyleindexpageref{eezz:\detokenize{module-eezz.service}}
\item\relax\sphinxstyleindexentry{eezz.session}\sphinxstyleindexpageref{eezz:\detokenize{module-eezz.session}}
\item\relax\sphinxstyleindexentry{eezz.table}\sphinxstyleindexpageref{eezz:\detokenize{module-eezz.table}}
\item\relax\sphinxstyleindexentry{eezz.websocket}\sphinxstyleindexpageref{eezz:\detokenize{module-eezz.websocket}}
\end{sphinxtheindex}

\renewcommand{\indexname}{Index}
\footnotesize\raggedright\printindex
\end{document}